\newpage
\section{Abelian categories}
\label{sec:abelian}
\begin{definition}
  \label{def:abelian_category}
  A category \(\cat{C}\) is {\bf abelian} if:
  \begin{enumerate}[label=(\arabic*)]
  \item has a zero object \({\bf 0}\),
  \item every pair of objects of \(\cat{C}\) has product and coproduct,
  \item every arrow in \(\cat{C}\) has a kernel and a cokernel,
  \item every mono in \(\cat{C}\) is the kernel of some arrow and every epi in \(\cat{C}\) is the cokernel of some arrow.
  \end{enumerate}
\end{definition}

\begin{remark}
  The notion of abelian category is autodual.
\end{remark}

\begin{example}
  \label{ex:ab_is_abelian}
  \(\catname{Ab}\), the category \(\catname{Mod}_R\) of left R-modules over a ring \(R\) (as well as the category of right-modules over \(R\)) are abelian categories. Indeed the goal behind Definition \ref{def:abelian_category} is that of capturing the fundamental properties of \(\catname{Ab}\) and, more generally, \(\catname{Mod}_R\).
\end{example}

\begin{proposition}
  \label{prop:iso_mono_epi}
  In an abelian category \(\cat{C}\) an arrow \(f\) is an isomorphism if and only if is both epic and monic.
\end{proposition}

\begin{proof}
  If \(f\) is an isomorphism then \(f\) is both split monic and split epic and thus both monic and epic. On the other hand suppose \(f\) is both monic and epic; since \(\cat{C}\) is abelian there is an arrow \(g\) such that \(f = \ker(g)\). From this last fact we get \(g\circ f = 0\) and so \(g = 0\) because \(f\) is also epic. Finally we have that \(f\) is the kernel of a zero arrow and thus by Proposition \ref{prop:kernel_of_zero_arrow} is an isomorphism.
\end{proof}

\begin{remark}
  The category \(\catname{Ring}\) of rings and rings homomorphisms is not abelian since the inclusion of \(\mathbb{Z}\) in \(\mathbb{Q}\) is both epic and monic, but clearly not an isomorphism.
\end{remark}

\begin{proposition}
  \label{prop:existence_of_intersection }
  In an abelian category \(\cat{C}\) given two monomorphisms \(a,b\) with codomain \(C\) their pullback always exists.
\end{proposition}

\begin{figure}
  \begin{center}
  \begin{tikzcd}[sep=huge]
    X \ar[dr, dashed, "w"] \ar[drr, "v", bend left] \ar[ddr, "u"', bend right] & & &\\
    & \ker\vvectormini{f}{g} \ar[d, dashed, "a'"'] \ar[r, dashed, "b'"] \ar[dr, tail, "k"] & B \ar[d, "b"] &\\
    & A \ar[r, "a"'] & C \ar[r, "f"] \ar[d, "g"'] \ar[dr, "\vvectormini{f}{g}"]& D\\
    & & E & D\times E \ar[u, "p_D"'] \ar[l, "p_E"]
  \end{tikzcd}
\end{center}


  \caption{}
  \label{diagram:intersection}
\end{figure}

\begin{proof}
  With reference to Diagram \ref{diagram:intersection} consider two monomorphisms \(a\colon A\rightarrowtail C,b\colon B\rightarrowtail C\); since \(\cat{C}\) is abelian there are morphisms \(f\colon C\to D,g\colon C\to E\) such that \(a=\ker(f)\) and \(b=\ker(g)\); let's also consider the arrow \(k=\ker\vvectormini{f}{g}\). We have that \(f\circ k = p_D\circ\vvectormini{f}{g}\circ k = 0\) so \(k\) factors as \(k = a\circ a'\) because \(a=\ker(f)\); in the same way we obtain \(k = b'\circ b\). We now want to prove that \(\ker\vvectormini{f}{g}\) together with the arrows \(a'\) and \(b'\) is the pullback of the pair \((a, b)\). Given arrows \(u,v\) such that \(a\circ u = b\circ v\) we get \(f\circ b\circ v = f\circ a\circ u = 0\) and since \(a=\ker(f)\) we obtain that there is a unique \(w\) such that \(k\circ w = b\circ v\). Now \(b\circ v = k\circ w = b\circ b'\circ w\) implies \(b'\circ w = v\) since \(b\) is monic. Similarly one has \(a\circ u = b\circ v = k\circ w = a\circ a' \circ w\) and thus \(u=a'\circ w\) because \(a\) is also monic. To prove the uniqueness of this factorization suppose that \(w'\) is an arrow such that \(u = a'\circ w'\) and \(v = b'\circ w'\), using the factorization through \(w\) we have:
  \begin{align*}
    k\circ w' = a\circ a'\circ w' = a\circ u = a\circ a'\circ w = k\circ w
  \end{align*}
  and thus \(w = w'\) because \(k\) is monic.
\end{proof}

\begin{proposition}
  \label{prop:completeness}
  An abelian category \(\cat{C}\) is finitely complete and finitely cocomplete.
\end{proposition}

\begin{proof}
  By duality it suffice to prove that \(\cat{C}\) is finitely complete. We know from  \cite[Proposition 2.8.2]{handbook1} that a category is finitely complete if and only if it has a terminal object, all binary products and all equalizers. \(\cat{C}\) is abelian and thus we just need to prove that it has all the equalizers.

  Let \(f,g:A\to B\) be arrows in \(\cat{C}\); we consider \(\vvectormini{1_A}{f}\), \(\vvectormini{1_A}{g}\), that are monic since they compose with \(p_A\) to give \(1_A\), and their pullback \((P, u, v)\). We have the following:
  \begin{gather*}
    u = 1_A\circ u = p_A\circ\vvector{1_A}{f}\circ u = p_A\circ\vvector{1_A}{g}\circ v = 1_A\circ v = v\\
    f\circ u = p_B\circ\vvector{1_A}{f}\circ u = p_B\circ\vvector{1_A}{g}\circ v = g\circ v = g\circ u
  \end{gather*}
  so \(u = v\) and \(f\circ u =g\circ u\). We now want to prove that \(u\) is really the equalizer of \(f\) and \(g\). With reference to diagram \ref{diagram:equalizers} suppose \(x:X\to A\) is an arrow such that \(f\circ x = g\circ x\); we consider the cone \((X,x,x)\) and thus obtain a unique factorization of \(x\) through \(u\). This proves that \(u=\ker(f, g)\).
  \begin{figure}[h]
    \begin{center}
  \begin{tikzcd}[sep=huge]
    X \ar[drr, "x", bend left] \ar[ddr, "x"', bend right] \ar[dr, dashed, ""] & & \\
    & P \ar[r, tail, "v"] \ar[d, tail, "u"'] & A \ar[d, tail, "\vvectormini{1_A}{g}"] \\
    & A \ar[r, tail, "\vvectormini{1_A}{f}"'] & A\times B
  \end{tikzcd}
\end{center}

    \caption{}
    \label{diagram:equalizers}
  \end{figure}
\end{proof}

\begin{proposition}
  \label{prop:mono_characterization}
  For a morphism \(f\colon A\to B\) in an abelian category \(\cat{C}\) the following conditions are equivalent:
  \begin{enumerate}[label=(\arabic*)]
  \item \(f\) is monic,
  \item \(\ker(f)=0\),
  \item for every \(g:C\to A\) we have \(f\circ g = 0\Rightarrow g = 0\).
  \end{enumerate}
\end{proposition}

\begin{proof}
  \((1)\Rightarrow(2)\) is exactly Proposition \ref{prop:the_ker_of_a_mono_is_zero} while \((1)\Rightarrow(3)\) is exactly Proposition \ref{prop:precomposition_with_mono}.

  To prove that \((2)\) implies \((1)\) we consider \(u,v\) such that \(f\circ u = f\circ v\) and prove that \(u = v\); we refer to Diagram \ref{diagram:mono}. Let \(q=\coker(u,v)\) (since \(\cat{C}\) is finitely cocomplete for \ref{prop:completeness}), \(q\) is epic and so there is an arrow \(w\) such that \(q = \coker(w)\). Since \(f\circ u = f\circ v\) then \(f\) factors uniquely as \(f = m\circ q\) through \(q = \coker(u, v)\). We now have \(f\circ w = m\circ q\circ w = m\circ 0 = 0\) so \(w\) factors uniquely as \(w=k\circ n\) through \(k = \ker(f)\); but we have that \(\ker(f) = 0\) so \(w = 0\) and finally since \(q = \coker(0)\), by Proposition \ref{prop:kernel_of_zero_arrow}, \(q\) is an isomorphism. Particularly \(q\) is monic and thus \(q\circ u =q\circ v\Rightarrow u = v\).
  
  To prove that \((3)\) implies \((2)\) we start with the trivial observation that \(f\circ 0_{{\bf 0}\to A} = 0_{{\bf 0}\to B}\) where \(f\colon A\to B\). Now if \(g\colon C\to A\) is such that \(f\circ g = 0_{C\to B}\) then by \((3)\) \(g = 0_{C\to A}\) and thus factors uniquely through \(0_{{\bf 0}\to A}\). This proves that \(\ker(f) = 0_{{\bf 0}\to A}\).
  \begin{figure}[h]
    \begin{center}
  \begin{tikzcd}[sep=huge]
    \ker(f) \ar[dr, "k"] & \bullet \ar[l, dashed, "n"'] \ar[d, "w"] & \\
    \bullet \ar[r, shift left = .5ex, "u"] \ar[r, shift right = .5ex, "v"'] & \bullet \ar[d, two heads, "q"']  \ar[r, "f"] & \bullet \\
    & \bullet \ar[ur, dashed, "m"'] & 
  \end{tikzcd}
\end{center}
    \caption{}
    \label{diagram:mono}
  \end{figure}
\end{proof}

\begin{theorem}[epi-mono factorization]
  \label{teo:epi_mono_factorization}
  Every morphism \(f\) in an abelian category can be factored uniquely (up to isomorphism) as \(f = i\circ p\) with \(i\) monic and \(p\) epic. Moreover we have that \(i = \ker(\coker(f))\) and \(p = \coker(\ker(f))\).
\end{theorem}

\begin{figure}[h]
  \begin{center}
  \begin{tikzcd}[sep=huge]
    & \bullet \ar[dl, "l"'] \ar[d, "h"] & \\
    \bullet \ar[r, "k"] & \bullet \ar[r, "f"] \ar[d, "p"'] & \bullet \\p
    \bullet \ar[r, "x"] & \bullet \ar[r, shift right = .5ex, "q"'] \ar[ur, "i"] & \bullet \ar[u, "r"'] \ar[l, shift right = .5ex, "s"']
  \end{tikzcd}
\end{center}
  \caption{}
  \label{diagram:factorization}
\end{figure}

\begin{proof}
  With reference to diagram \ref{diagram:factorization} we define \(k = \ker(f)\) and \(p = \coker(k)\). Since \(f\circ k = 0\), because \(k = \ker(f)\), \(f\) must factor as \(f = i\circ p\) through the cokernel \(p\) of \(k\). We now wish to prove that \(i\) is monic.

  Let \(x\) be an arrow such that \(i\circ x = 0\); if we can prove that \(x = 0\) then, by Proposition \ref{prop:mono_characterization}, \(i\) must be monic. Now \(i\circ x = 0\) immediately gives us the factorization \(i = r\circ q\) through \(q = \coker(x)\). Moreover \(q\circ p\) is the composition of two epimorphisms and thus is again an epimorphism, since we're in an abelian category there exists an arrow \(h\) such that \(q\circ p = \coker(h)\). From
  \begin{align*}
    f\circ h = i\circ p\circ h = r\circ q\circ p\circ h = r\circ 0 = 0
  \end{align*}
  we obtain that \(h\) factors as \(h = k\circ l\) since \(k = \ker(f)\). Now \(p\circ h = p\circ k\circ l = 0\circ l = 0\) since \(p=\coker(k)\) and thus \(p\) factors uniquely through \(\coker(h) = q\circ p\) and we get \(p = s\circ(q\circ p)\). From this last relation and the fact that \(p\) is epic we obtain \(s\circ q = 1\) thus \(q\) is a monomorphism so from \(q\circ x = 0\) we obtain \(x = 0\).

  The uniqueness of the factorization \(f = i\circ p\) follows from \cite[4.4.5]{handbook1} and \cite[4.3.6]{handbook1}. By duality \(f\) can be factored also as \(f = i'\circ p'\) where we have that \(i' = \ker(\coker(f))\) and that \(p'\) is epi. The uniqueness states that those two factorizations are isomorphic and this concludes the proof.
\end{proof}

\begin{corollary}
  \label{cor:mono_kernel_cokernel_epi_cokernel_kernel}
  In an abelian category the following hold:
  \begin{enumerate}[label=(\arabic*)]
  \item every monomorphism is the kernel of its cokernel,
  \item every epimorphism is the cokernel of its kernel.  
  \end{enumerate}
\end{corollary}

\begin{proof}
  Let \(f\) be an epimorphism; by Theorem \ref{teo:epi_mono_factorization} it can be factored as \(f=i\circ p\) with \(i\) monomorphism and \(p = \coker(\ker(f))\). Now since \(f\) is epic \(i\) must be too and thus \(i\) is an isomorphism by Proposition \ref{prop:iso_mono_epi}; finally we have \(f= \coker(\ker(f))\). By duality we obtain (1). 
\end{proof}

\begin{proposition}
  \label{prop:pullback_of_epi}
  In an abelian category \(\cat{C}\) monomorphisms are pushout-stable i.e. the pushout of a monic arrow is again a monic arrow. By duality epimorphisms are pullback-stable i.e. the pullback of an epi arrow is again an epi arrow.
\end{proposition}

\begin{figure}
  \begin{center}
  \begin{tikzcd}[sep=huge]
    A \ar[rr, "f"] \ar[dd, tail, "g"] \ar[dr, "l"] & & B \ar[dd, "k"] \ar[dl, shift left = 1, "s_B"]\\
    & B\oplus C \ar[dl, shift left = 1, "p_C"] \ar[rd, two heads, "p"] \ar[ru, shift left = 1, "p_S"] & \\
    C \ar[ru, shift left = 1, "s_C"] \ar[rr, "h"] & & D
  \end{tikzcd}
\end{center}

  \caption{}
  \label{diagram:pushepi}
\end{figure}

\begin{proof}
  As in Diagram \ref{diagram:pushepi} let \(g\) be mono and \(f\) an arbitrary arrow. Considering the biproduct \(B\oplus C\) we define
  \begin{equation*}
    l = \vvector{-f}{g} = 1_{B\oplus C}\circ\vvector{-f}{g} = \hvector{s_B}{s_C}\circ\vvector{-f}{g} = -(s_B\circ f) + (s_C\circ g).
  \end{equation*}
  We also set \(p = \coker(l)\), \(h = p\circ s_C\) and \(k = p\circ s_B\). We shall prove that the pair \((h, k)\) is the pushout of \((g, f)\). First we observe that
  \begin{align*}
    (h\circ g) - (k\circ f) &= (p\circ s_C \circ g) - (p\circ s_B\circ f)\\
                            &= p\circ((s_C\circ g) - (s_B\circ f))\\
                            &= p\circ l = 0
  \end{align*}
  and thus \(h\circ g = k\circ f\). Now let \(\fun{m}{B}{M}\) and \(\fun{n}{C}{M}\) be arrows such that \(m\circ f = n\circ g\) and consider the codiagonal morphism \(\fun{\hvectormini{m}{n}}{B\oplus C}{M}\); we have that
  \begin{align*}
    \hvector{m}{n}\circ l = \hvector{m}{n}\circ\vvector{-f}{g} = -(m\circ f) + (n\circ g) = 0
  \end{align*}
  and thus, since \(p\) is the cokernel of \(l\), \(\hvectormini{m}{n} = d\circ p\) for a unique arrow \(\fun{d}{D}{M}\). But now
  \begin{align*}
    d\circ k &= d\circ p\circ s_B = \hvectormini{m}{n}\circ s_B = m\\
    d\circ h &= d\circ p\circ s_C = \hvectormini{m}{n}\circ s_C = n
  \end{align*}
  and so \(d\) is the unique arrow that factorizes the cocone \((m, n)\). We finally have that \((k, h)\) is the pushout of \((g, f)\).

  Now to prove that \(k\) is mono we shall first prove that \(l\) is. Let \(\fun{x}{X}{A}\) be an arrow such that \(l\circ x = 0\). We immediately have that
  \begin{align*}
    0 = p_C\circ 0 &= p_C\circ l\circ x\\
                   &= p_C\circ(-(s_B\circ f) + (s_C\circ g))\circ x\\
                   &= g\circ x
  \end{align*}
  and thus \(x = 0\) because \(g\) is monic. By Proposition \ref{prop:mono_characterization} we confirm that \(l\) is mono and thus \(l = \ker(\coker(l)) = \ker(p)\) by Corollary \ref{cor:mono_kernel_cokernel_epi_cokernel_kernel}.

  Finally let \(\fun{y}{Y}{B}\) be an arrow such that \(k\circ y = 0\). We have that \(p\circ s_B\circ y = k\circ y = 0\) by our definition of \(p\) and thus \(s_B\circ y = l\circ z\) for a unique \(\fun{z}{Y}{A}\) since \(l\) is the kernel of \(p\). Now
  \begin{equation*}
    g\circ z = p_C\circ\vvector{-f}{g}\circ z = p_C\circ l\circ z = p_C\circ s_B\circ y = 0
  \end{equation*}
  and thus \(z = 0\) because \(g\) is a monomorphism by assumption. Now since \(s_B\) is mono and \(s_B\circ y = l\circ z = 0\) we obtain \(y = 0\) and finally, by Proposition \ref{prop:mono_characterization} again, that \(k\) is mono.
\end{proof}

\begin{lemma}
  \label{lemma:difference_lemma}
  Let \(\cat{C}\) be an abelian category, \(A\in\cat{C}\) one of its objects, \(\Delta = \vvectormini{1_A}{1_A}\colon A\to A\times A\) and \(q = \coker(\Delta)\colon A\times A\to Q\); then \(A\cong Q\).
\end{lemma}

\begin{figure}[h]
  \begin{center}
  \begin{tikzcd}[sep=huge]
    X \ar[r, "x"] \ar[d, "y"] & A \ar[d, tail, "\vvectormini{1_A}{0}"'] \ar[dr, "r" near end] & V \ar[l, "p_1\circ v"'] \ar[dl, "v"' near end]\\
    A \ar[d, "1_A"] \ar[dr, "1_A" near end] \ar[r, tail, "\Delta"] & A\times A \ar[r, two heads, "q"] \ar[ld, "p_1" near end] \ar[d, two heads, "p_2"] & Q \ar[d, "z"]\\
    A & A \ar[r, "t"] & Y
  \end{tikzcd}
\end{center}
  \caption{}
  \label{diagram:lemma1}
\end{figure}

\begin{proof}
  With reference to diagram \ref{diagram:lemma1} we define \(r = q\circ\vvectormini{1_A}{0}\) and prove that it is an isomorphism.

  Since \(\Delta\) is monic \(\Delta = \ker(\coker(\Delta)) = \ker(q)\). Via Notation \ref{not:matrix_product} we have \(p_1\circ\vvectormini{1_A}{0} = 1_A\) and \(p_2\circ\vvectormini{0}{1_A} = 1_A\) so \(p_1\) and \(p_2\) are epic while \(\vvectormini{1_A}{0}\) and \(\vvectormini{0}{1_A}\) are monic. Again via Notation \ref{not:matrix_product} we have \(p_2\circ\vvectormini{1_A}{0} = 0\) and, if \(p_2\circ v = 0\) for some arrow \(v\), then \(\vvectormini{1_A}{0}\circ(p_1\circ v) = v\) because:
  \begin{gather*}
    p_1\circ\vvector{1_A}{0}\circ(p_1\circ v) = p_1\circ v\\
    p_2\circ\vvector{1_A}{0}\circ(p_1\circ v) = 0\circ(p_1\circ v) = 0 = p_2\circ v
  \end{gather*}
  and this factorization is unique because \(\vvectormini{1_A}{0}\) is monic. With this we have established that \(\vvectormini{1_A}{0} = \ker(p_2)\) and with an analogue argument that \(\vvectormini{0}{1_A} = \ker(p_1)\). Finally since \(p_1\) is epic \(p_1 = \coker(\ker(p_1)) = \coker\vvectormini{0}{1_A}\) and similarly \(p_2 = \coker(\ker(p_2)) = \coker\vvectormini{1_A}{0}\).

  We prove that \(r\) is monic by Proposition \ref{prop:mono_characterization}. Let \(x\) be an arrow such that \(r\circ x = 0\). \(0 = r\circ x = q\circ\vvectormini{1_A}{0}\circ x\), \(\Delta = \ker(q)\) so we have the factorization \(\vvectormini{1_A}{0}\circ x =\Delta\circ y\); now \(y = p_2\circ\Delta\circ y = p_2\circ\vvectormini{1_A}{0}\circ x = 0\) so since \(\vvectormini{1_A}{0}\) is monic we have that \(x = 0\) and so \(r\) is monic.

  We prove that \(r\) is epic by the dual of Proposition \ref{prop:mono_characterization}. Let \(z\) be an arrow such that \(z\circ r = 0\). \(0 = z\circ r = z\circ q\circ\vvectormini{1_A}{0}\), \(p_2=\coker\vvectormini{1_A}{0}\) so we have the factorization \(z\circ q = t\circ p_2\); we now have \(t = t\circ p_2\circ\Delta = z\circ q\circ\Delta = z\circ 0 = 0\) because \(\Delta = \ker(q)\). Finally \(z\circ q = t\circ p_2 = 0\circ p_2 = 0\) and since \(q\) is epic \(z = 0\); this proves that \(r\) is epic and by Proposition \ref{prop:iso_mono_epi} that it is an isomorphism.
\end{proof}

\begin{definition}
  \label{def:difference_on_abelian_categories}
  Consider two arrows \(f,g\colon B\to A\) in an abelian category \(\cat{C}\). Using the notation of Lemma \ref{lemma:difference_lemma} we define a new arrow
  \begin{align*}
    \sigma_A\colon A\times A\overset{q}{\longrightarrow}Q\overset{r^{-1}}{\longrightarrow}A
  \end{align*}
  and call it {\bf difference} on \(A\). This arrow allows us to define the difference of two morphisms \(f,g\colon B\to A\) in the following way
  \begin{align*}
    f-g\colon B\overset{\vvectormini{f}{g}}{\longrightarrow} A\times A\overset{\sigma_A}{\longrightarrow} A
  \end{align*}
  And now the sum of two morphisms can be defined as \(f + g = f - (0 - g)\).
\end{definition}

\begin{remark}
  \label{rem:additive_structure}
  Notice that this last definition is not arbitrary but mirrors exactly what happens in an additive category; see proof of Proposition \ref{prop:uniqueness_of_additive_structure}, marked relation.
\end{remark}

\begin{remark}
  \label{remark:product_of_morphisms}
  Given an arrow \(f\colon A\to B\) we define \(f\times f = \vvectormini{f\circ p_1}{f\circ p_2}\colon A\times A\to B\times B\) where \(p_1\) and \(p_2\) are the projections of the product \(A\times A\). With this notation, via Remark \ref{remark:product_property}, we obtain the following:
  \begin{enumerate}[label=(\arabic*)]
  \item given \(f,g\colon A\to B\) and \(h\colon B\to C\) then \((h\times h)\circ\vvectormini{f}{g} = \vvectormini{h\circ f}{h\circ g}\),
  \item given \(f,g\colon A\to B\) and \(h\colon C\to A\) then \(\vvectormini{f}{g}\circ h = \vvectormini{f\circ h}{g\circ h}\).
  \end{enumerate}
  These two facts will be used to prove the next two results.
\end{remark}

\begin{lemma}
  \label{lemma:sigma_lemma}
  In an abelian category if \(f\colon B\to A\) then \(f\circ\sigma_B = \sigma_A\circ(f\times f)\).
\end{lemma}

\begin{figure}[h]
  \begin{center}
  \begin{tikzcd}[sep=huge]
    B \ar[d, tail, "\Delta_B"'] \ar[rrr, "f"] & & & A \ar[d, tail, "\Delta_A"]\\
    B\times B \ar[rrr, "f\times f"] \ar[dr, "\sigma_B"] & & & A\times A \ar[dl, "\sigma_A"']\\
    & B \ar[r, dashed, "g"] & A & \\
    B \ar[uu, "\vvectormini{1_B}{0}"] \ar[ur, "1_B"] \ar[rrr, "f"] & & & A \ar[uu, "\vvectormini{1_A}{0}"'] \ar[ul, "1_A"']
  \end{tikzcd}
\end{center}
  \caption{}
  \label{diagram:sigma}
\end{figure}

\begin{proof}
  We reference diagram \ref{diagram:sigma} and start with the observation that:
  \begin{align*}
    \sigma_A\circ(f\times f)\circ\Delta_B = \sigma_A\circ\Delta_A\circ f = 0\circ f = 0
  \end{align*}
  because \(\sigma_A\cong\coker(\Delta_A)\) from definition \ref{def:difference_on_abelian_categories}. This gives us the unique factorization \(\sigma_A\circ(f\times f) = g\circ\sigma_B\) because \(\sigma_B\cong\coker(\Delta_B)\); to complete the proof we will show that \(f = g\).

  From the proof of Lemma \ref{lemma:difference_lemma} we have \(r = q\circ\vvectormini{1_A}{0}\) so we get that \(\sigma_A\circ\vvectormini{1_A}{0} = r^{-1}\circ q\circ\vvectormini{1_A}{0} = r^{-1}\circ r = 1_A\); similarly \(\sigma_B\circ\vvectormini{1_B}{0} = 1_B\).
  From Remark \ref{remark:product_of_morphisms} we have
  \begin{align*}
    (f\times f)\circ\vvector{1_B}{0} = \vvector{f\circ 1_B}{0} = \vvector{1_A\circ f}{0} = \vvector{1_A}{0}\circ f
  \end{align*}
  and finally
  \begin{align*}
    g = g\circ\sigma_B\circ\vvector{1_B}{0} = \sigma_A\circ(f\times f)\circ\vvector{1_B}{0} = \sigma_A\circ\vvector{1_A}{0}\circ f = f
  \end{align*}
\end{proof}

\begin{theorem}[additivity of abelian categories]
  \label{teo:additivity_of_abelian_categories}
  Every abelian category is additive.
\end{theorem}

\begin{figure}[h]
  \begin{center}
  \begin{tikzcd}[sep=large]
    (A\times A)\times (A\times A) \ar[r, "p_i\times p_i"] \ar[d, "\sigma_{A\times A}"] & A\times A \ar[d, "\sigma_A"]\\
    A\times A \ar[r, "p_i"] & A
  \end{tikzcd}\begin{tikzcd}[sep=large]
    (A\times A)\times(A\times A) \ar[r, "\sigma_A\times\sigma_A"] \ar[d, "\sigma_{A\times A}"] & A\times A \ar[d, "\sigma_A"]\\
    A\times A \ar[r, "\sigma_A"] & A
  \end{tikzcd}
\end{center}

  \caption{}
  \label{diagram:abelian_additivity}
\end{figure}


\begin{proof}
  By applying Lemma \ref{lemma:sigma_lemma} with \(B = A\times A\) and \(f = p_i\) for \(i = 1,2\) we obtain the diagram on the left in Diagram \ref{diagram:abelian_additivity}, if instead we apply it with \(B=A\times A\) but \(f=\sigma_A\) we get the diagram on the right.

  Given four morphisms \(a,b,c,d\colon C\to A\) from Definition \ref{def:difference_on_abelian_categories} we have:
  \begin{gather*}
    \vvector{a}{b}-\vvector{c}{d} = \sigma_{A\times A}\circ\vvectorfour{a}{b}{c}{d}
  \end{gather*}
  From the left diagram:
  \begin{gather*}
    p_1\circ\sigma_{A\times A}\circ\vvectorfour{a}{b}{c}{d} = \sigma_A\circ(p_1\times p_1)\circ\vvectorfour{a}{b}{c}{d} = \sigma_A\circ\vvector{a}{c} = a - c\\
    p_2\circ\sigma_{A\times A}\circ\vvectorfour{a}{b}{c}{d} = \sigma_A\circ(p_2\times p_2)\circ\vvectorfour{a}{b}{c}{d} = \sigma_A\circ\vvector{b}{d} = b - d
  \end{gather*}
  so by Remark \ref{remark:product_property} we obtain \(\vvectormini{a}{b}-\vvectormini{c}{d} = \vvectormini{a - c}{b - d}\). From this last fact and the diagram on the right we obtain
  \begin{align*}
    (a - c)-(b - d) &= \sigma_A\circ\vvectormini{a - c}{b - d} = \sigma_A\circ\left(\vvectormini{a}{b}-\vvectormini{c}{d}\right)\\
                        &= \sigma_A\circ\sigma_{A\times A}\circ\vvectorfour{a}{b}{c}{d}\\
                        &= \sigma_A\circ(\sigma_A\times\sigma_A)\circ\vvectorfour{a}{b}{c}{d}\\
                        &= \sigma_A\circ\vvector{a - b}{c - d} = (a - b)-(c - d)
  \end{align*}
  In the proof of Lemma \ref{lemma:sigma_lemma} we showed that \(\sigma_A\circ\vvectormini{1_A}{0}=1_A\) and this implies that given \(a\colon C\to A\) we have \(a - 0 = \sigma_A\circ\vvectormini{a}{0}=\sigma_A\circ\vvectormini{1_A}{0}\circ a = a\), similarly from \(\sigma_A\circ\Delta_A = 0\) we obtain \(a - a =\sigma_A\circ\vvectormini{a}{a} = \sigma_A\circ\Delta_A\circ a = 0\); this gives us inverses and the identity element of \(\cat{C}(C, A)\). Finally using the relation \((a - c) - (b - d) = (a - b) - (c - d)\) we can prove commutativity and associativity of the sum in \(\cat{C}(C, A)\); thus that it is an abelian group.
  \begin{align}
    (0 - b) - c &= (0 - b) - (c - 0) = (0 - c) - (b - 0) = (0 - c) - b\\
    0 - (0 - d) &= (d - d) - (0 - d) = (d - 0) - (d - d) = (d - 0) - 0 = d\\
    b + c       &= b - (0 - c) \eqtext{(2)} (0 - (0 - b)) - (0 - c)\nonumber\\
                &= (0 - 0) - ((0 - b) - c) \eqtext{(1)} (0 - 0) - ((0 - c) -b))\nonumber\\
                &= (0 - (0 -c)) - (0 - b) \eqtext{(2)} c - (0 - b) = c + b\nonumber\\
    b + (0 - c) &= b - (0 - (0 - c)) \eqtext{(2)} b - c\\
    b + (0 - b) &\eqtext{(3)} b - b = 0\nonumber\\
    0 - (c - d) &= (0 - 0) - (c - d) = (0 - c) - (0 - d) = (0 - c) + d\\
    0 - (c + d) &= 0 - (c - (0 - d)) \eqtext{(4)} (0 - c) + (0 - d) \eqtext{(3)} (0 - c) - d\\
    (a - b) + d &= (a - b) - (0 - d) = (a - 0) - (b - d) = a - (b - d)\\
    (a + b) + d &= (a - (0 - b)) + d \eqtext{(6)} a - ((0 - b) - d) \eqtext{(5)} a - (0 - (b + d))\nonumber\\
                &= a + (b + d)\nonumber\\
    0 + a       &= 0 - (0 - a) = a\nonumber
  \end{align}
  Lastly we need to prove that the composition of arrows is a group homomorphism in both variables:
  \begin{itemize}
  \item given \(x\colon X\to C\) we have
    \begin{align*}
      (a - b)\circ x = \sigma_A\circ\vvector{a}{b}\circ x = \sigma_A\circ\vvector{a\circ x}{b\circ x} = (a\circ x) - (b\circ x),
    \end{align*}
  \item given \(y\colon A\to Y\), by using Lemma \ref{lemma:sigma_lemma} again, we have
    \begin{align*}
      y\circ(a - b) &= y\circ\sigma_A\circ\vvector{a}{b} = \sigma_Y\circ(y\times y)\circ\vvector{a}{b}\\
                    &= \sigma_Y\circ\vvector{y\circ a}{y\circ b} = (y\circ a) - (y\circ b),
    \end{align*}
  \end{itemize}
  and this completes the proof.
\end{proof}

\begin{corollary}
  \label{corollary:ab_additive}
  \(\catname{Ab}\) and \(\catname{Mod}_R\) are additive categories and the sum of morphisms is defined pointwise.
\end{corollary}

\begin{proof}
  \(\catname{Ab}\) and \(\catname{Mod}_R\) are abelian categories and thus, from Theorem \ref{teo:additivity_of_abelian_categories}, are additive as well. Moreover we know from Example \ref{ex:preadditive_category} that the pointwise sum of morphisms is a valid additive structure on both categories; by Proposition \ref{prop:uniqueness_of_additive_structure} it must be unique.
\end{proof}
