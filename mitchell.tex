\documentclass[12pt, letter paper, english, reqno]{article}
% \usepackage[left=3.75cm, right=3.75cm, top=3cm, bottom=3cm]{geometry}
\usepackage[utf8]{inputenc}

\setlength{\evensidemargin}{0.1in}
\setlength{\oddsidemargin}{0.1in}
\setlength{\textwidth}{6.3in}
\setlength{\topmargin}{0.0in}
\setlength{\textheight}{8.5in}
\setlength{\headheight}{0in}

% Hyperlinks.
\usepackage{xcolor}
\definecolor{Myblue}{rgb}{0,0,0.6}
\usepackage[a4paper,colorlinks,citecolor=Myblue,linkcolor=Myblue,urlcolor=Myblue,pdfpagemode=None]{hyperref}

% Math-related things.
\usepackage{mathtools}
\usepackage{amsmath, amsthm, amssymb}
\usepackage{mathrsfs}
\usepackage[margin=1cm]{caption}
\usepackage{enumitem}
\usepackage{newunicodechar}

% Tikz setup.
\usepackage{tikz}
\usetikzlibrary{cd, arrows, matrix}

% Amsmath stuff.
% \theoremstyle{definition}

\newtheoremstyle{myteo}{\topsep}{\topsep}
	{}
	{}
	{\bfseries}
	{.}
	{2pt}
	{\thmname{#1}\thmnumber{ #2}\thmnote{ (#3)}}
\theoremstyle{myteo}

\newtheorem{definition}{Definition}[section]
\newtheorem{example}[definition]{Example}
\newtheorem{notation}[definition]{Notation}
\newtheorem{remark}[definition]{Remark}
\newtheorem{fact}[definition]{Fact}
\newtheorem{proposition}[definition]{Proposition}
\newtheorem{theorem}[definition]{Theorem}
\newtheorem{corollary}[definition]{Corollary}
\newtheorem{lemma}[definition]{Lemma}

% Radom stuff.
\usepackage{csquotes}

% Custom commands.
\newcommand{\cat}[1]{{\mathscr #1}}
\newcommand{\catname}[1]{\text{\normalfont\textbf{#1}}}
\newcommand{\vvector}[2]{\begin{pmatrix}#1\\ #2\end{pmatrix}}
\newcommand{\hvector}[2]{\begin{pmatrix}#1 & #2\end{pmatrix}}
\newcommand{\vvectorfour}[4]{\begin{pmatrix}#1\\ #2\\ #3\\ #4\end{pmatrix}}
\newcommand{\vvectormini}[2]{\left(\begin{smallmatrix}#1\\ #2\end{smallmatrix}\right)}
\newcommand{\hvectormini}[2]{\left(\begin{smallmatrix}#1 & #2\end{smallmatrix}\right)}
\newcommand{\fun}[3]{#1\colon #2\to #3}
\newcommand{\epi}[3]{#1\colon #2\twoheadrightarrow #3}
\newcommand{\nat}[3]{#1\colon #2\Rightarrow #3}
\newcommand{\add}[2]{\text{\normalfont Add}[#1, #2]}
\newcommand{\functdef}[7]{\begin{align*}
                            #1\colon #2 &\longrightarrow #3\\
                            #4 &\longmapsto #5\\
                            #6 &\longmapsto #7
                          \end{align*}}
\newcommand{\fundef}[5]{\begin{align*}
                             #1\colon\begin{tikzcd}[sep=large]#2\end{tikzcd} &\longrightarrow \begin{tikzcd}[sep=large]#3\end{tikzcd}\\
                             \begin{tikzcd}[sep=large]#4\end{tikzcd} &\longmapsto \begin{tikzcd}[sep=large]#5\end{tikzcd}
                           \end{align*}}
\newcommand{\homset}[3]{#1(#2,#3)}
\newcommand{\precomp}[1]{{#1}^*}
\newcommand{\postcomp}[1]{{#1}_*}
\newcommand{\nathom}[2]{\text{\normalfont Nat}(#1,#2)}
\newcommand{\eqtext}[1]{\stackrel{\mathclap{\scriptsize\mbox{#1}}}{=}}
\newcommand{\op}[1]{#1^{\text{op}}}
\newcommand{\hh}[1]{{}_={#1}}
\newcommand{\vh}[1]{{#1}^{\parallel}}
\newcommand{\rec}[1]{{}^{\Box}{#1}}
\newcommand{\don}[1]{{#1}_{\Box}}

% Custom operators.
\DeclareMathOperator\coker{coker}
\DeclareMathOperator\im{im}
\DeclareMathOperator*\colim{colim}

% Yoneda.
\newcommand{\yo}{\text{\usefont{U}{min}{m}{n}\symbol{'210}}}
\DeclareFontFamily{U}{min}{}
\DeclareFontShape{U}{min}{m}{n}{<-> udmj30}{}

% Figures stuff.
\renewcommand{\figurename}{Diagram}
\renewcommand{\thefigure}{\arabic{section}.\arabic{figure}}

\begin{document}
% \rhead{\thepage}
% \lhead{\nouppercase{\leftmark}} 

\title{Notes on Abelian Categories and Mitchell's Embedding Theorem}
\author{Gabriele Rastello}
\maketitle
\tableofcontents

\newpage
\section*{Preface}
% These notes are a digested read of (a part of) the first chapter of Borceux's Handbook of Categorical Algebra 2.
% The intention is that of introducing Abelian Categories and then prove the celebrated Mitchell's Embedding Theorem (see \ref{teo:mitchell}).

% The text is tailored toward beginners in Category Theory

\medskip\noindent\emph{Why these notes exist.}
Around September 2019, given my interest in Category Theory and my upcoming graduation, I asked a professor that works in this branch to be the advisor of my graduation thesis.
After a brief meeting he assigned me the first chapter of Borceux's Handbook of Categorical Algebra (see \cite{handbook2}), with Mitchell's Embdedding Theorem as a specific goal.
With this in mind I started reading the book and writing these notes, adding here and there some details to the proofs, examples and so on.
Then the Covid19 pandemic hit and this document became the major way of communicating the progress I was making to my advisor.
Eventually everything was approved and these notes were recombined into my undergraduate thesis.
But in the process of doing so I realized that maybe someone, somewhere, could benefit from reading them and that publishing them wasn't hard at all (we have the internet) so this document was born.

\medskip\noindent\emph{Who could benefit from reading these notes?}
Anyone interested in learning the basics of Abelian Categories and seeing a proof of Mitchell's Embedding Theorem.
As I've already said these notes contain more detailed proofs that \cite{handbook2} and so I feel that can be most useful to beginners in the study of Category Theory.
Moreover they are more self-contained that \cite{handbook2} since most of the needed theorems that Borceux imports from his other book are re-stated and proven as well.
This removes the annoyance of having to continually switch between books.
However a minimal subset of results is not re-stated and re-proven; this mostly have to do with my advisor not requesting an explicit proof of them.
I might decide to include them in a later version of the notes, but this is very unlikely since it will basically require to re-write much of \cite{handbook1}'s chapter 2; that is very accessible to beginners anyway.

\medskip\noindent\emph{What do you mean with the term ``beginners''?}
Basically anyone that has the same degree of knowledge of category theory that I had back in September 2019.
Namely people that are comfortable with the notions of: categories, functors, natural transformations and (co)limits.
In this last requirement I include the definition of (co)limits and all the standard ones (initial/terminal objects, products/coproducs, pullbacks/pushouts, equalizers/coequalizers).

\medskip\noindent\emph{A word on section 6.}
The sixth and last section was initially intended to be a brief example of application of Mitchell's Embdedding Theorem.
However it grew into a quick introduction to the homology of (double) complexes followed by a proof of some classical lemmas.
The reader can safely ignore this section. 

\newpage
\section{Preadditive and additive categories}
\label{sec:preadditive}
\begin{definition}
  \label{def:zero_object}
  A {\bf zero object} of a category \(\cat{C}\) is an object \({\bf 0}\) of \(\cat{C}\)	that is both initial and terminal.
\end{definition}

\begin{proposition}
  \label{prop:zero_object_uniqueness}
  If a zero object \({\bf 0}\) exists then it is unique up to isomorphism.
\end{proposition}

\begin{proof}
  Let \({\bf 0'}\) be another zero object; then the unique arrows \({\bf 0}\to{\bf 0'}, {\bf 0}'\to{\bf 0}\) are mutually inverse since there is only one \({\bf 0}\to{\bf 0}\) arrow (that is \(1_{\bf 0}\)) and only one \({\bf 0'}\to{\bf 0'}\) arrow (that is \(1_{\bf 0'}\)).
\end{proof}

\begin{definition}
  \label{def:zero_arrow}
  In a category \(\cat{C}\) with a zero object \({\bf 0}\) an arrow \(f\colon A\to B\) that factors through \({\bf 0}\) is called a {\bf zero arrow} or a {\bf zero morphism}.
\end{definition}

\begin{proposition}
  \label{prop:existence_of_zero_arrows}
  In a category \(\cat{C}\) with a zero object \({\bf 0}\) given two objects \(A,B\) there is always exactly one zero arrow between them.
\end{proposition}

\begin{proof}
  The desired arrow is obtained by composing the unique arrows \(A\to{\bf 0}\) and \({\bf 0}\to B\). And is thus unique.
\end{proof}

\begin{notation}
  \label{notation:zero_arrows}
  We denote the zero arrow from \(A\) to \(B\) with \(0_{A\to B}\). However it is usually clear from the context what the domain and codomain of a zero arrow are so we will dispose of the subscript most of the times.
\end{notation}

\begin{remark}
  \label{remark:composition_with_zero_arrows}
  The composition of \(\fun{f}{A}{B}\) with the zero arrow \(0_{B\to C}\) is the zero arrow \(0_{A\to C}\). Indeed we have \(0_{B\to C}\circ f=0_{{\bf 0}\to C}\circ0_{B\to{\bf 0}}\circ f\), so \(0_{B\to C}\circ f\) factors through \({\bf 0}\) and is thus a zero arrow, by Proposition \ref{prop:existence_of_zero_arrows} it must be \(0_{A\to C}\). Similarly we have \(f\circ0_{C\to A}=0_{C\to B}\); dropping the subscripts we write \(0\circ f = 0\) and \(f\circ 0 = 0\).
\end{remark}

\begin{definition}
  \label{def:kernel}
  let \(\cat{C}\) be a category with a zero object \({\bf 0}\). We define the {\bf kernel} of an arrow \(f\colon A\to B\), when it exists, as the equalizer of \(f\) and \(0\colon A\to B\); dually we define the {\bf cokernel} of \(f\) as the coequalizer of \(f\) and \(0\).
\end{definition}

\begin{notation}
  \label{not:ker_coker}
  We write \(\ker(f, g)\) for the equalizer of \(f\) and \(g\) (when it exists) and \(\ker(f)\) for the kernel of \(f\); similarly we write \(\coker(f, g)\) for the equalizer of \(f\) and \(g\) and \(\coker(f)\) for the cokernel of \(f\).
\end{notation}

\begin{proposition}
  \label{prop:precomposition_with_mono}
  In a category \(\cat{C}\) with a zero object \({\bf 0}\) consider a monic arrow \(f\). If \(f\circ g = 0\) for some \(g\), then \(g = 0\).
\end{proposition}

\begin{proof}
  From Remark \ref{remark:composition_with_zero_arrows} we have \(f\circ0 = 0\) so \(f\circ g = 0 = f\circ 0\) and since \(f\) is monic then \(g=0\).
\end{proof}

\begin{proposition}
  \label{prop:the_ker_of_a_mono_is_zero}
  In a category \(\cat{C}\) with a zero object \({\bf 0}\) consider a monic arrow \(f\colon A\to B\); then the kernel of \(f\) is the zero arrow \({\bf 0}\to A\).
\end{proposition}

\begin{proof}
  Again \(f\circ0 = 0\) from Remark \ref{remark:composition_with_zero_arrows}. Now let \(g\) be an arrow such that \(f\circ g = 0\); from Proposition \ref{prop:precomposition_with_mono} we have \(g = 0\) and so \(g\) factors uniquely through \({\bf 0}\to A\); so \({\bf 0}\to A\) is the kernel of \(f\).
\end{proof}

\begin{proposition}
  \label{prop:kernel_of_zero_arrow}
  In a category \(\cat{C}\) with a zero object \({\bf 0}\) the kernel of a zero arrow \(0\colon A\to B\) is the identity \(1_A\) on \(A\).
\end{proposition}

\begin{proof}
  We have \(0\circ 1_A = 0\); now let \(g\colon X\to A\) be an arrow such that \(0\circ g = 0\). It is clear that there is a unique factorization of \(g\) through \(A\) that is \(g\) itself. This proves that \(1_A\) is the kernel of \(0\).
\end{proof}

\begin{definition}
  \label{def:preadditive_category}
  A {\bf preadditive category} is a category \(\cat{C}\) such that every hom-set \(\cat{C}(A, B)\) is an abelian group (that we will thus write additively) in such a way that the composition \(\circ\colon \cat{C}(A, B)\times\cat{C}(B, C)\to\cat{C}(A, C)\) is a group homomorphism in two variables for every choice of \(A,B,C\in\cat{C}\) (we also say that for every triple \(A,B,C\in\cat{C}\) the composition is bilinear).
\end{definition}

\begin{remark}
  \label{remark:dual_of_a_preadditive_category}
  The dual of a preadditive category is again preadditive.
\end{remark}

\begin{example}
  \label{ex:preadditive_category}
  The first and arguably most important example of preadditive category is the category of abelian groups \(\catname{Ab}\). Given two abelian groups \(A,B\) and two group homomorphisms \(f,g\colon A\to B\) we can define their sum as \(f + g\colon A\to B\) by setting \((f+g)(x) = f(x) + g(x)\) for every \(x\in A\). It is easily verified that \(f+g\) is again a group homomorphism, that the map that sends every element of \(A\) to the identity element of \(B\) is the identity element of \(\catname{Ab}(A, B)\) and that the composition of homomorphisms behave as expected. The same pointwise construction works for categories of left (or right) modules over a ring \(R\) as well.
\end{example}

\begin{example}
  \label{ex:ring_preadditive}
  Given a ring \(R\) we can see it as a preadditive category \(\cat{R}\) with a single object \(*\) and \(\cat{R}(*,*)=R\). Composition of arrows is just the product of the ring and the sum of arrows is the sum of the ring. Conversely a preadditive category \(\cat{R}\) with a single object yields a ring \(R\).
\end{example}

\begin{proposition}
  \label{prop:existence_of_zero_object}
  For a preadditive category \(\cat{C}\) the following are equivalent:
  \begin{enumerate}[label=(\arabic*)]
  \item \(\cat{C}\) has an initial object,
  \item \(\cat{C}\) has a final object,
  \item \(\cat{C}\) has a zero object.
  \end{enumerate}
  When any of the above hold then the identity elements of the hom-sets (as groups) are the zero morphisms.
\end{proposition}

\begin{proof}
  \((3) \Rightarrow (1)\) and \((3) \Rightarrow (2)\) are obvious from the definition of a zero object. By duality it is sufficient to prove only that \((1) \Rightarrow (3)\).

  Let \({\bf 0}\) be the initial object of \(\cat{C}\), we will show that it is also final. Since \({\bf 0}\) is initial we have \(\cat{C}({\bf 0},{\bf 0})=\{1_{\bf 0}\}\) and thus \(1_{\bf 0}\) is the identity element of the group \(\cat{C}({\bf 0},{\bf 0})\). Given \(C\in\cat{C}\) and \(f\in\cat{C}(C,{\bf 0})\) we have that \(1_{\bf 0}\circ f\) must be the identity element of \(\cat{C}(C, {\bf 0})\) because \(-\circ f\) is a group homomorphism (\(\cat{C}\) is preadditive); but \(1_{\bf 0}\circ f = f\) so \(f\) is the identity element of \(\cat{C}(C, {\bf 0})\) as well. By the uniqueness of the identity in a group we conclude that \(\cat{C}(C, {\bf 0})\) is a singleton and thus that \({\bf 0}\) is final.

  Assuming a zero object \({\bf 0}\) we now prove that the zero arrows are the identity elements of the hom-sets. For any two objects \(A,B\in\cat{C}\) the zero morphism \(0\colon A\to B\) is just the composition \(A\to{\bf 0}\to B\) of two arrows that are the identity elements of their respective hom-sets (because, from previous discussion, those hom-sets are singletons); so bilinearity of the composition immediately reveals that \(0\colon A\to B\) is the identity element of \(\cat{C}(A,B)\).
\end{proof}

\begin{remark}
  The category \(\catname{Ring}\) of rings and ring homomorphisms is not preadditive since it has both a final object (the trivial ring \(\{0\}\)) and an initial one (the ring of integers \({\mathbb Z}\)) but the two are not isomorphic, so \(\catname{Ring}\) has no zero object. The poset \(({\mathbb N}, \leq)\) regarded as a category has an initial object \(0\) but no final one, so it can't be preadditive.
\end{remark}

\begin{notation}
  \label{not:matrix_product}
  Suppose the product \(A\times B\) of \(A\) and \(B\) in a category \(\cat{C}\) exists and consider a cone \((C, f, g)\) with \(f\colon C\to A, g\colon C\to B\). Since \(A\times B\) is a product we know that there is a unique map \(m\colon C\to A\times B\) such that \(f=p_A\circ m\) and \(g = p_B\circ m\). We will here denote \(m\) with \(\vvectormini{f}{g}\) as in diagram \ref{diagram:product}; this notation will be fully justified once additive categories are introduced.

  \begin{figure}
    \begin{center}
  \begin{tikzcd}[sep=huge]
    C \ar[r, "f"] \ar[d, "g"'] \ar[dr, dashed, "\vvectormini{f}{g}" description] & A\\
    B & A\times B \ar[l, "p_B"] \ar[u, "p_A"']
  \end{tikzcd}
\end{center}
    \caption{}
    \label{diagram:product}
  \end{figure}
\end{notation}

\begin{remark}
  \label{remark:product_property}
  If \(A\times B\) is a product with projections \(p_A\) and \(p_B\) and we have two arrows \(f,g\colon X\to A\times B\) such that \(p_A\circ f = p_A\circ g\) and \(p_B\circ f = p_B\circ g\) then \(f = g\). This simple result follows from the uniqueness of the factorization through the product. In the following proofs the use of this property of products to show that two arrows are the same is a very common pattern.
\end{remark}

\begin{proposition}
  \label{prop:existence_of_products}
  For a preadditive category \(\cat{C}\) the following are equivalent:
  \begin{enumerate}[label=(\arabic*)]
  \item the product \((P, p_A, p_B)\) of \(A\) and \(B\) exists,
  \item the coproduct \((P, s_A, s_B)\) of \(A\) and \(B\) exists,
  \item there is an object \(P\) equipped with four arrows \(p_A,p_B,s_A,s_B\) where \(p_X\colon P\to X\) and \(s_X\colon X\to P\) for \(X=A,B\) in such a way that:
    \begin{itemize}
    \item \(p_A\circ s_A = 1_A\) and \(p_B\circ s_B = 1_B\),
    \item \(s_A\circ p_B = 0\) and \(s_B\circ p_A = 0\),
    \item \((s_A\circ p_A) + (s_B\circ p_B) = 1_P\).
    \end{itemize}
  \end{enumerate}
  When any of the above hold then we also have:
  \[s_A=\ker(p_B), s_B=\ker(p_A), p_A=\coker(s_B), p_B=\coker(s_A).\]
\end{proposition}

\begin{proof}
  By duality it is sufficient to prove that \((1) \Leftrightarrow (3)\).\\

  Suppose \((1)\) and define \(s_A=\vvectormini{1_A}{0}\) and \(s_B=\vvectormini{0}{1_B}\). We get
  \begin{align*}
    p_A\circ(s_A\circ p_A + s_B\circ p_B) = (1_A\circ p_A) + 0 = p_A\circ 1_P \\
    p_B\circ(s_A\circ p_A + s_B\circ p_B) = 0 + (1_B\circ p_B) = p_B\circ 1_P
  \end{align*}
  and so from \ref{remark:product_property} we get that \(s_A\circ p_A + s_B\circ p_B = 1_P\).

  Now, given condition \((3)\), consider two morphisms \(f\colon C\to A\) and \(g\colon C\to B\). We define \(h= s_A\circ f + s_B\circ g\colon C\to P\) and prove that it is a unique factorization for \(f\) and \(g\) through \(P\). We have
  \begin{align*}
    p_A\circ h = p_A\circ(s_A\circ f + s_B\circ g) = 1_A\circ f + 0\circ g = f\\
    p_B\circ h = p_B\circ(s_A\circ f + s_B\circ g) = 0\circ f + 1_B\circ g = g
  \end{align*}
  so \(h\) is a factorization of \(f\) and \(g\) through the product. Now suppose that \(h':C\to P\) is another arrow such that \(p_A\circ h' = f\) and \(p_B\circ h' = g\). We have
  \begin{align*}
    h' & = 1_P\circ h'\\
       &= (s_A\circ p_A + s_B\circ p_B)\circ h'\\
       & = s_A\circ p_A\circ h' + s_B\circ p_B\circ h' \\
       & = s_A\circ f + s_B\circ g = h
  \end{align*}
  and so \((P,p_A,p_B)\) is the product of \(A\) and \(B\).

  Finally we want to prove that \(s_A=\ker(p_B)\) assuming any of the conditions above. From \((3)\) we have that \(p_B\circ s_A = 0\); let \(x\colon C\to P\) be an arrow such that \(p_B\circ x = 0\). From the following and \ref{remark:product_property} it is clear that \(p_A\circ x\) is a factorization of \(x\) through \(s_A\):
  \begin{gather*}
    p_A\circ s_A\circ(p_A\circ x) = p_A\circ x\\
    p_B\circ s_A\circ(p_A\circ x) = 0 \circ p_A\circ x = 0 = p_B\circ x
  \end{gather*}
  The uniqueness of this factorization is obtained by observing that \(s_A\) is monic (since \(p_A\circ s_A = 1_A\)) and thus \(s_A=\ker(p_B)\). The other three relations follow similarly and by duality.
\end{proof}

\begin{remark}
  From Proposition \ref{prop:existence_of_products} it follows that \(\catname{Grp}\) is not a preadditive category. The binary product of two groups \(G,H\in\catname{Grp}\) is their Cartesian product \(G\times H\) while the binary coproduct of \(G\) and \(H\) is the free product \(G*H\). If \(G,H\) are nontrivial finite groups then \(G\times H\) is again finite but \(G*H\) is infinite and thus \(G\times H\not\cong G*H\); this shows that Proposition \ref{prop:existence_of_products} does not hold and hence \(\catname{Grp}\) cannot be preadditive.
\end{remark}

\begin{definition}
  \label{def:biproduct}
  In a preadditive category a quintuple \((P, p_A, p_B, s_A, s_B)\) as in Proposition \ref{prop:existence_of_products} is called the {\bf biproduct} of \(A\) and \(B\) and is usually denoted with \(A\oplus B\).
\end{definition}

\begin{definition}
  \label{def:additive_category}
  A preadditive category \(\cat{C}\) is said to be {\bf additive} if it has a zero object and all biproducts.
\end{definition}

Particularly in an additive category zero arrows are the identity elements of the hom-sets and we can freely use all the relations given by Proposition \ref{prop:existence_of_products}. This two things together allow us to prove the following, surprising, result.

\begin{proposition}
  \label{prop:uniqueness_of_additive_structure}
  On a category \(\cat{C}\) two additive structures are necessarily isomorphic.
\end{proposition}

\begin{proof}
  Let \(\cat{C}\) be endowed with an additive structure and \(C\) be an object of \(\cat{C}\); we consider the diagonal \(\Delta_C=\vvectormini{1_C}{1_C}\) and the difference of projections \(\sigma_C= p_1-p_2:C\oplus C\to C\); we shall prove that \(\sigma_C = \coker(\Delta_C)\). We have
  \begin{gather*}
    \sigma_C\circ\Delta_C = (p_1-p_2)\circ\Delta_C = (p_1\circ\Delta_C) - (p_2\circ\Delta_C) = 1_C - 1_C = 0
  \end{gather*}
  and from the following we also have that \(\Delta_C = s_1+s_2\):
  \begin{align*}
    p_1\circ(s_1+s_2) = (p_1\circ s_1) + (p_1\circ s_2) = 1_C\\
    p_2\circ(s_1+s_2) = (p_2\circ s_1) + (p_2\circ s_2) = 1_C
  \end{align*}
  Now let \(f:C\oplus C\to D\) be an arrow such that \(f\circ\Delta_C= 0\); we observe that
  \begin{gather*}
    (f\circ s_1) + (f\circ s_2) = f\circ(s_1+s_2) = f\circ\Delta_C = 0
  \end{gather*}
  Then \(g=f\circ s_1\) is a factorization of \(f\) through \(\sigma_C\):
  \begin{align*}
    g\circ\sigma_C &= f\circ s_1 \circ (p_1 - p_2)\\
                   &= (f\circ s_1\circ p_1) - (f\circ s_1\circ p_2)\\
                   &= (f\circ s_1\circ p_1) + (f\circ s_2\circ p_2)\\
                   &= f\circ(s_1\circ p_1 + s_2\circ p_2) = f
  \end{align*}
  Moreover this factorization is unique: if \(g'\colon C\to D\) is such that \(g'\circ\sigma_C = f\) then we have
  \begin{align*}
    g' &= (g'\circ 1_C) - (g'\circ 0)\\
       &= (g'\circ p_1\circ s_1) - (g'\circ p_2\circ s_1)\\
       &= g'\circ (p_1-p_2)\circ s_1\\
       &= g'\circ\sigma_C\circ s_1 =  f\circ s_1 = g
  \end{align*}
  and so \(\sigma_C=\coker(\Delta_C)\). This fact shows that the value of the difference \(p_1-p_2\) is not arbitrary, but it's determined up to isomorphism by the limit-colimit structure of \(\cat{C}\). It is now easy to show that the same holds for any difference: let \(a,b:A\to C\) be arrows and \(\vvectormini{a}{b}\) their factorization through \(C\oplus C\); we have
  \begin{align*}
    \tag{\(*\)}
    a-b = \left(p_1\circ\vvector{a}{b}\right) - \left(p_2\circ\vvector{a}{b}\right) = (p_1 - p_2)\circ\vvector{a}{b} = \sigma_C\circ\vvector{a}{b}
  \end{align*}
  and thus also the difference \(a-b\) is determined up to isomorphism by the limit-colimit structure of \(\cat{C}\). We already know from Proposition \ref{prop:existence_of_zero_object} that the identity elements of the hom-sets in an additive category are the zero arrows  and thus the proof is complete.
\end{proof}

This last proof, particularly the relation \((*)\), will be of considerable importance when we will endow every abelian category with an additive structure.

\begin{proposition}
  \label{prop:kernels_in_preadditive_category}
  In a preadditive category \(\cat{C}\), given two arrows \(f,g\colon A\to B\) the following are equivalent:
  \begin{enumerate}[label=(\arabic*)]
  \item \(\ker(f,g)\) exists,
  \item \(\ker(f-g)\) exists,
  \item \(\ker(g-f)\) exists.
  \end{enumerate}
\end{proposition}

\begin{proof}
  \(\ker(f,g) = \ker(g, f)\) so it is sufficient to prove the equivalence of \((1)\) and \((2)\). Given a morphisms \(x\colon X\to A\) we have
  \begin{align*}
    f\circ x = g\circ x \Leftrightarrow (f - g)\circ x = 0
  \end{align*}
  from which \((1)\Leftrightarrow(2)\) follows immediately.
\end{proof}

\begin{notation}
  \label{not:matrices}
  Given four objects \(A_1,A_2,B_1,B_2\) and an arrow \(f\colon A_1\oplus A_2\to B_1\oplus B_2\) in an additive category \(\cat{C}\) we have that the arrow \(f\) is uniquely identified by the four morphisms \(f_{ij}\colon A_j\to B_i\) for \(i,j=1,2\) such that \(f_{ij}=p_i\circ f\circ s_j\) where the \(p_i\) are the projections of \(B_1\oplus B_2\) and the \(s_j\) are the injections of \(A_1\oplus A_2\). It is now possible to write \(f\) as a matrix
  \begin{align*}
    \begin{pmatrix}
      f_{11} & f_{12}\\
      f_{21} & f_{22}
    \end{pmatrix}
  \end{align*}
  This notation can clearly be used for \(n\)-ary products as well. Moreover if we consider arrows \(f\colon A\to B\) and \(g\colon A\to C\) we can consider Notation \ref{not:matrix_product} as a particular instance of this more general matrix notation (here the matrix becomes a vector written vertically); dually for arrows \(f\colon B\to A\) and \(g\colon C\to A\) we write the factorization through \(B\oplus C\) as \(\hvectormini{f}{g}\colon B\oplus C\to A\).
\end{notation}

\begin{remark}
  \label{remark:matrix_operations}
  The following two properties justify the choice of arranging the arrows \(f_{ij}\) in a matrix.
  \begin{enumerate}[label=(\arabic*)]
  \item Given arrows \(f\colon A_1\oplus A_2\to B_1\oplus B_2\) and \(g\colon B_1\oplus B_2\to C_1\oplus C_2\) the matrix representation of \(g\circ f\) is exactly the product of the matrices representing \(f\) and \(g\).
  \item Given arrows \(f,g\colon A_1\oplus A_2\to B_1\oplus B_2\) the matrix that represents \(f + g\) is exactly the sum of the matrices that represent \(f\) and \(g\).
  \end{enumerate}
\end{remark}


\newpage
\section{Abelian categories}
\label{sec:abelian}
\begin{definition}
  \label{def:abelian_category}
  A category \(\cat{C}\) is {\bf abelian} if:
  \begin{enumerate}[label=(\arabic*)]
  \item has a zero object \({\bf 0}\),
  \item every pair of objects of \(\cat{C}\) has product and coproduct,
  \item every arrow in \(\cat{C}\) has a kernel and a cokernel,
  \item every mono in \(\cat{C}\) is the kernel of some arrow and every epi in \(\cat{C}\) is the cokernel of some arrow.
  \end{enumerate}
\end{definition}

\begin{remark}
  The notion of abelian category is autodual.
\end{remark}

\begin{example}
  \label{ex:ab_is_abelian}
  \(\catname{Ab}\), the category \(\catname{Mod}_R\) of left R-modules over a ring \(R\) (as well as the category of right-modules over \(R\)) are abelian categories. Indeed the goal behind Definition \ref{def:abelian_category} is that of capturing the fundamental properties of \(\catname{Ab}\) and, more generally, \(\catname{Mod}_R\).
\end{example}

\begin{proposition}
  \label{prop:iso_mono_epi}
  In an abelian category \(\cat{C}\) an arrow \(f\) is an isomorphism if and only if is both epic and monic.
\end{proposition}

\begin{proof}
  If \(f\) is an isomorphism then \(f\) is both split monic and split epic and thus both monic and epic. On the other hand suppose \(f\) is both monic and epic; since \(\cat{C}\) is abelian there is an arrow \(g\) such that \(f = \ker(g)\). From this last fact we get \(g\circ f = 0\) and so \(g = 0\) because \(f\) is also epic. Finally we have that \(f\) is the kernel of a zero arrow and thus by Proposition \ref{prop:kernel_of_zero_arrow} is an isomorphism.
\end{proof}

\begin{remark}
  The category \(\catname{Ring}\) of rings and rings homomorphisms is not abelian since the inclusion of \(\mathbb{Z}\) in \(\mathbb{Q}\) is both epic and monic, but clearly not an isomorphism.
\end{remark}

\begin{proposition}
  \label{prop:existence_of_intersection }
  In an abelian category \(\cat{C}\) given two monomorphisms \(a,b\) with codomain \(C\) their pullback always exists.
\end{proposition}

\begin{figure}
  \begin{center}
  \begin{tikzcd}[sep=huge]
    X \ar[dr, dashed, "w"] \ar[drr, "v", bend left] \ar[ddr, "u"', bend right] & & &\\
    & \ker\vvectormini{f}{g} \ar[d, dashed, "a'"'] \ar[r, dashed, "b'"] \ar[dr, tail, "k"] & B \ar[d, "b"] &\\
    & A \ar[r, "a"'] & C \ar[r, "f"] \ar[d, "g"'] \ar[dr, "\vvectormini{f}{g}"]& D\\
    & & E & D\times E \ar[u, "p_D"'] \ar[l, "p_E"]
  \end{tikzcd}
\end{center}


  \caption{}
  \label{diagram:intersection}
\end{figure}

\begin{proof}
  With reference to Diagram \ref{diagram:intersection} consider two monomorphisms \(a\colon A\rightarrowtail C,b\colon B\rightarrowtail C\); since \(\cat{C}\) is abelian there are morphisms \(f\colon C\to D,g\colon C\to E\) such that \(a=\ker(f)\) and \(b=\ker(g)\); let's also consider the arrow \(k=\ker\vvectormini{f}{g}\). We have that \(f\circ k = p_D\circ\vvectormini{f}{g}\circ k = 0\) so \(k\) factors as \(k = a\circ a'\) because \(a=\ker(f)\); in the same way we obtain \(k = b'\circ b\). We now want to prove that \(\ker\vvectormini{f}{g}\) together with the arrows \(a'\) and \(b'\) is the pullback of the pair \((a, b)\). Given arrows \(u,v\) such that \(a\circ u = b\circ v\) we get \(f\circ b\circ v = f\circ a\circ u = 0\) and since \(a=\ker(f)\) we obtain that there is a unique \(w\) such that \(k\circ w = b\circ v\). Now \(b\circ v = k\circ w = b\circ b'\circ w\) implies \(b'\circ w = v\) since \(b\) is monic. Similarly one has \(a\circ u = b\circ v = k\circ w = a\circ a' \circ w\) and thus \(u=a'\circ w\) because \(a\) is also monic. To prove the uniqueness of this factorization suppose that \(w'\) is an arrow such that \(u = a'\circ w'\) and \(v = b'\circ w'\), using the factorization through \(w\) we have:
  \begin{align*}
    k\circ w' = a\circ a'\circ w' = a\circ u = a\circ a'\circ w = k\circ w
  \end{align*}
  and thus \(w = w'\) because \(k\) is monic.
\end{proof}

\begin{proposition}
  \label{prop:completeness}
  An abelian category \(\cat{C}\) is finitely complete and finitely cocomplete.
\end{proposition}

\begin{proof}
  By duality it suffice to prove that \(\cat{C}\) is finitely complete. We know from  \cite[Proposition 2.8.2]{handbook1} that a category is finitely complete if and only if it has a terminal object, all binary products and all equalizers. \(\cat{C}\) is abelian and thus we just need to prove that it has all the equalizers.

  Let \(f,g:A\to B\) be arrows in \(\cat{C}\); we consider \(\vvectormini{1_A}{f}\), \(\vvectormini{1_A}{g}\), that are monic since they compose with \(p_A\) to give \(1_A\), and their pullback \((P, u, v)\). We have the following:
  \begin{gather*}
    u = 1_A\circ u = p_A\circ\vvector{1_A}{f}\circ u = p_A\circ\vvector{1_A}{g}\circ v = 1_A\circ v = v\\
    f\circ u = p_B\circ\vvector{1_A}{f}\circ u = p_B\circ\vvector{1_A}{g}\circ v = g\circ v = g\circ u
  \end{gather*}
  so \(u = v\) and \(f\circ u =g\circ u\). We now want to prove that \(u\) is really the equalizer of \(f\) and \(g\). With reference to diagram \ref{diagram:equalizers} suppose \(x:X\to A\) is an arrow such that \(f\circ x = g\circ x\); we consider the cone \((X,x,x)\) and thus obtain a unique factorization of \(x\) through \(u\). This proves that \(u=\ker(f, g)\).
  \begin{figure}[h]
    \begin{center}
  \begin{tikzcd}[sep=huge]
    X \ar[drr, "x", bend left] \ar[ddr, "x"', bend right] \ar[dr, dashed, ""] & & \\
    & P \ar[r, tail, "v"] \ar[d, tail, "u"'] & A \ar[d, tail, "\vvectormini{1_A}{g}"] \\
    & A \ar[r, tail, "\vvectormini{1_A}{f}"'] & A\times B
  \end{tikzcd}
\end{center}

    \caption{}
    \label{diagram:equalizers}
  \end{figure}
\end{proof}

\begin{proposition}
  \label{prop:mono_characterization}
  For a morphism \(f\colon A\to B\) in an abelian category \(\cat{C}\) the following conditions are equivalent:
  \begin{enumerate}[label=(\arabic*)]
  \item \(f\) is monic,
  \item \(\ker(f)=0\),
  \item for every \(g:C\to A\) we have \(f\circ g = 0\Rightarrow g = 0\).
  \end{enumerate}
\end{proposition}

\begin{proof}
  \((1)\Rightarrow(2)\) is exactly Proposition \ref{prop:the_ker_of_a_mono_is_zero} while \((1)\Rightarrow(3)\) is exactly Proposition \ref{prop:precomposition_with_mono}.

  To prove that \((2)\) implies \((1)\) we consider \(u,v\) such that \(f\circ u = f\circ v\) and prove that \(u = v\); we refer to Diagram \ref{diagram:mono}. Let \(q=\coker(u,v)\) (since \(\cat{C}\) is finitely cocomplete for \ref{prop:completeness}), \(q\) is epic and so there is an arrow \(w\) such that \(q = \coker(w)\). Since \(f\circ u = f\circ v\) then \(f\) factors uniquely as \(f = m\circ q\) through \(q = \coker(u, v)\). We now have \(f\circ w = m\circ q\circ w = m\circ 0 = 0\) so \(w\) factors uniquely as \(w=k\circ n\) through \(k = \ker(f)\); but we have that \(\ker(f) = 0\) so \(w = 0\) and finally since \(q = \coker(0)\), by Proposition \ref{prop:kernel_of_zero_arrow}, \(q\) is an isomorphism. Particularly \(q\) is monic and thus \(q\circ u =q\circ v\Rightarrow u = v\).
  
  To prove that \((3)\) implies \((2)\) we start with the trivial observation that \(f\circ 0_{{\bf 0}\to A} = 0_{{\bf 0}\to B}\) where \(f\colon A\to B\). Now if \(g\colon C\to A\) is such that \(f\circ g = 0_{C\to B}\) then by \((3)\) \(g = 0_{C\to A}\) and thus factors uniquely through \(0_{{\bf 0}\to A}\). This proves that \(\ker(f) = 0_{{\bf 0}\to A}\).
  \begin{figure}[h]
    \begin{center}
  \begin{tikzcd}[sep=huge]
    \ker(f) \ar[dr, "k"] & \bullet \ar[l, dashed, "n"'] \ar[d, "w"] & \\
    \bullet \ar[r, shift left = .5ex, "u"] \ar[r, shift right = .5ex, "v"'] & \bullet \ar[d, two heads, "q"']  \ar[r, "f"] & \bullet \\
    & \bullet \ar[ur, dashed, "m"'] & 
  \end{tikzcd}
\end{center}
    \caption{}
    \label{diagram:mono}
  \end{figure}
\end{proof}

\begin{theorem}[epi-mono factorization]
  \label{teo:epi_mono_factorization}
  Every morphism \(f\) in an abelian category can be factored uniquely (up to isomorphism) as \(f = i\circ p\) with \(i\) monic and \(p\) epic. Moreover we have that \(i = \ker(\coker(f))\) and \(p = \coker(\ker(f))\).
\end{theorem}

\begin{figure}[h]
  \begin{center}
  \begin{tikzcd}[sep=huge]
    & \bullet \ar[dl, "l"'] \ar[d, "h"] & \\
    \bullet \ar[r, "k"] & \bullet \ar[r, "f"] \ar[d, "p"'] & \bullet \\p
    \bullet \ar[r, "x"] & \bullet \ar[r, shift right = .5ex, "q"'] \ar[ur, "i"] & \bullet \ar[u, "r"'] \ar[l, shift right = .5ex, "s"']
  \end{tikzcd}
\end{center}
  \caption{}
  \label{diagram:factorization}
\end{figure}

\begin{proof}
  With reference to diagram \ref{diagram:factorization} we define \(k = \ker(f)\) and \(p = \coker(k)\). Since \(f\circ k = 0\), because \(k = \ker(f)\), \(f\) must factor as \(f = i\circ p\) through the cokernel \(p\) of \(k\). We now wish to prove that \(i\) is monic.

  Let \(x\) be an arrow such that \(i\circ x = 0\); if we can prove that \(x = 0\) then, by Proposition \ref{prop:mono_characterization}, \(i\) must be monic. Now \(i\circ x = 0\) immediately gives us the factorization \(i = r\circ q\) through \(q = \coker(x)\). Moreover \(q\circ p\) is the composition of two epimorphisms and thus is again an epimorphism, since we're in an abelian category there exists an arrow \(h\) such that \(q\circ p = \coker(h)\). From
  \begin{align*}
    f\circ h = i\circ p\circ h = r\circ q\circ p\circ h = r\circ 0 = 0
  \end{align*}
  we obtain that \(h\) factors as \(h = k\circ l\) since \(k = \ker(f)\). Now \(p\circ h = p\circ k\circ l = 0\circ l = 0\) since \(p=\coker(k)\) and thus \(p\) factors uniquely through \(\coker(h) = q\circ p\) and we get \(p = s\circ(q\circ p)\). From this last relation and the fact that \(p\) is epic we obtain \(s\circ q = 1\) thus \(q\) is a monomorphism so from \(q\circ x = 0\) we obtain \(x = 0\).

  The uniqueness of the factorization \(f = i\circ p\) follows from \cite[4.4.5]{handbook1} and \cite[4.3.6]{handbook1}. By duality \(f\) can be factored also as \(f = i'\circ p'\) where we have that \(i' = \ker(\coker(f))\) and that \(p'\) is epi. The uniqueness states that those two factorizations are isomorphic and this concludes the proof.
\end{proof}

\begin{corollary}
  \label{cor:mono_kernel_cokernel_epi_cokernel_kernel}
  In an abelian category the following hold:
  \begin{enumerate}[label=(\arabic*)]
  \item every monomorphism is the kernel of its cokernel,
  \item every epimorphism is the cokernel of its kernel.  
  \end{enumerate}
\end{corollary}

\begin{proof}
  Let \(f\) be an epimorphism; by Theorem \ref{teo:epi_mono_factorization} it can be factored as \(f=i\circ p\) with \(i\) monomorphism and \(p = \coker(\ker(f))\). Now since \(f\) is epic \(i\) must be too and thus \(i\) is an isomorphism by Proposition \ref{prop:iso_mono_epi}; finally we have \(f= \coker(\ker(f))\). By duality we obtain (1). 
\end{proof}

\begin{proposition}
  \label{prop:pullback_of_epi}
  In an abelian category \(\cat{C}\) monomorphisms are pushout-stable i.e. the pushout of a monic arrow is again a monic arrow. By duality epimorphisms are pullback-stable i.e. the pullback of an epi arrow is again an epi arrow.
\end{proposition}

\begin{figure}
  \begin{center}
  \begin{tikzcd}[sep=huge]
    A \ar[rr, "f"] \ar[dd, tail, "g"] \ar[dr, "l"] & & B \ar[dd, "k"] \ar[dl, shift left = 1, "s_B"]\\
    & B\oplus C \ar[dl, shift left = 1, "p_C"] \ar[rd, two heads, "p"] \ar[ru, shift left = 1, "p_S"] & \\
    C \ar[ru, shift left = 1, "s_C"] \ar[rr, "h"] & & D
  \end{tikzcd}
\end{center}

  \caption{}
  \label{diagram:pushepi}
\end{figure}

\begin{proof}
  As in Diagram \ref{diagram:pushepi} let \(g\) be mono and \(f\) an arbitrary arrow. Considering the biproduct \(B\oplus C\) we define
  \begin{equation*}
    l = \vvector{-f}{g} = 1_{B\oplus C}\circ\vvector{-f}{g} = \hvector{s_B}{s_C}\circ\vvector{-f}{g} = -(s_B\circ f) + (s_C\circ g).
  \end{equation*}
  We also set \(p = \coker(l)\), \(h = p\circ s_C\) and \(k = p\circ s_B\). We shall prove that the pair \((h, k)\) is the pushout of \((g, f)\). First we observe that
  \begin{align*}
    (h\circ g) - (k\circ f) &= (p\circ s_C \circ g) - (p\circ s_B\circ f)\\
                            &= p\circ((s_C\circ g) - (s_B\circ f))\\
                            &= p\circ l = 0
  \end{align*}
  and thus \(h\circ g = k\circ f\). Now let \(\fun{m}{B}{M}\) and \(\fun{n}{C}{M}\) be arrows such that \(m\circ f = n\circ g\) and consider the codiagonal morphism \(\fun{\hvectormini{m}{n}}{B\oplus C}{M}\); we have that
  \begin{align*}
    \hvector{m}{n}\circ l = \hvector{m}{n}\circ\vvector{-f}{g} = -(m\circ f) + (n\circ g) = 0
  \end{align*}
  and thus, since \(p\) is the cokernel of \(l\), \(\hvectormini{m}{n} = d\circ p\) for a unique arrow \(\fun{d}{D}{M}\). But now
  \begin{align*}
    d\circ k &= d\circ p\circ s_B = \hvectormini{m}{n}\circ s_B = m\\
    d\circ h &= d\circ p\circ s_C = \hvectormini{m}{n}\circ s_C = n
  \end{align*}
  and so \(d\) is the unique arrow that factorizes the cocone \((m, n)\). We finally have that \((k, h)\) is the pushout of \((g, f)\).

  Now to prove that \(k\) is mono we shall first prove that \(l\) is. Let \(\fun{x}{X}{A}\) be an arrow such that \(l\circ x = 0\). We immediately have that
  \begin{align*}
    0 = p_C\circ 0 &= p_C\circ l\circ x\\
                   &= p_C\circ(-(s_B\circ f) + (s_C\circ g))\circ x\\
                   &= g\circ x
  \end{align*}
  and thus \(x = 0\) because \(g\) is monic. By Proposition \ref{prop:mono_characterization} we confirm that \(l\) is mono and thus \(l = \ker(\coker(l)) = \ker(p)\) by Corollary \ref{cor:mono_kernel_cokernel_epi_cokernel_kernel}.

  Finally let \(\fun{y}{Y}{B}\) be an arrow such that \(k\circ y = 0\). We have that \(p\circ s_B\circ y = k\circ y = 0\) by our definition of \(p\) and thus \(s_B\circ y = l\circ z\) for a unique \(\fun{z}{Y}{A}\) since \(l\) is the kernel of \(p\). Now
  \begin{equation*}
    g\circ z = p_C\circ\vvector{-f}{g}\circ z = p_C\circ l\circ z = p_C\circ s_B\circ y = 0
  \end{equation*}
  and thus \(z = 0\) because \(g\) is a monomorphism by assumption. Now since \(s_B\) is mono and \(s_B\circ y = l\circ z = 0\) we obtain \(y = 0\) and finally, by Proposition \ref{prop:mono_characterization} again, that \(k\) is mono.
\end{proof}

\begin{lemma}
  \label{lemma:difference_lemma}
  Let \(\cat{C}\) be an abelian category, \(A\in\cat{C}\) one of its objects, \(\Delta = \vvectormini{1_A}{1_A}\colon A\to A\times A\) and \(q = \coker(\Delta)\colon A\times A\to Q\); then \(A\cong Q\).
\end{lemma}

\begin{figure}[h]
  \begin{center}
  \begin{tikzcd}[sep=huge]
    X \ar[r, "x"] \ar[d, "y"] & A \ar[d, tail, "\vvectormini{1_A}{0}"'] \ar[dr, "r" near end] & V \ar[l, "p_1\circ v"'] \ar[dl, "v"' near end]\\
    A \ar[d, "1_A"] \ar[dr, "1_A" near end] \ar[r, tail, "\Delta"] & A\times A \ar[r, two heads, "q"] \ar[ld, "p_1" near end] \ar[d, two heads, "p_2"] & Q \ar[d, "z"]\\
    A & A \ar[r, "t"] & Y
  \end{tikzcd}
\end{center}
  \caption{}
  \label{diagram:lemma1}
\end{figure}

\begin{proof}
  With reference to diagram \ref{diagram:lemma1} we define \(r = q\circ\vvectormini{1_A}{0}\) and prove that it is an isomorphism.

  Since \(\Delta\) is monic \(\Delta = \ker(\coker(\Delta)) = \ker(q)\). Via Notation \ref{not:matrix_product} we have \(p_1\circ\vvectormini{1_A}{0} = 1_A\) and \(p_2\circ\vvectormini{0}{1_A} = 1_A\) so \(p_1\) and \(p_2\) are epic while \(\vvectormini{1_A}{0}\) and \(\vvectormini{0}{1_A}\) are monic. Again via Notation \ref{not:matrix_product} we have \(p_2\circ\vvectormini{1_A}{0} = 0\) and, if \(p_2\circ v = 0\) for some arrow \(v\), then \(\vvectormini{1_A}{0}\circ(p_1\circ v) = v\) because:
  \begin{gather*}
    p_1\circ\vvector{1_A}{0}\circ(p_1\circ v) = p_1\circ v\\
    p_2\circ\vvector{1_A}{0}\circ(p_1\circ v) = 0\circ(p_1\circ v) = 0 = p_2\circ v
  \end{gather*}
  and this factorization is unique because \(\vvectormini{1_A}{0}\) is monic. With this we have established that \(\vvectormini{1_A}{0} = \ker(p_2)\) and with an analogue argument that \(\vvectormini{0}{1_A} = \ker(p_1)\). Finally since \(p_1\) is epic \(p_1 = \coker(\ker(p_1)) = \coker\vvectormini{0}{1_A}\) and similarly \(p_2 = \coker(\ker(p_2)) = \coker\vvectormini{1_A}{0}\).

  We prove that \(r\) is monic by Proposition \ref{prop:mono_characterization}. Let \(x\) be an arrow such that \(r\circ x = 0\). \(0 = r\circ x = q\circ\vvectormini{1_A}{0}\circ x\), \(\Delta = \ker(q)\) so we have the factorization \(\vvectormini{1_A}{0}\circ x =\Delta\circ y\); now \(y = p_2\circ\Delta\circ y = p_2\circ\vvectormini{1_A}{0}\circ x = 0\) so since \(\vvectormini{1_A}{0}\) is monic we have that \(x = 0\) and so \(r\) is monic.

  We prove that \(r\) is epic by the dual of Proposition \ref{prop:mono_characterization}. Let \(z\) be an arrow such that \(z\circ r = 0\). \(0 = z\circ r = z\circ q\circ\vvectormini{1_A}{0}\), \(p_2=\coker\vvectormini{1_A}{0}\) so we have the factorization \(z\circ q = t\circ p_2\); we now have \(t = t\circ p_2\circ\Delta = z\circ q\circ\Delta = z\circ 0 = 0\) because \(\Delta = \ker(q)\). Finally \(z\circ q = t\circ p_2 = 0\circ p_2 = 0\) and since \(q\) is epic \(z = 0\); this proves that \(r\) is epic and by Proposition \ref{prop:iso_mono_epi} that it is an isomorphism.
\end{proof}

\begin{definition}
  \label{def:difference_on_abelian_categories}
  Consider two arrows \(f,g\colon B\to A\) in an abelian category \(\cat{C}\). Using the notation of Lemma \ref{lemma:difference_lemma} we define a new arrow
  \begin{align*}
    \sigma_A\colon A\times A\overset{q}{\longrightarrow}Q\overset{r^{-1}}{\longrightarrow}A
  \end{align*}
  and call it {\bf difference} on \(A\). This arrow allows us to define the difference of two morphisms \(f,g\colon B\to A\) in the following way
  \begin{align*}
    f-g\colon B\overset{\vvectormini{f}{g}}{\longrightarrow} A\times A\overset{\sigma_A}{\longrightarrow} A
  \end{align*}
  And now the sum of two morphisms can be defined as \(f + g = f - (0 - g)\).
\end{definition}

\begin{remark}
  \label{rem:additive_structure}
  Notice that this last definition is not arbitrary but mirrors exactly what happens in an additive category; see proof of Proposition \ref{prop:uniqueness_of_additive_structure}, marked relation.
\end{remark}

\begin{remark}
  \label{remark:product_of_morphisms}
  Given an arrow \(f\colon A\to B\) we define \(f\times f = \vvectormini{f\circ p_1}{f\circ p_2}\colon A\times A\to B\times B\) where \(p_1\) and \(p_2\) are the projections of the product \(A\times A\). With this notation, via Remark \ref{remark:product_property}, we obtain the following:
  \begin{enumerate}[label=(\arabic*)]
  \item given \(f,g\colon A\to B\) and \(h\colon B\to C\) then \((h\times h)\circ\vvectormini{f}{g} = \vvectormini{h\circ f}{h\circ g}\),
  \item given \(f,g\colon A\to B\) and \(h\colon C\to A\) then \(\vvectormini{f}{g}\circ h = \vvectormini{f\circ h}{g\circ h}\).
  \end{enumerate}
  These two facts will be used to prove the next two results.
\end{remark}

\begin{lemma}
  \label{lemma:sigma_lemma}
  In an abelian category if \(f\colon B\to A\) then \(f\circ\sigma_B = \sigma_A\circ(f\times f)\).
\end{lemma}

\begin{figure}[h]
  \begin{center}
  \begin{tikzcd}[sep=huge]
    B \ar[d, tail, "\Delta_B"'] \ar[rrr, "f"] & & & A \ar[d, tail, "\Delta_A"]\\
    B\times B \ar[rrr, "f\times f"] \ar[dr, "\sigma_B"] & & & A\times A \ar[dl, "\sigma_A"']\\
    & B \ar[r, dashed, "g"] & A & \\
    B \ar[uu, "\vvectormini{1_B}{0}"] \ar[ur, "1_B"] \ar[rrr, "f"] & & & A \ar[uu, "\vvectormini{1_A}{0}"'] \ar[ul, "1_A"']
  \end{tikzcd}
\end{center}
  \caption{}
  \label{diagram:sigma}
\end{figure}

\begin{proof}
  We reference diagram \ref{diagram:sigma} and start with the observation that:
  \begin{align*}
    \sigma_A\circ(f\times f)\circ\Delta_B = \sigma_A\circ\Delta_A\circ f = 0\circ f = 0
  \end{align*}
  because \(\sigma_A\cong\coker(\Delta_A)\) from definition \ref{def:difference_on_abelian_categories}. This gives us the unique factorization \(\sigma_A\circ(f\times f) = g\circ\sigma_B\) because \(\sigma_B\cong\coker(\Delta_B)\); to complete the proof we will show that \(f = g\).

  From the proof of Lemma \ref{lemma:difference_lemma} we have \(r = q\circ\vvectormini{1_A}{0}\) so we get that \(\sigma_A\circ\vvectormini{1_A}{0} = r^{-1}\circ q\circ\vvectormini{1_A}{0} = r^{-1}\circ r = 1_A\); similarly \(\sigma_B\circ\vvectormini{1_B}{0} = 1_B\).
  From Remark \ref{remark:product_of_morphisms} we have
  \begin{align*}
    (f\times f)\circ\vvector{1_B}{0} = \vvector{f\circ 1_B}{0} = \vvector{1_A\circ f}{0} = \vvector{1_A}{0}\circ f
  \end{align*}
  and finally
  \begin{align*}
    g = g\circ\sigma_B\circ\vvector{1_B}{0} = \sigma_A\circ(f\times f)\circ\vvector{1_B}{0} = \sigma_A\circ\vvector{1_A}{0}\circ f = f
  \end{align*}
\end{proof}

\begin{theorem}[additivity of abelian categories]
  \label{teo:additivity_of_abelian_categories}
  Every abelian category is additive.
\end{theorem}

\begin{figure}[h]
  \begin{center}
  \begin{tikzcd}[sep=large]
    (A\times A)\times (A\times A) \ar[r, "p_i\times p_i"] \ar[d, "\sigma_{A\times A}"] & A\times A \ar[d, "\sigma_A"]\\
    A\times A \ar[r, "p_i"] & A
  \end{tikzcd}\begin{tikzcd}[sep=large]
    (A\times A)\times(A\times A) \ar[r, "\sigma_A\times\sigma_A"] \ar[d, "\sigma_{A\times A}"] & A\times A \ar[d, "\sigma_A"]\\
    A\times A \ar[r, "\sigma_A"] & A
  \end{tikzcd}
\end{center}

  \caption{}
  \label{diagram:abelian_additivity}
\end{figure}


\begin{proof}
  By applying Lemma \ref{lemma:sigma_lemma} with \(B = A\times A\) and \(f = p_i\) for \(i = 1,2\) we obtain the diagram on the left in Diagram \ref{diagram:abelian_additivity}, if instead we apply it with \(B=A\times A\) but \(f=\sigma_A\) we get the diagram on the right.

  Given four morphisms \(a,b,c,d\colon C\to A\) from Definition \ref{def:difference_on_abelian_categories} we have:
  \begin{gather*}
    \vvector{a}{b}-\vvector{c}{d} = \sigma_{A\times A}\circ\vvectorfour{a}{b}{c}{d}
  \end{gather*}
  From the left diagram:
  \begin{gather*}
    p_1\circ\sigma_{A\times A}\circ\vvectorfour{a}{b}{c}{d} = \sigma_A\circ(p_1\times p_1)\circ\vvectorfour{a}{b}{c}{d} = \sigma_A\circ\vvector{a}{c} = a - c\\
    p_2\circ\sigma_{A\times A}\circ\vvectorfour{a}{b}{c}{d} = \sigma_A\circ(p_2\times p_2)\circ\vvectorfour{a}{b}{c}{d} = \sigma_A\circ\vvector{b}{d} = b - d
  \end{gather*}
  so by Remark \ref{remark:product_property} we obtain \(\vvectormini{a}{b}-\vvectormini{c}{d} = \vvectormini{a - c}{b - d}\). From this last fact and the diagram on the right we obtain
  \begin{align*}
    (a - c)-(b - d) &= \sigma_A\circ\vvectormini{a - c}{b - d} = \sigma_A\circ\left(\vvectormini{a}{b}-\vvectormini{c}{d}\right)\\
                        &= \sigma_A\circ\sigma_{A\times A}\circ\vvectorfour{a}{b}{c}{d}\\
                        &= \sigma_A\circ(\sigma_A\times\sigma_A)\circ\vvectorfour{a}{b}{c}{d}\\
                        &= \sigma_A\circ\vvector{a - b}{c - d} = (a - b)-(c - d)
  \end{align*}
  In the proof of Lemma \ref{lemma:sigma_lemma} we showed that \(\sigma_A\circ\vvectormini{1_A}{0}=1_A\) and this implies that given \(a\colon C\to A\) we have \(a - 0 = \sigma_A\circ\vvectormini{a}{0}=\sigma_A\circ\vvectormini{1_A}{0}\circ a = a\), similarly from \(\sigma_A\circ\Delta_A = 0\) we obtain \(a - a =\sigma_A\circ\vvectormini{a}{a} = \sigma_A\circ\Delta_A\circ a = 0\); this gives us inverses and the identity element of \(\cat{C}(C, A)\). Finally using the relation \((a - c) - (b - d) = (a - b) - (c - d)\) we can prove commutativity and associativity of the sum in \(\cat{C}(C, A)\); thus that it is an abelian group.
  \begin{align}
    (0 - b) - c &= (0 - b) - (c - 0) = (0 - c) - (b - 0) = (0 - c) - b\\
    0 - (0 - d) &= (d - d) - (0 - d) = (d - 0) - (d - d) = (d - 0) - 0 = d\\
    b + c       &= b - (0 - c) \eqtext{(2)} (0 - (0 - b)) - (0 - c)\nonumber\\
                &= (0 - 0) - ((0 - b) - c) \eqtext{(1)} (0 - 0) - ((0 - c) -b))\nonumber\\
                &= (0 - (0 -c)) - (0 - b) \eqtext{(2)} c - (0 - b) = c + b\nonumber\\
    b + (0 - c) &= b - (0 - (0 - c)) \eqtext{(2)} b - c\\
    b + (0 - b) &\eqtext{(3)} b - b = 0\nonumber\\
    0 - (c - d) &= (0 - 0) - (c - d) = (0 - c) - (0 - d) = (0 - c) + d\\
    0 - (c + d) &= 0 - (c - (0 - d)) \eqtext{(4)} (0 - c) + (0 - d) \eqtext{(3)} (0 - c) - d\\
    (a - b) + d &= (a - b) - (0 - d) = (a - 0) - (b - d) = a - (b - d)\\
    (a + b) + d &= (a - (0 - b)) + d \eqtext{(6)} a - ((0 - b) - d) \eqtext{(5)} a - (0 - (b + d))\nonumber\\
                &= a + (b + d)\nonumber\\
    0 + a       &= 0 - (0 - a) = a\nonumber
  \end{align}
  Lastly we need to prove that the composition of arrows is a group homomorphism in both variables:
  \begin{itemize}
  \item given \(x\colon X\to C\) we have
    \begin{align*}
      (a - b)\circ x = \sigma_A\circ\vvector{a}{b}\circ x = \sigma_A\circ\vvector{a\circ x}{b\circ x} = (a\circ x) - (b\circ x),
    \end{align*}
  \item given \(y\colon A\to Y\), by using Lemma \ref{lemma:sigma_lemma} again, we have
    \begin{align*}
      y\circ(a - b) &= y\circ\sigma_A\circ\vvector{a}{b} = \sigma_Y\circ(y\times y)\circ\vvector{a}{b}\\
                    &= \sigma_Y\circ\vvector{y\circ a}{y\circ b} = (y\circ a) - (y\circ b),
    \end{align*}
  \end{itemize}
  and this completes the proof.
\end{proof}

\begin{corollary}
  \label{corollary:ab_additive}
  \(\catname{Ab}\) and \(\catname{Mod}_R\) are additive categories and the sum of morphisms is defined pointwise.
\end{corollary}

\begin{proof}
  \(\catname{Ab}\) and \(\catname{Mod}_R\) are abelian categories and thus, from Theorem \ref{teo:additivity_of_abelian_categories}, are additive as well. Moreover we know from Example \ref{ex:preadditive_category} that the pointwise sum of morphisms is a valid additive structure on both categories; by Proposition \ref{prop:uniqueness_of_additive_structure} it must be unique.
\end{proof}

\newpage
\section{Additive functors}
\label{sec:functors}
Consider two sets and the functions between them; if we enrich these two sets in some way (e.g. we make them into groups) then only some of the old functions will be ``well-behaved'' with respect to the newly introduced structure (e.g. only those functions that are group homomorphisms). Functors are mappings between categories so it is to be expected that if the categories are enriched in some way (e.g. by requiring every hom-set to be a group, as in Definition \ref{def:preadditive_category}) we have to restrict the family of ``meaningful functors'' to those that respect the chosen enrichment. The notion of additive functor thus arises.

\begin{definition}
  \label{def:additive_functor}
  A functor \(F\colon\cat{A}\to\cat{B}\) between preadditive categories is {\bf additive} if for all pairs of objects \(A,B\in\cat{A}\) the restriction of the funtor \(F_{A,B}\colon\cat{A}(A,B)\to\cat{B}(F(A),F(B))\) between the hom-sets is a group homomorphism.
\end{definition}

\begin{example}
  \label{ex:additive_functor}
  As we will prove in Proposition \ref{prop:representables_are_additive} the hom-functor \(\homset{\cat{C}}{A}{-}\) from an additive category \(\cat{C}\) to \(\catname{Ab}\), is additive.
\end{example}

\begin{proposition}
  \label{prop:category_of_additive_functors}
  Given two preadditive categories \(\cat{A},\cat{B}\) with \(\cat{A}\) small then the category \(\add{\cat{A}}{\cat{B}}\) of additive functors and natural transformations is preadditive. Moreover the preadditive structure is defined pointwise.
\end{proposition}

\begin{proof}
  Given natural transformations \(\nat{\alpha,\beta}{F}{G}\) we define their sum \(\alpha + \beta\) by \((\alpha+\beta)_A=\alpha_A+\beta_A\).
  We need to check that \(\nat{\alpha+\beta}{F}{G}\) is again a natural transformation. Let \(f\colon A\to B\) be an arrow in \(\cat{A}\); by using that \(\alpha,\beta\) are natural transformations we have:
  \begin{align*}
    (\alpha+\beta)_B \circ F(f) &= (\alpha_B + \beta_B)\circ F(f)\\
                                &= (\alpha_B\circ F(f))+(\beta_B\circ F(f))\\
                                &= (G(f)\circ\alpha_A)+(G(f)\circ\beta_A)\\
                                &= G(f)\circ(\alpha_A+\beta_A)\\
                                &= G(f)\circ(\alpha+\beta)_A
  \end{align*}
  so \(\alpha+\beta\) is a natural transformation between \(F\) and \(G\).
  The identity element of \(\text{Nat}(F,G)\) is obviously the natural transformation \(\nat{0}{F}{G}\) obtained by setting \(0_A\) to be the identity element of the group \(\cat{B}(F(A),G(A))\); \(\nat{0}{F}{G}\) is a natural transformation because composing with a zero arrow yields another zero arrow and zero arrows are unique. The inverse of a natural transformation \(\nat{\alpha}{F}{G}\) is the natural transformation \(-\alpha\) defined by \((-\alpha)_A = -\alpha_A\); \(\nat{-\alpha}{F}{G}\) is a natural transformation because \(\alpha\) is.
  Finally commutativity and associativity of the sum of natural transformations as well as bilinearity of their composition follow from the fact that \(\cat{B}\) is preadditive. It follows that \([\cat{A}, \cat{B}]\) is preadditive and, since it's a full subcategory, so is \(\add{\cat{A}}{\cat{B}}\).
\end{proof}

One could rightfully wonder why functors that preserve preadditive structure are not called ``preadditive functors''; as intuitively the name ``additive functor'' should be reserved for those functors between additive categories that preserve the additive structure. The following result shows that, between additive categories, ``preadditive functors'' are the same as additive ones.

\begin{proposition}
  \label{prop:additive_criteria}
  For a functor \(F\colon\cat{A}\to\cat{B}\) between additive categories the following are equivalent:
  \begin{enumerate}[label=(\arabic*)]
  \item \(F\) is additive,
  \item \(F\) preserves biproducts,
  \item \(F\) preserves finite products,
  \item \(F\) preserves finite coproducts.
  \end{enumerate}
\end{proposition}

\begin{proof}
  \((3)\) and \((4)\) are equivalent by duality and imply the preservation of both binary products and binary coproducts thus biproducts i.e. \((2)\).\\
  To prove that \((2)\Rightarrow(3)\) we first recall that binary products in an additive category are always biproducs (Proposition \ref{prop:existence_of_products}) and so are preserved by \(F\); to prove that all finite products are preserved we now need to prove that the product of the empty family (the terminal object \({\bf 0}\)) is preserved. For every \(B\in\cat{B}\) there is at least an arrow \(B\to F({\bf 0})\) because \(\cat{B}\) is additive and thus has the zero arrows. We consider the biproduct \({\bf 0}\oplus{\bf 0}\) in \(\cat{A}\), since \({\bf 0}\) is terminal we have that the two projections \(p_1,p_2\colon{\bf 0}\oplus{\bf 0}\to{\bf 0}\) are really the same arrow; so \(p_1=p_2\). Since \(F\) preserves biproducts by hypothesis we have \(F({\bf 0}\oplus{\bf 0})\cong F({\bf 0})\oplus F({\bf 0})\) and while for the projections we have \(q_1=F(p_1)\) and \(q_2=F(p_2)\) thus \(q_1= q_2\) (since \(p_1=p_2\)). Now given \(f,g\colon B\to F({\bf 0})\), by using Proposition \ref{prop:uniqueness_of_additive_structure}:
  \begin{gather*}
    f - g = \sigma_{F({\bf 0})}\circ\vvector{f}{g} = (q_1 - q_2)\circ\vvector{f}{g} = 0\circ\vvector{f}{g} = 0
  \end{gather*}
  so \(f - g = 0\) and thus \(f = g\). This proves that \(\text{Hom}(B, F({\bf 0}))\) is a singleton and thus \(F({\bf 0})\) is terminal.

  \((1)\Rightarrow(2)\) follows trivially from the definition of a biproduct (see Definition \ref{def:biproduct}); it remains to prove that \((2)\Rightarrow(1)\). We have already proved that if \((2)\) holds then \(F\) preserves the zero object, so it preserves the zero arrows and thus the identity elements of the groups. To prove that \(F\) preserves the difference of arrows, by using Proposition \ref{prop:uniqueness_of_additive_structure} again (particularly the marked expression \((*)\) in its proof), it is sufficient to prove that \(F\) preserves the difference of the projections \(p_1-p_2\) for all \(A\oplus A\). Since \(F\) preserves biproducts \(F(s_1),F(s_2)\) are the injections of \(F(A)\oplus F(A)\) and we have:
  \begin{gather*}
    F(p_1 - p_2) \circ F(s_1) = F((p_1 - p_2)\circ s_1) = F(1_A- 0) = 1_{F(A)}\\
    (F(p_1) - F(p_2))\circ F(s_1) = F(p_1\circ s_1) - F(p_2\circ s_1) = F(1_A) - 0 = 1_{F(A)}\\
    F(p_1 - p_2) \circ F(s_2) = F((p_1 - p_2)\circ s_2) = F(0 - 1_A) = -1_{F(A)}\\
    (F(p_1) - F(p_2))\circ F(s_2) = F(p_1\circ s_2) - F(p_2\circ s_2) = 0 - F(1_A) = -1_{F(A)}
  \end{gather*}
  so \(F(p_1 - p_2) = F(p_1) - F(p_2)\) and thus \((1)\) holds.
\end{proof}

\begin{example}
  \label{ex:category_of_additive_functors}
  With Example \ref{ex:ring_preadditive} in mind we will now show that, for any ring \(R\), the category \(\catname{Mod}_R\) of left \(R\)-modules is isomorphic to the category \(\text{Add}(\cat{R},\catname{Ab})\) of additive functors from \(\cat{R}\) to \(\catname{Ab}\) and natural transformations between them.
  
  Given a left \(R\)-module \(M\) we define a functor \(F\colon\cat{R}\to\catname{Ab}\) by setting \(F(*) = (M, +)\) and \(F(r)\colon M\to M,x\mapsto r\cdot x\) for \(r\in R\). The definition of R-module ensures that \(F\) is additive and so \(F\in\text{Add}(\cat{R},\catname{Ab})\). Given a linear function \(\fun{f}{M}{N}\) on \(R\)-modules let \(G\) be the functor associated to \(N\) while \(F\) is the functor associated to \(M\) as above. We define a natural transformation \(\nat{\varphi}{F}{G}\) by setting \(\varphi_* = f\). Let \(\fun{r}{*}{*}\) be an arrow of \(\cat{R}\) and \(x\in M\):
  \begin{equation*}
    (f\circ(r\cdot-))(x) = f(r\cdot x) = r\cdot f(x) = ((r\cdot-)\circ f)(x)
  \end{equation*}
  so \(\varphi\) is really a natural transformation.

  Conversely an additive functor \(F\in\add{\cat{R}}{\catname{Ab}}\) gives us an abelian group \(F(*)\) and for every \(r\in R\) a group homomorphism \(\fun{F(r)}{F(*)}{F(*)}\). We can easily define a scalar multiplication on \(F(*)\) as follows
  \fundef{\cdot}{R\times F(*)}{F(*)}{(r,x)}{r\cdot x = F(r)(x)}
  From the fact that \(F(r)\) is a group homomorphism we obtain that \(r\cdot (x + y) = r\cdot x + r\cdot y\) and from the functoriality of \(F\) we obtain \((r\cdot s)\cdot x = r\cdot(s\cdot x)\) and \(1\cdot x = x\). Moreover from the additivity of \(F\) we have
  \begin{align*}
    (r + s)\cdot x &= F(r + s)(x)\\
                   &= (F(r) + F(s))(x) \tag{\(*\)}\\
                   &= F(r)(x) + F(s)(x)\\
                   &= r\cdot x + s\cdot x
  \end{align*}
  and this completes the construction of an \(R\)-module out of \(F\).

  Finally if \(\nat{\varphi}{F}{G}\) is a natural transformation then the group homomorphism \(\fun{\varphi_*}{F(*)}{G(*)}\) is such that \(\varphi_*\circ F(r) = G(r)\circ\varphi_*\) for every \(r\in R\). Taking \(x\in F(*)\) we have
  \begin{equation*}
    \varphi_*(r\cdot x) = \varphi_*(F(r)(x)) = G(r)(\varphi_*(x)) = r\cdot\varphi_*(x)
  \end{equation*}
  so \(\varphi_*\) is an \(R\)-linear map.

  The two functors we have constructed are mutually inverse, so \(\catname{Mod}_R\) and \(\add{\cat{R}}{\catname{Ab}}\) are isomorphic.
\end{example}

\begin{remark}
  \label{rem:uniqueness_is_cool}
  Notice that in the above example, particularly at the marked line \((*)\), we utilize the fact that the sum of morphisms in \(\catname{Ab}\) is defined pointwise and we can safely do so because of Corollary \ref{corollary:ab_additive}.
\end{remark}

When working with a standard category \(\cat{C}\) every object \(C\in\cat{C}\) gives a (covariant) hom-functor \(\homset{\cat{C}}{C}{-}\) from \(\cat{C}\) to \(\catname{Set}\) defined in the usual way. A functor \(\fun{F}{\cat{C}}{\catname{Set}}\) is related to the hom-functors of \(\cat{C}\) by the Yoneda Lemma; that we can then use to realize a contravariant embedding of \(\cat{C}\) in \([\cat{C},\catname{Set}]\). The following results show that the same can be done for additive categories, without losing the group structure on the hom-sets.

\begin{definition}
  \label{def:representable_functor}
  Consider the functor
  \fundef{\homset{\cat{A}}{A}{-}}{\cat{A}}{\catname{Ab}}{B \ar[d, "\displaystyle f"']\\ C}{\homset{\cat{A}}{A}{B} \ar[d, "\displaystyle\postcomp{f}"']\\ \homset{\cat{A}}{A}{C}}
  where \(\cat{A}\) is an additive category, \(A\) and \(B\) elements of \(\cat{A}\), \(\fun{f}{B}{C}\) an arrow of \(A\) and \(\fun{\postcomp{f}}{\homset{\cat{A}}{A}{B}}{\homset{\cat{A}}{A}{C}}\) the post-composition with \(f\). We call such a functor a {\bf hom-functor} and a functor naturally isomorphic to a hom-functor a {\bf representable-functor}.
\end{definition}

\begin{remark}
  \label{rem:representable_natural_transformation}
  Every morphism \(\fun{f}{A}{B}\) induces a natural transformation \(\nat{\homset{\cat{A}}{f}{-}}{\homset{\cat{A}}{B}{-}}{\homset{\cat{A}}{A}{-}}\) with components \(\homset{\cat{A}}{f}{-}_C = \homset{\cat{A}}{f}{C} = \precomp{f}\); where we denote with \(\precomp{f}\) the pre-composition with \(f\).
\end{remark}

\begin{proposition}
  \label{prop:representables_are_additive}
  The hom-functor \(\homset{\cat{A}}{A}{-}\) is additive.
\end{proposition}

\begin{proof}
  Given \(f,g\colon X\to Y\) for \(h\in\cat{A}(A, X)\) we have:
  \begin{align*}
    \postcomp{(f+g)}(h) &= (f + g)\circ h\\
                        &= (f\circ h) + (g\circ h)\\
                        &= \postcomp{f}(h) + \postcomp{g}(h)\\
                        &= (\postcomp{f} + \postcomp{g})(h)
  \end{align*}
  so \(\postcomp{(f + g)} = \postcomp{f} + \postcomp{g}\); it follows that \(\homset{\cat{A}}{A}{-}\) is additive.
\end{proof}

\begin{lemma}[Additive Yoneda Lemma]
  \label{lemma:additive_yoneda_lemma}
  If \(\cat{A}\) is a preadditive category, \(A\in\cat{A}\) and \(F\colon\cat{A}\to\catname{Ab}\) is an additive functor then there is an isomorphism of groups
  \begin{gather*}
    \theta_{F,A}\colon\text{Nat}(\cat{A}(A,-),F)\overset{\cong}{\longrightarrow}F(A)
  \end{gather*}
  moreover this isomorphism is natural in \(A\) and, when \(\cat{A}\) is small, also in \(F\).
\end{lemma}

\begin{proof}
  We first prove the existence of a bijection between \(\text{Nat}(\cat{A}(A,-),F)\) and \(F(A)\) that is natural in both \(F\) and \(A\). Then we will prove that such bijection is really a group homomorphism, completing the proof.

  Given a natural transformation \(\alpha\colon\cat{A}(A,-)\Rightarrow F\) we define \(\theta_{F,A}(\alpha) = \alpha_A(1_A)\). Given \(a\in F(A)\) and \(B\in\cat{A}\) we define a map
  \fundef{\tau(a)_B}{\homset{\cat{A}}{A}{B}}{F(B)}{f}{F(f)(a)}
  Using the additivity of \(F\) and the additive structure of \(\catname{Ab}\) we have
  \begin{align*}
    \tau(a)_B(f+g) &= F(f + g)(a)\\
                   &= (F(f) + F(g))(a)\\
                   &= F(f)(a) + F(g)(a)\\
                   &= \tau(a)_B(f) + \tau(a)_B(g).
  \end{align*}
  So \(\tau(a)_B\) is a group homomorphism. Given an arrow \(g\colon B\to C\) in \(\cat{A}\) and an \(f\in\cat{A}(A,B)\) we obtain
  \begin{align*}
    (F(g)\circ\tau(a)_B)(f) &= (F(g)\circ F(f))(a)\\
                            &= F(g\circ f)(a)\\
                            &= \tau(a)_C(g\circ f)\\
                            &= (\tau(a)_C\circ\postcomp{g})(f)
  \end{align*}
  so \(F(g)\circ\tau(a)_B = \tau(a)_C\circ\postcomp{g}\) and thus every \(a\in F(A)\) gives us a natural trasformation \(\nat{\tau(a)}{\homset{\cat{A}}{A}{-}}{F}\) with components \(\tau(a)_B\) for \(B\in\cat{A}\). Now we have that
  \begin{itemize}
  \item  if \(a\in F(A)\) then one has
    \begin{equation*}
      \theta_{F,A}(\tau(a)) = \tau(a)_A(1_A) = F(1_A)(a) = 1_{F(A)}(a) = a,
    \end{equation*}
  \item if \(\nat{\alpha}{\homset{\cat{A}}{A}{-}}{F}\) and \(\fun{f}{A}{B}\) then
    \begin{align*}
      \tau(\theta_{F,A}(\alpha))_B(f) &= \tau(\alpha_A(1_A))_B(f)\\
                                      &= F(f)(\alpha_A(1_A))\\
                                      &= \alpha_B(\postcomp{f}(1_A))\\
                                      &= \alpha_B(f\circ 1_A)\\
                                      &= \alpha_B(f).
  \end{align*}
  \end{itemize}
  So \(\tau\) and \(\theta_{F,A}\) are mutually inverse assignments and thus bijections.

  To prove the naturality in \(A\) we consider the functor
  \fundef{N}{\cat{A}}{\catname{Ab}}
  {A \ar[d, "\displaystyle f"']\\ B}
  {\nathom{\homset{\cat{A}}{A}{-}}{F} \ar[d, "\displaystyle\precomp{\homset{\cat{A}}{f}{-}}"]\\ \nathom{\homset{\cat{A}}{B}{-}}{F}}
  The following calculations, for \(\alpha\in\nathom{\homset{\cat{A}}{A}{-}}{F}\) show that \(\nat{\eta}{N}{F}\) defined by \(\eta_A = \theta_{F,A}\) is a natural transformation:
  \begin{align*}
    (\theta_{F,B}\circ N(f))(\alpha) &= \theta_{F,B}(\alpha\circ\cat{A}(f,-))\\
                                     &= (\alpha\circ\cat{A}(f,-))_B(1_B) = \alpha_B(f),\\
    (F(f)\circ\theta_{F,A})(\alpha)  &= F(f)(\alpha_A(1_A))\\
                                     &= (\alpha_B\circ\postcomp{f})(1_A) = \alpha_B(f).
  \end{align*}

  If \(\cat{A}\) is small then it makes sense to consider \(\text{Add}(\cat{A},\catname{Ab})\); the category of additive functors from \(\cat{A}\) to \(\catname{Ab}\). Fixing an object \(A\in\cat{A}\) on one hand we consider the functor
  \fundef{M}{\add{\cat{A}}{\catname{Ab}}}{\catname{Ab}}
  {F \ar[d, "\displaystyle\gamma"']\\ G}
  {\nathom{\homset{\cat{A}}{A}{-}}{F} \ar[d, "\displaystyle\postcomp{\gamma}"']\\ \nathom{\homset{\cat{A}}{A}{-}}{G}}
  and on the other the ``evaluation at \(A\)'' functor
  \fundef{\text{ev}_A}{\add{\cat{A}}{\catname{Ab}}}{\catname{Ab}}
  {F \ar[d, "\displaystyle\gamma"']\\ G}
  {F(A) \ar[d, "\displaystyle\gamma_A"']\\ G(A)}
  Now we have:
  \begin{gather*}
    (\theta_{G,A}\circ M(\gamma))(\alpha) = \theta_{G,A}(\gamma\circ\alpha) = (\gamma\circ\alpha)_A(1_A)\\
    (\text{ev}_A(\gamma)\circ\theta_{F,A})(\alpha) = \gamma_A(\alpha_A(1_A)) = (\gamma\circ\alpha)_A(1_A)
  \end{gather*}
  so \(\nat{\mu}{M}{ev_A}\) defined by \(\mu_F = \theta_{F,A}\) is a natural transformation.

  To complete the proof we need to show that \(\tau\) is a group homomorphism. Given \(a,b\in F(A)\) one has
  \begin{equation*}
    \tau(a + b)_B(f) = F(f)(a + b) = F(f)(a) + F(f)(b) = \tau(a)_B(f) + \tau(b)_B(f)
  \end{equation*}
  so \(\tau(a + b) = \tau(a) + \tau(b)\); thus \(\tau\) is a group homomorphism.
\end{proof}

\begin{definition}
  \label{def:yoneda_embedding}
  For an additive category \(\cat{A}\) the contravariant functor
  \fundef{\yo^{*}}{\cat{A}}{\add{\cat{A}}{\catname{Ab}}}
  {B \ar[d, "\displaystyle f"']\\ C}
  {\homset{\cat{A}}{B}{-} \\ \homset{\cat{A}}{C}{-} \ar[u, "\displaystyle\homset{\cat{A}}{f}{-}"']}
  is called the (contravariant) {\bf Yoneda embedding} of \(\cat{A}\) in \(\add{\cat{A}}{\catname{Ab}}\).
\end{definition}

\begin{remark}
  \label{rem:yoneda_embedding_full_and_faithful}
  By the Yoneda Lemma the Yoneda embedding of \(\cat{A}\) in \(\add{\cat{A}}{\catname{Ab}}\) is fully faithful.
\end{remark}

The Yoneda Lemma and the Yoneda Embedding, while interesting in their own right, are useful and necessary tools for our goal. Indeed in \S\ref{sec:5} we make vast use of them to prove a number of results that will ultimately lead us to the Faithful Embedding Theorem (\ref{teo:faithful_embedding}) and, lastly, to Mitchell's Embedding Theorem (\ref{teo:mitchell}).

\newpage
\section{Exact sequences and exact functors}
\label{sec:exact}
\begin{definition}
  \label{def:exact_sequence}
  In an abelian category a pair of composable arrows \(\fun{f}{A}{B},\fun{g}{B}{C}\) is an {\bf exact sequence} when the image of \(f\) is the kernel of \(g\).
\end{definition}

\begin{figure}[h]
  \begin{center}
  \begin{tikzcd}[row sep=huge]
    A \ar[rr, "f"] \ar[dr, two heads, "p"] & & B \ar[rr, "g"] \ar[dr, two heads, "q"] & & C\\
    & I \ar[ru, tail, "i"] & & J \ar[ru, tail, "j"] &
  \end{tikzcd}
\end{center}
  \caption{(here \(I\) and \(J\) are the images of \(f\) and \(g\)).}
  \label{diagram:exact_sequence}
\end{figure}

\begin{lemma}
  \label{lemma:ker}
  With reference to Diagram \ref{diagram:exact_sequence} \(\ker(g) = \ker(q)\) and \(\im(f) = \im(i)\).
\end{lemma}

\begin{proof}
  We prove only that \(\ker(g) = \ker(q)\) as \(\im(f) = \im(i)\) is obtained similarly. First let's notice that \(g\circ\ker(g) = 0\) so \(j\circ q\circ\ker(g) = 0\) and thus \(q\circ\ker(g) = 0\) because \(j\) is a monomorphism. We shall now prove that \(\ker(g)\) factorizes uniquely any arrow \(x\) such that \(q\circ x = 0\). If this last relation holds we obtain \(j\circ q\circ x = j\circ 0\) and thus \(g\circ x = 0\); a unique factorization of \(x\) through \(ker(g)\) exists from the definition of kernel.
\end{proof}

\begin{proposition}
  \label{prop:characterization_of_exact_sequences}
  Consider Diagram \ref{diagram:exact_sequence}. The following conditions are equivalent:
  \begin{enumerate}[label=(\arabic*)]
  \item \((f, g)\) is an exact sequence,
  \item \((i, g)\) is an exact sequence,
  \item \((f, q)\) is an exact sequence,
  \item \((i, q)\) is an exact sequence.
  \end{enumerate}
\end{proposition}

\begin{proof}
  The result follows immediately from Lemma \ref{lemma:ker}.
\end{proof}

\begin{remark}
  \label{rem:coexact_sequences}
  The concept of an exact sequence is autodual.
\end{remark}

\begin{proof}
  We have that \((f, g)\) is {\bf coexact} in an abelian category \(\cat{C}\) if \((\op{g}, \op{f})\) is exact in \(\op{\cat{C}}\). From Proposition \ref{prop:characterization_of_exact_sequences} this is equivalent to the condition \(\op{q} = \ker(\op{i})\) (again referencing Diagram \ref{diagram:exact_sequence}) that is in turn equivalent to \(q = \coker(i)\). Now recall that \(i\) is mono and thus, by Corollary \ref{cor:mono_kernel_cokernel_epi_cokernel_kernel}, \(i=\ker(\coker(i)) = \ker(q)\); this proves that \((f, g)\) is exact by Proposition \ref{prop:characterization_of_exact_sequences}. Dually if \((f, g)\) is exact then it is also coexact.
\end{proof}

\begin{definition}
  \label{def:longer_exact_sequence}
  An arbitrarily long (finite or infinite) sequence of composable morphisms in an abelian category is called {\bf exact} if every pair of consecutive morphisms is exact.
\end{definition}

\begin{proposition}
  \label{prop:sequences_with_zero}
  In an abelian category the following hold.
  \begin{enumerate}[label=(\arabic*)]
  \item \begin{tikzcd}[sep=small]{\bf 0} \ar[r, ""] & A \ar[r, "f"] & B\end{tikzcd} is exact if and only if \(f\) is mono,
  \item \begin{tikzcd}[sep=small]B \ar[r, "g"] & A \ar[r, ""] & {\bf 0}\end{tikzcd} is exact if and only if \(g\) is epi,
  \item \begin{tikzcd}[sep=small]{\bf 0} \ar[r, ""] & A \ar[r, "f"] & B \ar[r, "g"] & C\end{tikzcd} is exact if and only if \(f = \ker(g)\),
  \item \begin{tikzcd}[sep=small]C \ar[r, "g"] & B \ar[r, "f"] & A \ar[r, ""] & {\bf 0}\end{tikzcd} is exact if and only if \(f = \coker(g)\).
  \end{enumerate}
\end{proposition}

\begin{proof}
  By duality it is sufficient to prove (1) and (3) only.

  Since \({\bf 0}\) is terminal \begin{tikzcd}[sep=small]{\bf 0} \ar[r, ""] & A\end{tikzcd} is monic so \begin{tikzcd}[sep=small]{\bf 0} \ar[r, ""] & A \ar[r, "f"] & C\end{tikzcd} is exact if and only if \(\ker(f) = 0\) but this is equivalent to saying that \(f\) is monic by Proposition \ref{prop:mono_characterization} thus (1) holds.

  Now if \begin{tikzcd}{\bf 0} \ar[r, ""] & A \ar[r, "f"] & B \ar[r, "g"] & C\end{tikzcd} is exact in particular we have that \((0_{{\bf 0}\to A}, f)\) and \((f, g)\) are exact. By (1) we have that \(f\) is mono and thus, since a mono image factors as itself, \(f = \ker(g)\). On the other hand if \(f = \ker(g)\) then \(f\) is mono and the pair \((f, g)\) is obviously exact; from (1) follows that the pair \((0_{{\bf 0}\to A})\) is exact too and thus the whole sequence is.
\end{proof}

\begin{definition}
  \label{def:short_exact_sequence}
  In an abelian category an exact sequence of the form \begin{tikzcd}[sep=small]{\bf 0} \ar[r, ""] & A \ar[r, "f"] & B \ar[r, "g"] & C \ar[r, ""] & {\bf 0}\end{tikzcd} is called a {\bf short-exact sequence}.
\end{definition}

\begin{proposition}
  \label{prop:short_exact_sequences_equivalences}
  For a short exact sequence as in Definition \ref{def:short_exact_sequence} the following are equivalent.
  \begin{enumerate}[label=(\arabic*)]
  \item There is an arrow \(s\colon C\to B\) such that \(g\circ s = 1_C\),
  \item there is an arrow \(r\colon B\to A\) such that \(r\circ f = 1_A\),
  \item there are arrows \(s\colon C\to B\) and \(r\colon B\to A\) such that \((B, r, g, f, s)\) is the biproduct of \(A\) and \(C\).
  \end{enumerate}
\end{proposition}

\begin{proof}
  By duality is sufficient to prove that (1) is equivalent to (3).

  If (3) holds then (1) holds by the definition of biproduct (see also Proposition \ref{prop:existence_of_products}).

  Assume (1). We observe that \(g\circ(1_B - s\circ g) = g - g = 0\) and that, by Proposition \ref{prop:sequences_with_zero}, \(f = \ker(g)\); we thus obtain a factorization \(1_B-s\circ g = f\circ r\) for a unique arrow \(r\colon B\to A\). Now, since \(g\circ f = 0\) by exactness of \((f, g)\), we have that
  \begin{gather*}
    f\circ r\circ f = (1_B - s\circ g)\circ f = f - s\circ g\circ f = f = f\circ 1_A
  \end{gather*}
  and, since \(f\) is mono by Proposition \ref{prop:sequences_with_zero} again, we obtain \(r\circ f = 1_A\). Finally we obtain
  \begin{gather*}
    f\circ r\circ s = (1_B - s\circ g)\circ s = s - s\circ g\circ s = s - s = 0
  \end{gather*}
  from which \(r\circ s = 0\) (again because \(f\) is monic); moreover \(1_B - s\circ g = f\circ r\) gives us \(f\circ r + s\circ g = 1_B\). This proves, by Definition \ref{def:biproduct}, that \(B\) is really the biproduct of \(A\) and \(C\).
\end{proof}

\begin{definition}
  \label{def:split_exact_sequence}
  In an abelian category an exact sequence that respects one of the conditions of Proposition \ref{prop:short_exact_sequences_equivalences} is said to be a {\bf split-exact} sequence.
\end{definition}

\begin{definition}
  \label{def:exact_functor}
  Let \(\fun{F}{\cat{A}}{\cat{B}}\) be an additive functor between two abelian categories. We say that \(F\) is {\bf left-exact} if it preserves exact sequences of the form \begin{tikzcd}[sep=small]{\bf 0} \ar[r, ""] & A \ar[r, ""] & B \ar[r, ""] & C\end{tikzcd}, that it is {\bf right-exact} if it preserves exact sequences of the form \begin{tikzcd}[sep=small]A \ar[r, ""] & B \ar[r, ""] & C \ar[r, ""] & {\bf 0}\end{tikzcd} and that it is {\bf exact} if it preserves exact sequences of the form \begin{tikzcd}[sep=small]{\bf 0} \ar[r, ""] & A \ar[r, ""] & B \ar[r, ""] & C \ar[r, ""] & {\bf 0}\end{tikzcd}.
\end{definition}

\begin{remark}
  \label{rem:left_preserve_kernels}
  A functor is exact if and only if it is both right-exact and left-exact. By Proposition \ref{prop:sequences_with_zero} a left-exact functor preserves kernels, and a right-exact functor preserves cokernels.
\end{remark}

\begin{example}
  \label{ex:exact_example}
  The representable functors \(\homset{\cat{A}}{A}{-}\) from an abelian category \(\cat{A}\) to \(\catname{Ab}\) are left-exact. A proof of this fact can be found in the proof of Lemma \ref{lemma:left_exactness_of_U}.
\end{example}

\begin{proposition}
  \label{prop:characterization_of_exact_functors}
  For an additive functor \(\fun{F}{\cat{A}}{\cat{B}}\) between abelian categories the following equivalences hold.
  \begin{enumerate}[label=(\arabic*)]
  \item \(F\) is left-exact if and only if it preserves finite limits,
  \item \(F\) is right-exact if and only if it preserves finite colimits,
  \item \(F\) is exact if and only if it preserves finite limits and finite colimits.
  \end{enumerate}
\end{proposition}

\begin{proof}
  By Remark \ref{rem:left_preserve_kernels}, point (1) and point (2)  we immediately obtain (3); by duality it is then sufficient to prove only (1).

  Suppose \(F\) preserves finite limits. Then \(F\) preserves kernels and thus by Proposition \ref{prop:sequences_with_zero} (equivalence (3)) is left-exact.

  Now suppose that \(F\) is left-exact; by Remark \ref{rem:left_preserve_kernels} we know that \(F\) preserves kernels and since \(F\) is additive it preserves finite products by Proposition \ref{prop:additive_criteria}. In a preadditive category \(\ker(f, g) = \ker(f - g)\) (Proposition \ref{prop:kernels_in_preadditive_category}); \(\cat{A}, \cat{B}\) are abelian, thus preadditive, and so \(F\) preserves equalizers as well. By \cite[2.9.2]{handbook1} \(F\) preserves finite limits.
\end{proof}

Looking back at the previous definitions one might wonder why exact functors aren't simply functors that preserve exact sequences in the sense of Definition \ref{def:exact_sequence}. The following result shows that this idea and the chosen definition are equivalent.

\begin{proposition}
  \label{prop:exact_functor_preserves_exact_sequences}
  Given an additive functor \(\fun{F}{\cat{A}}{\cat{B}}\) between two abelian categories the following conditions are equivalent.
  \begin{enumerate}[label=(\arabic*)]
  \item \(F\) is exact,
  \item \(F\) preserves all exact sequences.
  \end{enumerate}
\end{proposition}

\begin{proof}
  The implication (2)\(\Rightarrow\)(1) is obvious. To prove the converse let \((f, g)\) be an exact sequence in \(\cat{A}\) and consider once again the epi-mono factorizations of \(f\) and \(g\) as in Diagram \ref{diagram:exact_sequence}. From Remark \ref{rem:left_preserve_kernels} we immediately obtain that \(F\) preserves both kernels and cokernels and thus it preserves monomorphisms and epimorphisms (because, by definition of abelian category, they are respectively kernels and cokernels); this shows that the epi-mono factorizations of \(f\) and \(g\) are preserved by \(F\). Finally from Proposition \ref{prop:characterization_of_exact_sequences} we know \((i, q)\) is an exact sequence so \(i = \ker(q)\) and thus \(F(i) = F(\ker(q)) = \ker(F(q))\). This proves that the pair \((F(i), F(q))\) is exact and thus \((F(f), F(g))\) is.
\end{proof}

\begin{proposition}
  \label{prop:left_condition}
  For a left-exact functor \(\fun{F}{\cat{A}}{\cat{B}}\) the following are equivalent.
  \begin{enumerate}[label=(\arabic*)]
  \item \(F\) is exact,
  \item \(F\) preserves epimorphisms.
  \end{enumerate}
\end{proposition}

\begin{proof}
  If \(F\) is exact then by Remark \ref{rem:left_preserve_kernels} it preserves cokernels and thus epimorphisms because in an abelian category every epimorphism is a cokernel.

  Now let's assume that \(F\) preserves epis and consider an exact sequence
  \[\begin{tikzcd}{\bf 0} \ar[r, ""] & A \ar[r, "\alpha"] & B \ar[r, "\beta"] & C \ar[r, ""] & {\bf 0}\end{tikzcd}\]
  in \(\cat{A}\). By applying \(F\) we obtain a sequence 
  \[\begin{tikzcd}{\bf 0} \ar[r, ""] & F(A) \ar[r, "F(\alpha)"] & F(B) \ar[r, "F(\beta)"] & F(C) \ar[r, ""] & {\bf 0}\end{tikzcd}\]
  in \(\cat{B}\). Since \(F\) is left-exact the first two pairs of arrows in this last sequence are exact, we just need to show that also the pair \((F(\beta), 0_{F(C)\to{\bf 0}})\) is. Since \((\beta, 0_{C\to{\bf 0}})\) is exact in \(A\) we obtain that \(\beta\) is epi by Proposition \ref{prop:sequences_with_zero} and since \(F\) preserves epimorphisms \(F(\beta)\) must be epi as well. But now by Proposition \ref{prop:sequences_with_zero} again we obtain the exactness of \((F(\beta), 0_{F(C)\to{\bf 0}})\) and thus \(F\) is exact.
\end{proof}

\newpage
\section{Embedding theorems}
\label{sec:embedding}
This section's objective is that of presenting and proving two important embedding results for abelian categories. The first one, the Faithful Embedding Theorem, shows (as the name implies) that any abelian category can be faithfully and exactly (in the sense of Definition \ref{def:exact_functor}) embedded in \(\catname{Ab}\). The second one, called Mitchell's Embedding Theorem (or the Freyd-Mitchell Embedding Theorem), provides a fully faithful exact embedding of any abelian category in \(\catname{Mod}_R\) for a suitable ring \(R\).

We start by presenting some necessary results about (co)limits and, particularly, (co)limits of functors.

\begin{definition}
  \label{def:filtered_category}
  A category \(\cat{C}\) is {\bf filtered} when:
  \begin{enumerate}[label=(\arabic*)]
  \item it is not empty,
  \item for every pair of objects \(A,B\in\cat{C}\) there is an object \(C\in\cat{C}\) and two arrows \(A\to C, B\to C\),
  \item for every pair of arrows \(\fun{f,g}{A}{B}\) in \(\cat{C}\) there is an arrow \(\fun{h}{B}{C}\) such that \(h\circ f = h\circ g\).
  \end{enumerate}
  Dually \(\cat{C}\) is {\bf cofiltered} when \(\op{\cat{C}}\) is filtered.
\end{definition}

\begin{lemma}
  \label{lemma:filtered_category}
  Let \(\cat{C}\) be a filtered category and \(\cat{D}\) a finite one.
  Then every functor \(\fun{F}{\cat{D}}{\cat{C}}\) has a cocone.
\end{lemma}

\begin{proof}
  See \cite[Lemma 2.13.2]{handbook1}.
\end{proof}

\begin{proposition}[Interchange of (co)limits]
  \label{prop:interchange_of_limits}
  If \(\cat{A}\) is a (co)complete category, \(\cat{C},\cat{D}\) are two small categories and \(\fun{F}{\cat{C}\times\cat{D}}{\cat{A}}\) a functor out of the product category \(\cat{C}\times\cat{D}\) then the following holds.
  \begin{itemize}
  \item If \(\cat{A}\) is complete
    \[\lim_{C\in\cat{C}}\left(\lim_{D\in\cat{D}}F(C, D)\right) \cong \lim_{D\in\cat{D}}\left(\lim_{C\in\cat{C}}F(C, D)\right).\]
  \item If \(\cat{A}\) is cocomplete
    \[\colim_{C\in\cat{C}}\left(\colim_{D\in\cat{D}}F(C, D)\right) \cong \colim_{D\in\cat{D}}\left(\colim_{C\in\cat{C}}F(C, D)\right).\]
  \end{itemize}
\end{proposition}

\begin{proof}
  See \cite[Proposition 2.12.1]{handbook1}.
\end{proof}

\begin{proposition}
  \label{prop:limits_and_filtered_colimits}
  In \(\catname{Ab}\) finite limits commute with filtered colimits. That is if \(\cat{C}\) is a small filtered category, \(\cat{D}\) is a finite category and \(\fun{F}{\cat{C}\times\cat{D}}{\catname{Ab}}\) is a functor then the following holds.
  \begin{equation*}
    \colim_{C\in\cat{C}}\left(\lim_{D\in\cat{D}}F(C,D)\right) \cong \lim_{D\in\cat{D}}\left(\colim_{C\in\cat{C}}F(C,D)\right)
  \end{equation*}
\end{proposition}

\begin{proof}
  See \cite[Theorem 2.13.4]{handbook1}.
\end{proof}

\begin{proposition}
  \label{prop:pointwise_limits}
  Let \(\cat{C},\cat{D}\) be small categories and \(\fun{F}{\cat{D}}[\cat{C},\cat{A}]\) a functor where \(\cat{A}\) is a category such that the (co)limit of \(\fun{F(-)(C)}{\cat{D}}{\cat{A}}\) exists for all \(C\in\cat{C}\). Under these hypotheses \(F\) has a (co)limit too and this (co)limit is calculated pointwise; that is: \((\lim_{D\in\cat{D}}F(D))(C) = \lim_{D\in\cat{D}}(F(D)(C))\) (and similarly for colimits).
\end{proposition}

\begin{proof}
  Each object \(C\in\cat{C}\) induces a functor
  \fundef{F(-)(C)}{\cat{D}}{\cat{A}}{D \ar[d, "f"]\\ D'}{F(D)(C) \ar[d, "F(f)_C"]\\ F(D')(C)}
  and each arrow \(\fun{f}{C}{C'}\) in \(\cat{C}\) induces a natural transformation \(F(-)(f)\) from \(F(-)(C)\) to \(F(-)(C')\) with components \(F(-)(f)_D = F(D)(f)\) for all \(D\in\cat{D}\). Let \((L(C), (p^C_D)_{D\in\cat{D}})\) be the limit of \(F(-)(C)\) for all \(C\in\cat{C}\); now the natural transformation \(F(-)(f)\) induces a factorization \(\fun{L(f)}{L(C)}{L(C')}\) and the relation \(F(D)(f)\circ p^C_D = p^{C'}_D\circ L(f)\) holds. We now have that \(\fun{L}{\cat{C}}{\cat{A}}\) is a functor and \(\nat{p_D}{L}{F(D)}\) a natural transformation with components \(p^C_D\). In order to prove our thesis we need to prove that \((L, (p_D)_{D\in\cat{D}})\) is the limit of \(F\); the pointwise character of this limit is already evinced by the construction of \(L(C)\).

  To show that \((L, (p_D)_{D\in\cat{D}})\) is a cone on \(F\) consider an arrow \(\fun{d}{D}{D'}\) in \(\cat{D}\). Since the \(p^C_D\)'s are projections of the limit of \(F(-)(C)\) we have \(F(d)(C)\circ p^C_D=p^C_{D'}\). This immediately gives us that \(F(d)\circ p_D = p_{D'}\) and so \((L, (p_D)_{D\in\cat{D}})\) is a cone on \(F\).

  Finally let \((M, (q_D)_{D\in\cat{D}})\) be a cone on \(F\). Each \((M(C), (q^C_D)_{D\in\cat{D}})\), where we indicate with \(q^C_D\) the component of \(q_D\) indexed by \(C\), is a cone on \(F(-)(C)\) because we have, for any \(\fun{d}{D}{D'}\) in \(\cat{D}\), that \(F(d)\circ q_D = q_{D'}\) and so \(F(d)(C)\circ q^C_D = q^C_{D'}\). This fact yields a unique \(\fun{r_C}{M(C)}{L(C)}\) such that \(q^C_D = p^C_D\circ r_C\); moreover those \(r_C\)'s are the components of a natural transformation \(\nat{r}{M}{L}\). This follows by the following computation together with the uniqueness of the factozation tough a limit. Given \(\fun{f}{C}{C'}\) then we have
  \begin{align*}
    p^{C'}_D\circ L(f)\circ r_C &= F(D)(f)\circ p^C_D\circ r_C\\
                             &= F(D)(f)\circ q^C_D\\
                             &= q^{C'}_D\circ M(f)\\
                             &= p^{C'}_D\circ r_{C'}\circ M(f).
  \end{align*}
  Clearly we have \(q_D=p_D\circ r\) and \(r\) is the only natural transformation with such property because the \(r_C\)'s are unique.

  One can argue in an analogous way for colimits, thus the proof is complete.
\end{proof}

\begin{theorem}
  \label{teo:pointwise_limits}
  If \(\cat{D}\) is a (co)complete category and \(\cat{C}\) a small category then the category \([\cat{C},\cat{D}]\) is (co)complete. Moreover (co)limits in \([\cat{C}, \cat{D}]\) are computed pointwise.
\end{theorem}

\begin{proof}
  Immediate from Proposition \ref{prop:pointwise_limits}.
\end{proof}

\begin{proposition}
  \label{prop:limits_of_additive_functors}
  Let \(\cat{A},\cat{B}\) be additive categories and \(\fun{F_i}{\cat{A}}{\cat{B}}\) a diagram of additive functors (i.e. a diagram in \(\add{\cat{A}}{\cat{B}}\)) then \(\lim_iF_i\) and \(\colim_iF_i\) (when they exists) are again additive.
\end{proposition}

\begin{proof}
  To prove that \(\colim_iF_i\) is additive it is sufficient to prove that it preserves finite coproducts by Proposition \ref{prop:additive_criteria}. Let \(P_j\) with \(j\in\cat{J}\) be a finite family of objects of \(\cat{A}\); by using Proposition \ref{prop:interchange_of_limits} (keeping in mind that coproducts are particular colimits), Theorem \ref{teo:pointwise_limits} and the fact that each \(F_i\) is additive (thus preserves finite coproducts) we carry out the following computation.
  \begin{align*}
    \left(\colim_iF_i\right)\left(\coprod_{j\in\cat{J}}P_j\right) &= \colim_i\left(F_i\left(\coprod_{j\in\cat{J}}P_j\right)\right)\\
                                                                  &\cong\colim_i\left(\coprod_{j\in\cat{J}}F_i(P_j)\right)\\
                                                                  &\cong\coprod_{j\in\cat{J}}\left(\colim_i\left(F_i(P_j)\right)\right)\\
                                                                  &=\coprod_{j\in\cat{J}}\left(\left(\colim_iF_i\right)(P_j)\right)
  \end{align*}
  This shows that \(\colim_iF_i\) preserves finite coproducts and thus is additive. Proving that \(\lim_iF_i\) is additive is accomplished in the same way using products instead of coproducts.
\end{proof}

\begin{corollary}
  \label{coroll:add_complete}
  Given a small additive category \(\cat{A}\) and a finitely (co)com\-plete additive category \(\cat{B}\) we have that \(\add{\cat{A}}{\cat{B}}\) is finitely (co)complete with (co)limits in it calculated pointwise.
\end{corollary}

\begin{proof}
  Immediate from Theorem \ref{teo:pointwise_limits} and \ref{prop:limits_of_additive_functors}.
\end{proof}

\begin{proposition}
  \label{prop:exactness_and_additiveness_of_filtered_diagrams}
  Let \(\cat{A}\) be a small abelian category and \(\fun{F_i}{\cat{A}}{\catname{Ab}}\) a filtered diagram of left-exact functors; then \(\colim_iF_i\) is again left-exact.
\end{proposition}

\begin{proof}
  First notice that, according to Definition \ref{def:exact_functor}, a left-exact functor must necessarily be additive; this is taken care of by Proposition \ref{prop:limits_of_additive_functors}. To prove that \(\colim_iF_i\) is left-exact we shall prove that it preserves finite limits, as this is an equivalent condition by Proposition \ref{prop:characterization_of_exact_functors}. Let \(\fun{G}{\cat{B}}{\cat{A}}\) be a functor with \(\cat{B}\) finite; by using Proposition \ref{prop:limits_and_filtered_colimits}, Theorem \ref{teo:pointwise_limits} and the fact that each \(F_i\) is left-exact (thus preserves finite limits) we carry out the following computation.
    \begin{align*}
      \left(\colim_iF_i\right)\left(\lim_{B\in\cat{B}}G(B)\right) &=\colim_i\left(F_i\left(\lim_{B\in\cat{B}}G(B)\right)\right)\\
                                                                  &\cong\colim_i\left(\lim_{B\in\cat{B}}F_i(G(B))\right)\\
                                                                  &\cong\lim_{B\in\cat{B}}\left(\colim_iF_i(G(B))\right)\\
                                                                  &=\lim_{B\in\cat{B}}\left(\left(\colim_iF_i\right)(G(B))\right)\\
                                                                  &=\lim_{B\in\cat{B}}\left(\left(\left(\colim_iF_i\right)\circ G\right)(B)\right)
    \end{align*}
    So \(\colim_iF_i\) preserves finite limits and is thus a left-exact functor.
  \end{proof}

  \begin{definition}
    \label{def:category_of_elements}
    Given a functor \(\fun{F}{\cat{C}}{\catname{Set}}\) we construct a new category whose objects are all the pairs \((C,c)\) with \(C\in\cat{C}\) and \(c\in F(C)\) and an arrow \((C,c)\to(C',c')\) is an arrow \(\fun{f}{C}{C'}\) of \(\cat{C}\) such that \(F(f)(c)=c'\).
    The category of elements of \(F\) is denoted with \(\int F\).
    Every category of elements has an associated forgetful functor \(\fun{\Phi_{F}}{\int F}{\cat{C}}\) that sends \((C,c)\) to \(C\).
  \end{definition}

\begin{theorem}[density]
  \label{teo:density}
  If \(\cat{C}\) is a small preadditive category and \(\fun{F}{\cat{C}}{\catname{Ab}}\) an additive (covariant) functor then \(F\) is the colimit in \(\add{\cat{C}}{\catname{Ab}}\) of a diagram of representable (covariant) functors and natural transformations between them.
\end{theorem}

\begin{proof}
  Consider the following composition of functors
  \begin{equation*}
    \begin{tikzcd}[sep=large]
      \int F \ar[r, "\displaystyle\Phi_F"] & \cat{C} \ar[r, "\displaystyle\yo^*"] & \add{\cat{C}}{\catname{Ab}}
    \end{tikzcd}
  \end{equation*}
  where \(\int F\) is the category of elements of \(F\) and \(\Phi_F\) the associated forgetful functor. We will prove that \(F\) is the colimit of \(\yo^*\circ\Phi_F\) i.e. the colimit of a diagram of representable covariant functors.

  Let \((A, a)\) be an element of \(\int F\) so we have that \(a\in F(A)\) and thus \(a\) corresponds to a natural transformation \(\nat{s_{(A,a)}}{\homset{\cat{C}}{A}{-}}{F}\) by the Yoneda Lemma (see \ref{lemma:additive_yoneda_lemma}). Given an arrow \(\fun{f}{(A,a)}{(B,b)}\) in \(\int F\) one has \(F(f)(a) = b\) and thus, by naturality of the Yoneda isomorphisms, \(s_{(A,a)}\circ\homset{\cat{C}}{f}{-} = s_{(B,b)}\). This last fact shows that \((F, (s_{(A,a)})_{(A,a)\in\int F})\) is a cocone on \(\yo^*\circ\Phi_F\).

  Now let \((G, (t_{(A,a)})_{(A,a)\in\int F})\) be a cocone on \(\yo^*\circ\Phi_F\). We wish to find a natural transformation \(\nat{\alpha}{F}{G}\) that uniquely factorizes this cocone. For every element \(x\in F(C)\) we consider the object \((C, x)\in\int F\); the natural transformation \(\nat{t_{(C,x)}}{\homset{\cat{C}}{C}{-}}{G}\) correspond, via Yoneda, to an element of \(G(C)\) so we set \(\alpha_C(x)\) to be that element. In this way an arrow \(\fun{\alpha_C}{F(C)}{G(C)}\) is defined for every \(C\in\cat{C}\); we shall now prove that the family of those arrows is a natural transformation.

  Consider an arrow \(\fun{g}{C}{D}\) in \(\cat{C}\) so \(\fun{g}{(C,x)}{(D, F(g)(x))}\) is an arrow in \(\int F\). Since the \(t_{(A,a)}\)'s constitute a cocone on \(\yo^*\circ\Phi_F\) we have \(t_{(C,x)}\circ\homset{\cat{C}}{g}{-}=t_{(D, F(g)(x))}\) that via Yoneda reveals that \(G(g)(\alpha_C(x)) = \alpha_D(F(g)(x))\) and thus \(\alpha\) is really a natural transformation.

  Finally given \((C,x)\in\int F\) we need to prove that \(\alpha\circ s_{(C,x)} = t_{(C,x)}\). By the naturality of the Yoneda Lemma in the functorial variable we have the following diagram
  \begin{center}
    \begin{tikzcd}[column sep=3cm, row sep=1.5cm]
      \nathom{\homset{\cat{C}}{C}{-}}{F} \ar[d, "\theta_{C, F}"] \ar[r, "\postcomp{\alpha}"] & \nathom{\homset{\cat{C}}{C}{-}}{G} \ar[d, "\theta_{C, G}"]\\
      F(C) \ar[r, "\alpha_C"] & G(C)
    \end{tikzcd}
  \end{center}
  from which the needed relation follows immediately. Lastly if \(\nat{\beta}{F}{G}\) is a natural transformation such that \(\beta\circ s_{(C,x)}=t_{(C,x)}\) again by the naturality of the Yoneda isomorphisms we obtain \(\beta_C(x) = \alpha_C(x)\) and so \(\alpha\) is unique.
\end{proof}

Let \(\cat{A}\) be a small abelian category; an embedding of \(\cat{A}\) in \(\catname{Ab}\) is an additive functor \(\fun{U}{\cat{A}}{\catname{Ab}}\) and thus, by Theorem \ref{teo:density}, \(U\) is the colimit of a diagram of representable functors \(\homset{\cat{A}}{A_i}{-}\) in \(\add{\cat{A}}{\catname{Ab}}\). Representable functors are by definition in a one-to-one correspondence with the objects of \(\cat{A}\) and thus \(U\) can be seen as the colimit of the following composition of functors.
\begin{equation*}\begin{tikzcd}[sep=large]
  \cat{D} \ar[r, "\displaystyle\phi"] & \cat{A} \ar[r, "\displaystyle\yo^*"] & \add{\cat{A}}{\catname{Ab}}
\end{tikzcd}\end{equation*}
Where \(\phi\) is just a functor from an arbitrary category \(\cat{D}\) to \(\cat{A}\) and \(\yo^*\) is the contravariant Yoneda Embedding (see \ref{def:yoneda_embedding}). We shall now construct \(\cat{D}\) and \(\phi\) in such a way that the resulting colimit functor \(U = \colim_{D\in\cat{D}}\homset{\cat{A}}{\phi(D)}{-}\) has the wanted properties; namely faithfullness and exactness.

\begin{notation}
  \label{not:useful_notation}
  From now on, unless specified differently, the symbols \(U, \phi\), \(\cat{D}\) and \(\cat{A}\) are to be interpreted as in the discussion above.
\end{notation}

\begin{proposition}
  \label{prop:strange_relation}
  Let \(\cat{C}\) be a small filtered category and \(\fun{F}{\cat{C}}{\catname{Set}}\) a functor. For \(x\in F(C)\) and \(x'\in F(C')\) where \(C,C'\in\cat{C}\) we set \(x\sim x'\) if and only if there are arrows \(\fun{f}{C}{C''}\) and \(\fun{g}{C'}{C''}\) in \(\cat{C}\) such that \(F(f)(x) = F(g)(x')\); this is an equivalence relation.

  Remembering that the coproduct in \(\catname{Set}\) is the disjoint union we have that the colimit \((L, (s_c)_{C\in\cat{C}})\) of \(F\) is \(\left(\coprod_{C\in\cat{C}}F(C)\right)/\sim\) and the injections \(\fun{s_C}{F(C)}{L}\) are the functions that send each \(x\in F(C)\) to the class \([x]\in L\).
\end{proposition}

\begin{proof}
  See \cite[Proposition 2.13.3]{handbook1}.
\end{proof}
  
\begin{proposition}
  \label{prop:ab_forgetful}
  The forgetful functor \(\fun{\Phi}{\catname{Ab}}{\catname{Set}}\) preserves and reflects limits. It also preserves and reflects filtered colimits.
\end{proposition}

\begin{proof}
  The forgetful functor \(\Phi\) obviously preserves limits, as they are constructed in the same way in the two categories, and reflects isomorphisms, thus by \cite[Proposition 2.9.7]{handbook1} it reflects limits as well. We shall now prove that \(\Phi\) preserves filtered colimits.

  Let \(\fun{F}{\cat{C}}{\catname{Ab}}\) be a functor from a small filtered category \(\cat{C}\) and let \((L, (s_C)_{C\in\cat{C}})\) be the colimit of \(\Phi\circ F\) in \(\catname{Set}\).
  This object can be described as in Proposition \ref{prop:strange_relation}; we are interested in constructing a group structure on it in such a way that the injections \(s_C\) become group homomorphisms.
  Consider two classes \([x],[y]\in L\) with \(x\in F(C)\) and \(y\in F(C')\).
  We can pick, by filteredness of \(\cat{C}\), two arrows \(\fun{f}{C}{C''},\fun{g}{C'}{C''}\) in \(\cat{C}\) and we now have \([x] = [F(f)(x)]\) and \([y] = [F(g)(y)]\).
  Since all the \(F(C)\) for \(C\in\cat{C}\) are groups we can use these two arrows to define an operation on \(L\) as follows.
  \[[x] + [y] = [F(f)(x) + F(g)(y)]\]
  We have to check that this operation does not depend upon the choice of the arrows \(f\) and \(g\) nor upon the choice of the representatives \(x\) and \(y\).

  Let \(\fun{f'}{C}{C'''}\) and \(\fun{g'}{C'}{C'''}\) be arrows of \(\cat{C}\) and \(L\) the vertex of a cocone on Diagram \ref{diagram:rel1} with canonical injections denoted by \(\fun{t_C}{C}{L}\) for \(C\) element of the diagram (a cocone must exist because of Lemma \ref{lemma:filtered_category}).
  \begin{figure}[h]
    \begin{center}
      \begin{tikzcd}[sep=huge]
        C \ar[r, "f"] \ar[dr, "f'" near end] & C''\\
        C' \ar[r, "g'"] \ar[ru, "g" near end] & C'''
      \end{tikzcd}
    \end{center}
    \caption{}
    \label{diagram:rel1}
  \end{figure}

  \noindent We now have
  \begin{align*}
    &F(t_{C''})(F(f)(x)+F(g)(y))\\
    &= F(t_{C''}\circ f)(x) + F(t_{C''}\circ g)(y)\\
    &= F(t_C)(x) + F(t_{C'})(y)\\
    &= F(t_{C'''}\circ f')(x) + F(t_{C'''}\circ g')(y)\\
    &= F(t_{C'''})(F(f')(x) + F(g')(y))
  \end{align*}
  and thus \([F(f)(x)+F(g)(y)]=[F(f')(x)+F(g')(y)]\).
  On the other hand if we pick a second representative \(x'\in F(D)\) of \([x]\) then we have arrows \(e,e'\) as in Diagram \ref{diagram:rel2} such that \(F(e)(x)=F(e')(x')\).
  Let \(f'\) and \(g'\) be the arrows used to define \([x']+[y]\) and \(L\) be a cocone of the diagram with injections labeled as before.
  \begin{figure}[h]
    \begin{center}
      \begin{tikzcd}[sep=huge]
        C \ar[d, "f"'] \ar[dr, "e"' near end] & C' \ar[dl, "g" near end] \ar[dr, "g'"' near end] & D \ar[d, "f'"] \ar[dl, "e'" near end]\\
        C'' & E & D'
      \end{tikzcd}
    \end{center}
    \caption{}
    \label{diagram:rel2}
  \end{figure}

  \noindent One has
  \begin{align*}
    &F(t_{C''})(F(f)(x)+F(g)(y))\\
    &=F(t_{C''}\circ f)(x) + F(t_{C''}\circ g)(y)\\
    &=F(t_C)(x) + f(t_{C'})(y)\\
    &=F(t_E\circ e)(x) + F(t_{D'}\circ g')(y)\\
    &=F(t_E\circ e')(x') + F(t_{D'}\circ g')(y)\\
    &=F(t_D)(x') + F(t_{D'}\circ g')(y)\\
    &=F(t_{D'}\circ f')(x') + F(t_{D'}\circ g')(y)\\
    &=F(t_{D'})(F(f')(x') + F(g')(y))
  \end{align*}
  so \([x]+[y] = [x']+[y]\).
  Similarly the choice of a representative for \([y]\) does not matter either.
  
  With this \(L\) becomes a group and the injections \(s_C\) become group homomorphisms. Indeed we have, for \(x_1,x_2\in F(C)\) (choosing \(f = g = 1_C\)):
  \begin{align*}
    s_C(x_1 + x_2) &= [x_1 + x_2]\\
    &= [F(1_C)(x_1 + x_2)]\\
    &= [F(1_C)(x_1) + F(1_C)(x_2)]\\
    &= [x_1] + [x_2]\\
    &= s_C(x_1) + s_C(x_2).
  \end{align*}
  Moreover the \(s_C\)'s in \(\catname{Ab}\) constitute a cocone on \(F\) because the underlying maps in \(\catname{Set}\) do. Now, to prove that \(L\in\catname{Ab}\) is a colimit, consider a cocone \((M, (t_C)_{C\in\cat{C}})\) on \(F\) and the unique factorization \(\fun{t}{L}{M}\) that exists in \(\catname{Set}\). The following computation shows that \(t\) is a group homomorphism and thus that \(L\) is the colimit of \(F\).
  \begin{align*}
    t([x] + [y]) &= t([F(f)(x) + F(g)(y)])\\
    &= t(s_{C''}(F(f)(x) + F(g)(y)))\\
    &= t_{C''}(F(f)(x) + F(g)(y))\\
    &= t_{C''}(F(f)(x)) + t_{C''}(F(g)(y))\\
    &= t(s_{C''}(F(f)(x))) + f(s_{C''}(F(g)(y)))\\
    &= t([F(f)(x)]) + t([F(g)(y)])\\
    &= t([x]) + t([y])
  \end{align*}
  By how \(\Phi\) operates it is now clear that the colimit \(L\) is preserved; thus \(\Phi\) preserves filtered colimits.
  Finally one can easily adapt the proof of \cite[Proposition 2.9.7]{handbook1} to the case of filtered colimits and so we obtain that \(\Phi\) reflects filtered colimits as well.
\end{proof}

\begin{proposition}
  \label{prop:mod_ab_limits_colimits}
  The forgetful functor \(\fun{\Phi}{\catname{Mod}_R}{\catname{Ab}}\) preserves and reflects limits and colimits.
\end{proposition}

\begin{proposition}
  \label{prop:hom_preserves_limits}
  The hom-functor \(\homset{\cat{C}}{C}{-}\) from a category \(\cat{C}\) to \(\catname{Set}\) preserves limits.
\end{proposition}

\begin{proof}
  Consider a functor \(\fun{F}{\cat{D}}{\cat{C}}\) that has limit \((L, (p_D)_{D\in\cat{D}})\). Clearly \((\homset{\cat{C}}{C}{L}, \homset{\cat{C}}{C}{p_D}_{D\in\cat{D}})\) is a cone over \(\homset{\cat{C}}{C}{F(-)}\) by functoriality of the hom-functor; to prove its universality consider another cone \((M, (q_D)_{D\in\cat{D}})\) in \(\catname{Set}\) over the functor \(\homset{\cat{C}}{C}{F(-)}\).

  For every \(m\in M\) the family \((\fun{q_D(m)}{C}{F(D)})_{D\in\cat{D}}\) is a cone on \(F\) and thus there is a unique arrow \(\fun{q(m)}{C}{L}\) of \(\cat{C}\) such that \(p_D\circ q(m) = q_D(m)\).
  This association of an element \(m\in M\) with a factorization \(q(m)\) defines a map \(\fun{q}{M}{\homset{\cat{C}}{C}{L}}\) such that \(\homset{\cat{C}}{C}{p_D}\circ q = q_D\).
  Moreover this \(q\) is unique since all the \(q(m)\)'s are.
  We have proved that \(\homset{\cat{C}}{C}{L}\) is the limit of \(\homset{\cat{C}}{C}{F(-)}\) and so that \(\homset{\cat{C}}{C}{-}\) preserves limits.
\end{proof}

\begin{lemma}
  \label{lemma:left_exactness_of_U}
  If \(\cat{D}\) is cofiltered then \(U\) is left-exact.
\end{lemma}

\begin{proof}
  Suppose \(\fun{\homset{\cat{A}}{A}{-}}{\cat{A}}{\catname{Ab}}\) is left-exact for any \(A\in\cat{A}\) then, since \(\cat{D}\) is cofiltered and \(\yo^*\) is contravariant, the image of \(\cat{D}\) through \(\yo^*\circ\phi\) is a filtered diagram of left-exact additive functors in \(\add{\cat{A}}{\catname{Ab}}\); thus by Proposition \ref{prop:exactness_and_additiveness_of_filtered_diagrams} we obtain that \(U = \colim_{D\in\cat{D}}\homset{\cat{A}}{\phi(D)}{-}\) is a left-exact additive functor.

  It remains to prove that \(\homset{\cat{A}}{A}{-}\) is indeed left-exact. Consider a functor \(\fun{F}{\cat{C}}{\cat{A}}\) and let \(L\in\cat{A}\) be its limit. From Proposition \ref{proposition:hom_preserves_limits} \(\homset{\cat{A}}{A}{L}\in\catname{Set}\) is the limit of \(\fun{\homset{\cat{A}}{A}{-}\circ F}{\cat{C}}{\catname{Set}}\). This is however equivalent to saying that \(\homset{\cat{A}}{A}{L}\) is the limit of the composite functor
  \begin{equation*}
    \begin{tikzcd}[sep=huge]
      \cat{C} \ar[r, "\displaystyle F"] & \cat{A} \ar[r, "\displaystyle\homset{\cat{A}}{A}{-}"] & \catname{Ab} \ar[r, "\displaystyle\Phi"] & \catname{Set}
    \end{tikzcd}.
  \end{equation*}
  But since \(\Phi\) reflects limits \(\homset{\cat{A}}{A}{L}\) as a group (i.e. an element of \(\catname{Ab}\)) is the limit of \(\fun{\homset{\cat{A}}{A}{-}\circ F}{\cat{C}}{\catname{Ab}}\) so the hom functor \(\homset{\cat{A}}{A}{-}\) from \(\cat{A}\) to \(\catname{Ab}\) preserves limits. By Proposition \ref{prop:characterization_of_exact_functors} we obtain that representables are left-exact and this concludes the proof.
\end{proof}

\begin{lemma}
  \label{lemma:U_preserves_epis}
  \(U\) preserves epimorphisms as long as \(\cat{D}\) is cofiltered and every epimorphism of the form \(\fun{f}{A}{\phi(D)}\) in \(\cat{A}\) can be written as \(f = \phi(d)\) for a suitable arrow \(\fun{d}{D'}{D}\) in \(\cat{D}\).
\end{lemma}

\begin{proof}
  An arrow in \(\catname{Ab}\) is an epimorphism if and only if it is surjective, so if \(\fun{g}{B}{C}\) is an arrow in \(\cat{A}\) its image \(U(g)\) is epi if and only if it is surjective. Filtered colimits are computed in \(\catname{Ab}\) exactly as in \(\catname{Set}\) by Proposition \ref{prop:ab_forgetful} thus we can use Proposition \ref{prop:strange_relation}.

  Let \(x\in U(C) = \colim_{D\in\cat{D}}\homset{\cat{A}}{\phi(D)}{C}\); but from the discussion above we have that \(\colim_{D\in\cat{D}}\homset{\cat{A}}{\phi(D)}{C} = \left(\coprod_{D\in\cat{D}}\homset{\cat{A}}{\phi(D)}{C}\right)/\sim\) so \(x\) is the class of some arrow \(\fun{\gamma}{\phi(D)}{C}\) for some \(D\in\cat{D}\). To show that \(U(g)\) is a surjection we must find some \(y\in U(B) = \colim_{D\in\cat{D}}\homset{\cat{A}}{\phi(D)}{B}\) such that \(U(g)(y) = x\). Repeating the previous argument such a \(y\) must be the class of an arrow \(\fun{\beta}{\phi(D')}{B}\) and now the condition \(U(g)(y) = x\) becomes \(U(g)([\beta]) = [g\circ\beta] = [\gamma]\); finally this amounts to finding arrows \(\fun{d}{D''}{D}\) and \(\fun{d'}{D''}{D'}\) in \(\cat{D}\) such that
  \begin{equation*}
    \homset{\cat{A}}{\phi(d')}{C}(g\circ\beta) = \homset{\cat{A}}{\phi(d)}{C}(\gamma)
  \end{equation*}
  that is in turn equivalent to showing that \(g\circ\beta\circ\phi(d') = \gamma\circ\phi(d)\).

  Now consider the pullback of \(g\) along \(\gamma\) of Diagram \ref{diagram:pullback_eq}; we obtain immediately that \(f\) is epi by Proposition \ref{prop:pullback_of_epi}. By assumption we can write \(f\) as \(f = \phi({d})\) for an appropriate \(\fun{d}{D''}{D}\) and thus
  \begin{align*}
    \gamma\circ\phi(d) &= \gamma\circ f = g\circ\alpha = g\circ\alpha\circ\phi(1_{D''}).
  \end{align*}
   This proves that \(U(g)\) is surjective and so the proof is complete.

  \begin{figure}[h]
    \begin{center}
  \begin{tikzcd}[sep=large]
    A \ar[r, two heads, "f"] \ar[d, "\alpha"] & \phi(D) \ar[d, "\gamma"]\\
    B \ar[r, two heads, "g"] & C
  \end{tikzcd}
\end{center}

    \caption{}
    \label{diagram:pullback_eq}
  \end{figure}
\end{proof}

\begin{lemma}
  \label{lemma:U_faithful}
  If \(\cat{D}\) is cofiltered and
  \begin{enumerate}[label=(\arabic*)]
  \item every \(A\in\cat{A}\) can be written as \(A = \phi(D)\) for some \(D\in\cat{D}\) i.e. \(\phi\) is surjective on objects,
  \item for every morphism \(d\in\cat{D}\) the image \(\phi(d)\in\cat{A}\) is always epi,
  \end{enumerate}
  then \(U\) is faithful.
\end{lemma}

\begin{proof}
  Let \(\fun{f,g}{A}{B}\) be morphisms in \(\cat{A}\) such that \(U(f) = U(g)\). By Proposition \ref{prop:strange_relation} we have that
  \begin{equation*}
    U(A) = \colim_{D\in\cat{D}}\homset{\cat{A}}{\phi(D)}{A}=\left(\coprod_{D\in\cat{D}}\homset{\cat{A}}{\phi(D)}{A}\right)/\sim
  \end{equation*}
  and thus an element \(a\in U(A)\) is the class of an arrow \(\fun{\alpha}{\phi(D)}{A}\) for some \(D\in\cat{D}\). By assumption \(U(f) = U(g)\) and thus \(U(f)(a) = U(g)(a)\); this last relation can be re-read in terms of classes of arrows as \([f\circ\alpha] = [g\circ\alpha]\) that in turn ensures the existence of arrows \(\fun{x,x'}{X}{D}\) in \(\cat{D}\) such that
  \begin{equation*}
    \homset{\cat{A}}{\phi(x)}{B}(f\circ\alpha) = \homset{\cat{A}}{\phi(x')}{B}(g\circ\alpha).
  \end{equation*}
  By cofilteredness of \(\cat{D}\) there is an arrow \(\fun{d'}{D'}{X}\) such that \(x\circ d' = x'\circ d'\); thus we can define \(d = x\circ d' = x'\circ d'\). This is an arrow from \(D'\) to \(D\) such that
  \begin{equation*}
    \homset{\cat{A}}{\phi(d)}{B}(f\circ\alpha) = \homset{\cat{A}}{\phi(d)}{B}(g\circ\alpha)
  \end{equation*}
  that is \(f\circ\alpha\circ\phi(d) = g\circ\alpha\circ\phi(d)\).

  Now by hypothesis (1) we have that \(A = \phi(\overline{D})\) for some \(\overline{D}\in\cat{D}\). Moreover \(1_A\in\homset{\cat{A}}{\phi(\overline{D})}{A}\) so it represents a class \(\overline{a}\in U(A)\) and thus by the previous discussion there is an arrow \(d\in\cat{D}\) such that
    \begin{equation*}
      f\circ 1_A\circ\phi(d) = g\circ 1_A\circ\phi(d)
    \end{equation*}
    from which we obtain \(f\circ\phi(d) = g\circ\phi(d)\). By hypothesis (2) \(\phi(d)\) is epi and thus \(f = g\); this proves that \(U\) is faithful.
\end{proof}

\begin{theorem}
  \label{teo:faithful_embedding}
  Every small abelian category admits a faithful and exact embedding in \(\catname{Ab}\).
\end{theorem}

\begin{proof}
  We shall construct a category \(\cat{D}\) and a functor \(\phi\) such that all the hypothesis of Lemmas \ref{lemma:left_exactness_of_U}, \ref{lemma:U_preserves_epis} and \ref{lemma:U_faithful} are satisfied; this way \(U\) will be a faithful functor from \(\cat{A}\) to \(\catname{Ab}\) that is left-exact and preserves epimorphisms, thus is exact by Proposition \ref{prop:left_condition}.

  Let \(\cat{D}_0\subseteq\cat{D}_1\subseteq\ldots\) be a sequence of posets (regarded as categories in the usual way) and \(\phi_0,\phi_1,\ldots\) a sequence of corresponding functors of type \(\cat{D}_n\to\cat{A}\) such that
  \begin{enumerate}[label=(\arabic*)]
  \item \(\cat{D}_n\) is a meet-semilattice,
  \item if \(n\leq m\) then \(\phi_m\) and \(\phi_n\) coincide on \(\cat{D}_n\) (in symbols \(\phi_m\upharpoonright\cat{D}_n=\phi_n\)),
  \item for every arrow \(d\in\cat{D}_n\) the image \(\phi_n(d)\in\cat{A}\) is always epi,
  \item for every \(A\in\cat{A}\) there is a \(D\in\cat{D}_1\) such that \(\phi_1(D) = A\),
  \item for every \(A\in\cat{A}\), \(D\in\cat{D}_n\) and \(f\colon A\twoheadrightarrow\phi_n(D)\) in \(\cat{A}\) there is an arrow \(d\in\cat{D}_{n+1}\) such that \(\phi_{n+1}(d) = f\).
  \end{enumerate}
  Once such a sequence is found we can set \(\cat{D} = \bigcup_n\cat{D}_n\) and \(\phi\) the natural extension of all the \(\phi_n\). For such a pair we have that
  \begin{itemize}
  \item \(\cat{D}\) is a meet-semilattice thus it is cofiltered and the hypothesis of Lemma \ref{lemma:left_exactness_of_U} hold,
  \item condition (5) ensures that the hypothesis of Lemma \ref{lemma:U_preserves_epis} hold,
  \item conditions (3), (4) ensure that the hypothesis of Lemma \ref{lemma:U_faithful} hold.
  \end{itemize}
  This is exactly what we want: Lemma \ref{lemma:left_exactness_of_U} and Lemma \ref{lemma:U_preserves_epis} ensure that \(U\) is left-exact and preserves epimorphisms thus, by Proposition \ref{prop:left_condition}, it is exact; faithfulness of \(U\) is taken care of by Lemma \ref{lemma:U_faithful}. All that's left is to construct such a sequence of posets and functors.

  We set \(\cat{D}_0\) to be the discrete category \(\{*\}\) and \(\phi_0(*)\) to be the zero object \({\bf 0}\) of \(\cat{A}\); \(\phi_0(1_*) = 1_{\bf 0}\) is clearly epi and, being a singleton, \(\cat{D}_0\) is clearly a meet-semilattice.

  The poset \(\cat{D}_{n+1}\) is constructed by induction. Suppose \(\cat{D}_0\subseteq\ldots\subseteq\cat{D}_n\) have been defined and are such that conditions (1), (2), (3) and (5) hold for all of them. Consider all the pairs \((D, f)\) where \(D\in\cat{D}_n\) and \(f\colon A\twoheadrightarrow\phi_n(D)\) is an epimorphism in \(\cat{A}\) for some \(A\in\cat{A}\); we index those pairs by successive successor ordinals\footnote{To fully justify this step one can first observe that the collection of pairs \((D, f)\) described is a (small) set because \(\cat{A}\) is a small category and then apply the Well-Ordering Theorem to impose a well-order on it. In this way we obtain the needed indexation. Also notice that we \emph{do not} use limit ordinals to index any pair.} starting from 1 (0 will be handled differently). We then construct, by transfinite induction, another sequence of posets \(\cat{D}_n^0\subseteq\ldots\subseteq\cat{D}_n^\alpha\subseteq\ldots\) up to the supremum of the ordinals used as indexes and a corresponding sequence of functors \(\fun{\phi_n^\alpha}{\cat{D}_n^\alpha}{\cat{A}}\) such that
  \begin{enumerate}[label=(\alph*)]
  \item \(\cat{D}_n^\alpha\) is a meet-semilattice,
  \item if \(\beta\leq\alpha\) then \(\phi_n^\alpha\upharpoonright\cat{D}_n^\beta = \phi_n^\beta\),
  \item for every arrow \(d\in\cat{D}_n^\alpha\) the image \(\phi_n^\alpha(d)\in\cat{A}\) is epi,
  \item if \(\alpha\) indexes the pair \((D,\epi{f}{A}{\phi_n(D)})\) then there is an arrow \(d\in\cat{D}_n^\alpha\) such that \(\phi_n^\alpha(d) = f\).
  \end{enumerate}
  As before we shall define \(\cat{D}_{n+1}=\bigcup_\alpha\cat{D}_n^\alpha\) and \(\phi_{n+1}\) to be the extension of all the \(\phi_n^\alpha\). This construction ensures that \(\cat{D}_{n+1}\) satisfies conditions (1), (2), (3) and (5). It just remains to prove that we can indeed find such a sequence.

  First we set \((\cat{D}_n^0, \phi_n^0) = (\cat{D}_n, \phi_n)\); this pair satisfies (a) and (c) because \(\cat{D}_n\) satisfies (1) and (3), while (b) and (d) are trivially satisfied.

  If \(\beta\) is a limit ordinal we set \(\cat{D}_n^\beta=\bigcup_{\alpha<\beta}\cat{D}_n^\alpha\) and extend the \(\phi_n^\alpha\) to \(\phi_n^\beta\) as before. Clearly \((\cat{D}_n^\beta,\phi_n^\beta)\) satisfies conditions (a), (b) and (c) because the same conditions are satisfied for all \((\cat{D}_n^\alpha, \phi_n^\alpha)\) with \(\alpha<\beta\) while (d) is trivially satisfied.

  Finally if \((\cat{D}_n^0,\phi_n^0),\ldots,(\cat{D}_n^\alpha,\phi_n^\alpha)\) have been defined, satisfy (a), (b), (c), (d), and \(\alpha+1\) indexes the pair \((D, f)\) we consider the initial segment \(\downarrow D = \{D'\in\cat{D}_n^\alpha\colon D'\leq D\}\) and perform the disjoint union \(\cat{D}_n^\alpha\amalg\downarrow D\); we will write \(D'^*\) for the copy of \(D'\in\cat{D}_n^\alpha\) that lies in \(\downarrow D\), if any. On the two components of the disjoint union we leave the original ordering, but we impose \(D'^*\leq D'\) for all \(D'\in{}\downarrow D\); this is enough to generate a poset structure on \(\cat{D}_n^\alpha\amalg\downarrow D\). We set \(\cat{D}_n^{\alpha+1} = \cat{D}_n^\alpha\amalg\downarrow D\).

  \begin{figure*}
    \begin{center}
      \begin{tikzpicture}[scale=2]
        \filldraw (0, 1) -- (2, 1) circle (1pt) node[above] {\(D'^*\)} -- (3, 1) circle (1pt) node[above] {\(D^*\)};
        \filldraw (0, 0) -- (2, 0) circle (1pt) node[below] {\(D'\)}   -- (3, 0) circle (1pt) node[below] {\(D\)} -- (4, 0);
        \draw (5, 0) node {\(\cat{D}_n^\alpha\)};
        \draw (5, 1) node {\(\downarrow D\)};
        \draw[->] (2, .8) -- (2, .2);
        \draw[->] (3, .8) -- (3, .2);
      \end{tikzpicture}
    \end{center}
    \captionsetup{labelformat=empty}
    \caption{\small A depiction of the poset \(\cat{D}_n^\alpha\amalg\downarrow D\). Keep in mind that the posets are drawn as linear orders so that the picture does not become too complicated but, generally speaking, \emph{they are not}.}
  \end{figure*}

  It is immediate that \(\cat{D}_n^{\alpha +1}\) is a meet-semilattice: both \(\cat{D}_n^\alpha\) and \(\downarrow D\) are meet-semilattices thus the meet of two elements of \(\cat{D}_n^{\alpha+1}\) that lie both in one component is simply the meet of the two elements in that component; the mixed case is handled by computing the meet of the two in \(\cat{D}_n^\alpha\) and then taking the copy of the result in \(\downarrow D\). This shows that (a) holds for \(\cat{D}_n^{\alpha+1}\).

  We need to define a functor \(\fun{\phi_n^{\alpha+1}}{\cat{D}_n^{\alpha+1}}{\cat{A}}\); in order for condition (b) to hold we are forced to have \(\phi_n^{\alpha+1}\) mimic \(\phi_n^\alpha\)  on the component \(\cat{D}_n^\alpha\) of \(\cat{D}_n^{\alpha+1}\). We can then set \(\phi_n^{\alpha+1}(D^*) = A\) and \(\phi_n^{\alpha+1}(D^*\leq D) = f\); this takes care of condition (d). Given \(D'\in{}\downarrow D\) we set \(\phi_n^{\alpha+1}(D'^*)\) to be the object obtained via the following pullback (Diagram \ref{diagram:foo}).
  \begin{figure}[h]
    \begin{center}
      \begin{tikzcd}[sep=huge]
        \phi_n^{\alpha+1}(D'^*) \ar[r, two heads, "u"] \ar[d, two heads, "v"] & A \ar[d, two heads, "f"]\\
        \phi_n^\alpha(D') \ar[r, two heads, "\phi_n^\alpha(D'\leq D)"] & \phi_n^\alpha(D) = \phi_n(D)
      \end{tikzcd}
    \end{center}
    \caption{}
    \label{diagram:foo}
  \end{figure}\\
  \noindent Moreover observing that \(A = \phi_n^{\alpha+1}(D^*)\) and that \(\phi_n^\alpha(D') = \phi_n^{\alpha+1}(D')\) we set \(\phi_n^{\alpha+1}(D'^*\leq D^*) = u\) and \(\phi_n^{\alpha+1}(D'^*\leq D') = v\).

  Now if \(D'\leq D''\) and \(D',D''\in{}\downarrow D\) consider Diagram \ref{diagram:bar} where the left and right squares are pullbacks by construction.
  \begin{figure}[h]
    \begin{center}
      \begin{tikzcd}[sep=huge]
        P \ar[d, two heads, "x"] \ar[r, two heads, "y"] & \phi_n^{\alpha+1}(D''^*) \ar[d, two heads, ""] \ar[r, two heads, ""] & A \ar[d, two heads, "f"]\\
        \phi_n^\alpha(D') \ar[r, two heads, "\phi_n^\alpha(D'\leq D'')"] & \phi_n^\alpha(D'') \ar[r, two heads, "\phi_n^\alpha(D''\leq D)"] & \phi_n(D) = \phi_n^\alpha(D)
      \end{tikzcd}
    \end{center}
    \caption{}
    \label{diagram:bar}
  \end{figure}
  By the pullback lemma the outer square is also a pullback. By functoriality of \(\phi_n^\alpha\) the lower composite is just \(\phi_n^\alpha(D'\leq D)\) so, by the above discussion, we have \(P = \phi_n^{\alpha + 1}(D'^*)\) and \(x = \phi_n^{\alpha+1}(D'^*\leq D')\). We thus set \(\phi_n^{\alpha+1}(D'^*\leq D''^*) = y\).
  
  The last case is that of arrows of the form \(D'^*\leq D''\) for \(D'\in{}\downarrow D\) and \(D'\leq D''\). But we can simply set
  \begin{equation*}
    \phi_n^{\alpha+1}(D'^*\leq D'') = \phi_n^{\alpha+1}(D'\leq D'')\circ\phi_n^{\alpha+1}(D'^*\leq D')
  \end{equation*}
  since \(\phi_n^{\alpha+1}(D'^*\leq D')\) has already been defined and \(\phi_n^{\alpha+1}=\phi_n^\alpha\) over \(\cat{D}_n^\alpha\).

  The construction of \(\phi_n^{\alpha + 1}\) is finally complete and moreover \(\phi_n^{\alpha+1}(d)\) is epi for every arrow \(d\in\cat{D}_n^{\alpha+1}\) because \(\phi_n^{\alpha+1}\) extends \(\phi_n^\alpha\) (for which condition (c) holds at this stage of the induction) and every arrow in the previous diagrams is an epimorphism by Proposition \ref{prop:pullback_of_epi}. The pair \((\cat{D}_n^{\alpha+1},\phi_n^{\alpha+1})\) thus satisfies conditions (a), (b), (c) and (d).

  The last thing we need to prove is that condition (4) holds. The construction of the poset \(\cat{D}_1\) is indexed by all the pairs \((*, f)\) with \(\epi{f}{A}{\phi_0(*)}\) epimorphism; but by construction \(\phi_0(*) = {\bf 0}\) and thus there is only one such \(f\) for every \(A\in\cat{A}\) because \({\bf 0}\) is both initial and final. By condition (d) for every \(A\in\cat{A}\) the poset \(\cat{D}_1\) will contain an element \(D\) such that \(\phi_1(D) = A\) and this is exactly (4).
\end{proof}

We shall now prove some additional properties of the pair \((\cat{D},\phi)\) just constructed; such properties will then be used to turn the embedding \(U\) in a fully faithful exact embedding in \(\catname{Mod}_R\), for a suitable ring \(R\).

\begin{lemma}
  \label{lemma:finite_segment}
  Given \(D_1,D_2\in\cat{D}\) the segment \([D_1, D_2] = \{D\in\cat{D}\colon D_1\leq D\leq D_2\}\) is always finite.
\end{lemma}

\begin{proof}
  Recall that \(\cat{D}\) is the union of all the \(\cat{D}_0\subseteq\cat{D}_1\subseteq\ldots\) that are constructed by induction. In turn each \(\cat{D}_{n+1}\) is also the union of a sequence \(\cat{D}_n^0\subseteq\cat{D}_n^1\subseteq\ldots\) that is constructed by (transfinite) induction. This means that \(D_1\) and \(D_2\) must be introduced at some point of the construction of those sequences; we will prove that
  \begin{enumerate}[label=(\arabic*)]
  \item at that level \([D_1, D_2]\) is finite,
  \item the segment \([D_1, D_2]\) remains unchanged at all further levels.
  \end{enumerate}
  Condition (1) holds trivially for \(\cat{D}_0=\{*\}\). Now let's assume that (1) holds for \(\cat{D}_n^\alpha\) and consider \(D_1,D_2\in\cat{D}_n^{\alpha}\); by construction in \(\cat{D}_n^{\alpha+1}\) one never has \(D'\leq D^*\) for a new \(D^*\) and an old \(D'\) (the poset ``grows downwards''), so \([D_1, D_2]\) remains unchanged in the step from \(\alpha\) to \(\alpha + 1\).

  If instead \(D_1, D_2\not\in\cat{D}_n^\alpha\) then they are part of the second component of \(\cat{D}_n^{\alpha+1}\) thus we have \(D_1 = D^*\) and \(D_2 = D'^*\) for appropriate \(D, D'\in\cat{D}_n^\alpha\). The segment \([D, D']\) is finite by inductive hypothesis and thus is \([D^*, D'^*] = [D_1, D_2]\) because they are isomorphic.

  The case of \(D_1\in\cat{D}_n^\alpha\) and \(D_2\not\in\cat{D}_n^\alpha\) results in an empty segment (as before no new element can be bigger than an old one); so the final case to analyze is that of \(D_1\not\in\cat{D}_n^\alpha\) and \(D_2\in\cat{D}_n^\alpha\). This means that there is a \(D\in\cat{D}_n^\alpha\) such that \(D_1 = D^*\) so \([D_1, D_2] = [D^*, D_2^*]\cup[D, D_2]\) and both are finite by induction.

  We have that (1) and (2) remain valid at each successor step but they obviously hold at each limit step too and thus the proof is complete.
\end{proof}

\begin{lemma}
  \label{lemma:limit_factorization_is_epi}
  Given \(D_1\leq D_2\) in \(\cat{D}\) the factorization arrow
  \begin{equation*}
    \begin{tikzcd}[sep=huge]
      \phi(D_1) \ar[r, ""] & \displaystyle\lim_{D_1<D\leq D_2}\phi(D)
    \end{tikzcd}
  \end{equation*}
  is always epi.
\end{lemma}

\begin{proof}
  First we should clarify that such factorization actually exists. The segment \([D_1, D_2]\) is a poset and \(D_1\) is its initial object so we have a arrow \(\phi(D_1\leq D)\) for every \(D\in[D_1, D_2]\). Moreover \([D_1, D_2]\) is finite by Lemma \ref{lemma:finite_segment} and \(\cat{A}\) is finitely complete because it is abelian (Proposition \ref{prop:completeness}) so \(\lim_{D_1<D\leq D_2}\phi(D)\) exists and \((\phi(D_1), \phi(D_1\leq D)_{D_1<D\leq D_2})\) is a cone on \(\phi\) (restricted to the \(D_1<D\leq D_2\)). By the definition of limit we obtain a unique factorization \(\phi(D_1)\to\lim_{D_1<D\leq D_2}\phi(D)\) as in the Lemma's statement.

  We proceed by induction on the level at which \(D_1\) is introduced. We already observed (in the proof of Lemma \ref{lemma:finite_segment}) that, when moving from a level of the construction to the next, new objects are never bigger than the ones we already had and so when \(D_1\) is introduced \(D_2\) and all the \(D\) such that \(D_1<D\leq D_2\) exist as well.

  If \(D_1\in\cat{D}_0\) then, since \(\cat{D}_0 = \{*\}\), we have \(D_2 = D_1 = *\) and thus the limit is the limit of an empty diagram i.e. the final object {\bf 0}. The factorization in the Lemma's statement is thus the unique arrow from \(\phi(D_1) = {\bf 0}\) to \({\bf 0}\) that is the identity on the zero object of \(\cat{A}\); clearly an epimorphism.

  In our construction an object is never introduced at a limit step so we only need to consider the case in which \(D_1\) is added when \(\cat{D}_n^{\alpha + 1}\) is constructed. There will thus be a \(D_0\in\cat{D}_n^\alpha\) such that \(D_1 = D_0^*\) while for the object \(D_2\) we have two cases: either \(D_2\in\cat{D}_n^\alpha\) (if \(D_2\) already exists when \(D_1\) is added) or \(D_2\not\in\cat{D}_n^\alpha\) (if \(D_2\) is ``new'' as well).

  If \(D_1 = D_2\) (so \(D_2\not\in\cat{D}_n^\alpha\)) then the limit is again the limit of an empty diagram; thus the zero object \({\bf 0}\). The factorization in the Lemma's statement this time is a zero arrow to \({\bf 0}\) that is epi because \({\bf 0}\) is terminal.

  If \(D_1 \not= D_2\) and \(D_2\not\in\cat{D}_n^\alpha\) then \(D_2 = D_{00}^*\) for an appropriate \(D_{00}\in\cat{D}_n^\alpha\). Consider Diagram \ref{diagram:three_squares} for all \(D_0<D\leq D_{00}\) where we use the letter \(p\), with the appropriate subscript, for the projections of the two limits involved.
  \begin{figure}[h]
    \begin{center}
      \begin{tikzcd}[sep=huge]
        \displaystyle\phi(D_0^*) \ar[dr, phantom, "(1)" description] \ar[d, ""] \ar[r, two heads, "f^*"] & \displaystyle\lim_{D_0<D\leq D_{00}}\phi(D^*) \ar[d, ""] \ar[r, "p_{D^*}"] \ar[dr, phantom, "(2)" description] &
        \displaystyle\phi(D^*) \ar[r, "\phi(D^*\leq D_{00}^*)"] \ar[d, ""]\ar[dr, phantom, "(3)" description] & \displaystyle\phi(D_{00}^*) \ar[d, ""]\\
        \displaystyle\phi(D_0) \ar[r, two heads, "f"] & \displaystyle\lim_{D_0<D\leq D_{00}}\phi(D) \ar[r, "p_D"] & \phi(D) \ar[r, "\phi(D\leq D_{00})"] & \phi(D_{00})
      \end{tikzcd}
    \end{center}
    \caption{}
    \label{diagram:three_squares}
  \end{figure}
  By construction of \((\cat{D},\phi)\) the squares (3) are pullbacks. By definition of limit \(\phi(D\leq D_{00})\circ p_D = p_{D_{00}}\) and \(\phi(D^*\leq D_{00}^*)\circ p_{D^*} = p_{D_{00}^*}\) so the perimeter of the rectangle composed of the squares (2) and (3) is the same for all \(D_0<D\leq D_{00}\). This rectangle is the limit of the pullbacks (3) so, by interchange of limits (Proposition \ref{prop:interchange_of_limits}), it is itself a pullback. Again by construction of \((\cat{D}, \phi)\) the outer rectangle is a pullback and thus the square (1) is a pullback by the Pullback Lemma. By inductive hypothesis \(f\), the factorization, is epi and so, by Proposition \ref{prop:pullback_of_epi}, \(f^*\) is epi as well.

  The last case is that of \(D_1 = D_0^*\) as before and \(D_2\in\cat{D}_n^\alpha\); suppose that \(\alpha + 1\) indexes the pair \((\overline{D}, \overline{f})\). If \(\overline{D} = D_0\) then \(\{D\colon D_1<D\leq D_2\} = [\overline{D},D_2]\) so it has an initial object \(\overline{D}\); the inclusion of \(\{{\bf 0}\}\) (the subcategory of only the initial object \({\bf 0}\)) in \([\overline{D}, D_2]\) is a final functor by \cite[Proposition 2.11.4]{handbook1} and thus the limit of \(\phi\) is just \(\phi(\overline{D})\). In this case the factorization of the Lemma's statement is just \(\overline{f}\) that is epi by the details of the construction.

  On the other hand it is also possible that \(D_0 < \overline{D}\) (equivalently \(D_1 = D_0^* < \overline{D}^*\)) and \(D_2\in\cat{D}_n^\alpha\). Let \(D_3 = D_2\land\overline{D}\) be a third element of \(\cat{D}_n^\alpha\); if \(D_1<D\leq D_2\) then since \(D_1\leq \overline{D}^*<\overline{D}\) we obtain \(D_1\leq\overline{D}\land D\leq\overline{D}\land D_2 = D_3\) that in turn gives us that \(\{D\colon D_1<D\leq D_3\}\) is an initial segment of \(\{D\colon D_1<D\leq D_2\}\). The inclusion functor of the first in the second is final because it satisfies the sufficient condition given by \cite[Proposition 2.11.2]{handbook1} so it is sufficient to compute the limit of \(\phi\) over this first initial segment; we wish to prove that the limit is \(\phi(D_1)\)\footnote{One may wonder why such a construction is employed. The advantage that it offers is that in \(\{D\colon D_1<D\leq D_3\}\) every element in the component \(\cat{D}_n^\alpha\) has a double in the other component \(\downarrow\overline{D}\); this is not guaranteed to happen in \(\{D\colon D_1<D\leq D_2\}\).}.
  More precisely we want to prove that \(\left(\phi(D_1), \phi(D_1\leq D)_{D_1< D\leq D_3}\right)\) is a universal cone on \(\phi\) so let \(\left(A, (f_D\colon A\to D)_{D_1<D\leq D_3}\right)\) be another cone on \(\phi\) and consider Diagram \ref{diagram:f_cone} where the right bottom square is a pullback by definition of \((\cat{D}, \phi)\).
  \begin{figure}[h]
    \begin{center}
      \begin{tikzcd}[sep=huge]
        A \ar[rd, dashed, "f" description] \ar[ddr, bend right, "f_{D_0}"] \ar[rrd, bend left, "f_{D_3^*}"] & & \\
        & \phi(D_0^*) \ar[d, ""] \ar[r, ""] & \phi(D_3^*) \ar[d, ""]\\
        & \phi(D_0) \ar[r, ""] & \phi(D_3)
      \end{tikzcd}
    \end{center}
    \caption{}
    \label{diagram:f_cone}
  \end{figure}

  The outer square commutes by the definition of cone:
  \begin{align*}
    \phi(D_3^*\leq D_3)\circ f_{D_3^*} = f_{D_3}\\
    \phi(D_0\leq D_3)\circ f_{D_0} = f_{D_3}
  \end{align*}
  and so we obtain \(\fun{f}{A}{\phi(D_0^*)}\) as factorization through the pullback. We will now prove that such \(f\) does not factor only \(f_{D_3^*}\) and \(f_{D_0}\) but every projection of the cone \(A\); thus establishing the universality of the cone \(\phi(D_1)\).

  Fix a \(D\) such that \(D_1< D\leq D_3\). When \(D\in\cat{D}_n^\alpha\) then \(D_0\leq D\) and so by using the factorization introduced above and the definition of a cone one obtains
  \begin{align*}
    \phi(D_0^*\leq D)\circ f &= \phi(D_0\leq D)\circ\phi(D_0^*\leq D_0)\circ f\\
                             &= \phi(D_0\leq D)\circ f_{D_0}\\
                             &= f_D
  \end{align*}
  so this case is solved. On the other hand when \(D\not\in\cat{D}_n^\alpha\) there is a \(D'\in\cat{D}_n^\alpha\) such that \(D = D'^*\); moreover this \(D'\) is such that \(D_0^*\leq D_0\leq D'\leq D_3\). Consider Diagram \ref{diagram:last_one} where the squares are again pullbacks by construction; we are interested in proving that the triangle \((*)\) commutes. By the following computations
  \begin{align*}
    \phi(D'^*\leq D_3^*)\circ\phi(D_0^*\leq D'^*)\circ f &= \phi(D_0^*\leq D_3^*)\circ f\\
                                                         &= f_{D_3^*}\\
                                                         &= \phi(D'^*\leq D_3^*)\circ f_{D'^*}
  \end{align*}
  \begin{align*}
    \phi(D'^*\leq D')\circ\phi(D_0^*\leq D'^*)\circ f &= \phi(D_0\leq D')\circ\phi(D_0^*\leq D_0)\circ f\\
                                                      &= \phi(D_0\leq D')\circ f_{D_0}\\
                                                      &= f_{D'}\\
                                                      &= \phi(D'^*\leq D')\circ f_{D'^*}
  \end{align*}
  we obtain that \(f_{D'^*}\) and \(\phi(D_0^*\leq D'^*)\circ f\) both factorize the same cone on the pullback that is the rightmost bottom square; thus are equal.

  \begin{figure}[h]
    \begin{center}
      \begin{tikzcd}[sep=huge]
        A \ar[ddr, bend right, "f_{D_0}"] \ar[rd, dashed, "f" description] \ar[drr, bend left, "f_{D'^*}"] \ar[drrr, bend left, "f_{D_3^*}"] \ar[drr, phantom, "(*)"]& & &\\
        & \phi(D_0^*) \ar[r, ""] \ar[d, ""] & \phi(D'^*) \ar[r, ""] \ar[d, ""] & \phi(D_3^*) \ar[d, ""]\\
        & \phi(D_0) \ar[r, ""] & \phi(D') \ar[r, ""] & \phi(D_3)
      \end{tikzcd}
    \end{center}
    \caption{}
    \label{diagram:last_one}
  \end{figure}
  
  Finally we have that \(\lim_{D_1<D\leq D_2}\phi(D) = \phi(D_1)\) so the factorization \(\phi(D_1)\to\lim_{D_1<D\leq D_2}\phi(D)\) in the Lemma's statement is obviously the identity \(1_{\phi(D_1)}\) that is epi.
\end{proof}

\begin{lemma}
  \label{lemma:third_lemma}
  Consider an element \(D_0\in\cat{D}\) and a functor \(\fun{\Gamma}{\downarrow D_0}{\cat{A}}\) such that for every \(D_1\leq D_2\leq D_0\) the factorization
  \begin{equation*}
    \begin{tikzcd}[sep=huge]
      \Gamma(D_1) \ar[r, ""] & \displaystyle\lim_{D_1\leq D\leq D_2}\Gamma(D)
    \end{tikzcd}
  \end{equation*}
  is an epimorphism. Consider then an exact functor \(\fun{F}{\cat{A}}{\catname{Ab}}\) and a natural transformation \(\nat{\alpha_{D_0}}{\homset{\cat{A}}{\Gamma(D_0)}{-}}{F}\). Then \(\alpha_{D_0}\) can be written as the composite
  \begin{equation*}
    \begin{tikzcd}[sep=huge]
      \homset{\cat{A}}{\Gamma(D_0)}{-} \ar[r, "s_{D_0}"] & \displaystyle\colim_{D\leq D_0}\homset{\cat{A}}{\Gamma(D)}{-} \ar[r, "\alpha"] & F
    \end{tikzcd}
  \end{equation*}
  with \(s_{D_0}\) injection of the colimit and \(\alpha\) another natural transformation.
\end{lemma}

\begin{proof}
  Let \(\mathcal{S}\) be the set of all pairs \((S, (\alpha_D)_{D\in S})\) where \(S\) is a final segment of \(\downarrow D_0\) and the \(\nat{\alpha_D}{\homset{\cat{A}}{\Gamma(D)}{-}}{F}\) are natural transformations that constitute a cocone for the functor \(\yo^*\circ\Gamma\) restricted to all the \(D\) in \(S\). We endow \({\cal S}\) with a poset structure by imposing \((S, (\alpha_D)_{D\in S}) \leq (S', (\alpha'_D)_{D\in S'})\) if and only if \(S\subseteq S'\) and \(\alpha'_D = \alpha_D\) for all \(D\in S\). Given a chain in \(\mathcal{S}\) we can take its union as an upper bound and so, by Zorn's Lemma, we know \({\cal S}\) has a maximal; let it be \((\overline{S}, (\alpha_D)_{D\in\overline{S}})\). If we are able to show that \(\overline{S} = \downarrow D_0\) then the natural transformations \(\alpha_D\) for \(D\leq D_0\) will form a cocone on \(\yo^*\circ\Gamma\); the unique factorization through the colimit will then provide the natural transformation \(\alpha\) as in the Lemma's thesis.

  By contradiction let's assume \(\overline{S}\not=\downarrow D_0\) i.e. there is \(D_1\not\in\overline{S}\) such that \(D_1\leq D_0\). The segment \([D_1, D_0]\) is finite by Lemma \ref{lemma:finite_segment} so the sub-order \(\{D\in[D_1, D_0]\colon D\not\in\overline{S}\}\) is finite as well; let then \(D_2\) be one of its maximals, this element is such that every \(D\) for which \(D_2<D\leq D_0\) holds is an element of \(\overline{S}\).

  The Yoneda Lemma (\ref{lemma:additive_yoneda_lemma}) ensures that the natural transformations \(\alpha_D\) from \(\homset{\cat{A}}{\Gamma(D)}{-}\) to \(F\) for \(D_2<D\leq D_0\) correspond to a family of elements \(a_D\in F(\Gamma(D))\) and moreover this correspondence is natural in \(D\): if \(D\leq D'\) Diagram \ref{diagram:y1} commutes. By chasing \(\alpha_D\) around the diagram we obtain \(F(\Gamma(D\leq D'))(a_D) = a_{D'}\)  and thus if we take the limit of \(F\circ\Gamma\) the family \((a_D)_{D_2<D\leq D_0}\) will correspond to a single element \(a\in\lim_{D_2<D\leq D_0}F(\Gamma(D))\)\footnote{Recall that the limit of a functor \(\fun{F}{\cat{C}}{\catname{Set}}\) is the subset of \(\prod_{C\in\cat{C}}F(C)\) composed of all the sequences \((x_C)_{C\in\cat{C}}\) such that for every \(C\in\cat{C}\) and every arrow arrow \(\fun{f}{C}{C'}\) in \(\cat{C}\) we have \(F(f)(x_C)=x_{C'}\). And that limits in \(\catname{Ab}\) are calculated as in \(\catname{Set}\) (see Proposition \ref{prop:ab_forgetful}).}; but \(F\) is left-exact, thus preserves finite limits, so \(a\in\lim_{D_2<D\leq D_0}F(\Gamma(D)) = F(\lim_{D_2<D\leq D_0}\Gamma(D))\).
  \begin{figure}[h]
    \begin{center}
      \begin{tikzcd}[column sep=3cm, row sep=1.5cm]
        \nathom{\homset{\cat{A}}{\Gamma(D)}{-}}{F} \ar[r, "\precomp{\homset{\cat{A}}{\Gamma(D\leq D')}{-}}"] \ar[d, "\theta_{F, D}"]& 
        \nathom{\homset{\cat{A}}{\Gamma(D')}{-}}{F} \ar[d, "\theta_{F, D'}"]\\
        F(\Gamma(D)) \ar[r, "F(\Gamma(D\leq D'))"] & F(\Gamma(D'))\\
        \alpha_D \ar[r, maps to, ""] \ar[d, maps to, ""] & \alpha_{D'} \ar[d, maps to, ""]\\
        a_D \ar[r, maps to,  ""] & a_{D'}
      \end{tikzcd}
    \end{center}
    \caption{}
    \label{diagram:y1}
  \end{figure}

  By hypothesis the factorization \(\fun{\beta}{\Gamma(D_2)}{\lim_{D_2<D\leq D_0}\Gamma(D)}\), obtained as discussed in the proof of Lemma \ref{lemma:limit_factorization_is_epi}, is epi and \(F\) is exact thus preserves epimorphisms via Proposition \ref{prop:left_condition}. It follows that \(\fun{F(\beta)}{F(\Gamma(D_2))}{F(\lim_{D_2<D\leq D_0}\Gamma(D))}\) is surjective and therefore there is an element \(a_{D_2}\in F(\Gamma(D_2))\) such that \(F(\beta)(a_{D_2}) = a\). For any \(D_2<D\leq D_0\) we have
  \begin{align*}
    F(\Gamma(D_2\leq D))(a_{D_2}) &= F(p_D\circ\beta)(a_{D_2})\\
                                  &= F(p_D)(F(\beta)(a_{D_2}))\\
                                  &= F(p_D)(a)\\
                                  &= a_D
  \end{align*}
  where \(p_D\) are the projections of the limit and so, since \(F\) preserves this limit, \(F(p_D)\) are the projections of \(F(\lim_{D_2<D\leq D_0}\Gamma(D)))\). Again by the Yoneda Lemma \(a_{D_2}\in F(\Gamma(D_2))\) corresponds to \(\nat{\alpha_{D_2}}{\homset{\cat{A}}{\Gamma(D_2)}{-}}{F}\) and by the naturality in \(D\) we obtain Diagram \ref{diagram:y2}.
    \begin{figure}[h]
    \begin{center}
      \begin{tikzcd}[column sep=3cm, row sep=1.5cm]
        F(\Gamma(D_2)) \ar[r, "F(\Gamma(D_2\leq D))"] \ar[d, "\theta_{F, D_2}"] & F(\Gamma(D)) \ar[d, "\theta_{F, D}"]\\
        \nathom{\homset{\cat{A}}{\Gamma(D_2)}{-}}{F} \ar[r, "\precomp{\homset{\cat{A}}{\Gamma(D_2\leq D)}{-}}"] & \nathom{\homset{\cat{A}}{\Gamma(D)}{-}}{F}
      \end{tikzcd}
    \end{center}
    \caption{}
    \label{diagram:y2}
  \end{figure}
  By the above computation we have that \((\theta_{F, D}\circ F(\Gamma(D_2\leq D)))(a_{D_2}) = \alpha_D\) and so, by chasing \(a_{D_2}\) on the other path in the diagram (the ``down then right'' composite) we obtain \(\alpha_D = \alpha_{D_2}\circ\homset{\cat{A}}{\Gamma(D_2\leq D)}{-}\). But since \(D_2\not\in\overline{S}\) we can now strictly extend \((\overline{S}, (\alpha_D)_{D\in\overline{S}})\) to \((\overline{S}\cup\{D_2\}, (\alpha_D)_{D\in\overline{S}\cup D_2})\) because we have just proved that the new injection \(\alpha_{D_2}\) makes all the needed triangles commute. This is a contradiction with the maximality of \((\overline{S}, (\alpha_D)_{D\in\overline{S}})\) and thus \(\overline{S}=\downarrow D_0\).
\end{proof}

\begin{theorem}[Mitchell's Embedding Theorem]
  \label{teo:mitchell}
  Every small abelian category admits a full, faithful and exact embedding in \(\catname{Mod}_R\), the category of left modules over a ring \(R\).
\end{theorem}

\begin{proof}
  The proof is divided in four steps.\\

  \noindent\emph{Step 1: construction of the embedding}.\\

  \noindent Let \(\fun{U}{\cat{A}}{\catname{Ab}}\) be the faithful and exact embedding of Theoreem \ref{teo:faithful_embedding} and recall from Proposition \ref{prop:category_of_additive_functors} that \(\add{\cat{A}}{\catname{Ab}}\) is preadditive. The subcategory composed of \(U\) and all the natural transformations in \(\nathom{U}{U}\) is a preadditive category with a single object, thus it is a ring (as observed in Example \ref{ex:ring_preadditive}). Let \(R\) be \(\nathom{U}{U}\).

  Given \(A\in\cat{A}\) we define a scalar multiplication on \(U(A)\) as follows.
  \fundef{\cdot}{R\times U(A)}{U(A)}{(r, x)}{r_A(x)}
  Now since \(r_A\) is a group homomorphism (it is, by definition, a map in \(\catname{Ab}\)) and addition on \(R = \nathom{U}{U}\) is defined pointwise (because so is in \(\add{\cat{A}}{\catname{Ab}}\)) we get that \(U(A)\) has been endowed with an \(R\)-module structure. The following computation shows that if \(\fun{f}{A}{B}\) is an arrow in \(\cat{A}\) then \(\fun{U(f)}{U(A)}{U(B)}\) is an \(R\)-linear map with respect to the newly introduced \(R\)-module structure: if \(x\in U(A)\) then
  \begin{equation*}
    U(f)(r\cdot x) = U(f)(r_A(x)) = r_B(U(f)(x)) = r\cdot U(f)(x).
  \end{equation*}

  Now let \(\fun{\Phi}{\catname{Mod}_R}{\catname{Ab}}\) be the forgetful functor that forgets the module structure. We have that our embedding \(U\) factors through \(\catname{Mod}_R\) via a new functor \(\fun{V}{\cat{A}}{\catname{Mod}_R}\) as follows
  \begin{equation*}
    \begin{tikzcd}[sep=huge]
      \cat{A} \ar[r, "V"] & \catname{Mod}_R \ar[r, "\Phi"] & \catname{Ab}.
    \end{tikzcd}
  \end{equation*}
  Since \(U\) is faithful so is \(V\) and since limits and colimits are calculated in the same way in \(\catname{Ab}\) and \(\catname{Mod}_R\) (see Proposition \ref{prop:mod_ab_limits_colimits}) exactness of \(U\) implies exactness of \(V\) via Proposition \ref{prop:characterization_of_exact_functors}. If we succeed in showing that \(V\) is also full then the proof is complete.\\

  \noindent\emph{Step 2: canonical presentation of representable functors}.\\

  \noindent Given an object \(A\in\cat{A}\) and the zero arrow \(0_{A\to{\bf 0}}\) we consider the unique element \(*\) of \(\cat{D}_0\) and the index \(\alpha + 1\) that corresponds to the pair \((*, 0_{A\to{\bf 0}})\) in the construction of \(\cat{D}_1\) (see proof of Theorem \ref{teo:faithful_embedding}). Following the construction \(\cat{D}_0^{\alpha+1}\) is obtained by ``duplication'' of \(\downarrow *\) that in this case is \(\cat{D}_0^\alpha\) itself; let \(\delta_A\) be the copy of \(*\) introduced at this step so that one has \(\phi(\delta_A) = A\) and \(\phi(\delta_A\leq *) = 0_{A\to{\bf 0}}\).

  For every \(D,D'\in\cat{D}\) such that \(D\leq D'\leq\delta_A\) consider Diagram \ref{diagram:define_pi} where \((u_D, v_D)\) is the kernel pair of \(\phi(D\leq\delta_A)\) i.e. the pullback of \(\phi(D\leq\delta_A)\) along itself.
  \begin{figure}[h]
    \begin{center}
      \begin{tikzcd}[sep=huge]
        \pi(D) \ar[r, shift left = 1, "u_D"] \ar[r, shift right = 1, "v_D"'] \ar[d, "\pi(D\leq D')"'] &
        \phi(D) \ar[r, two heads, "\phi(D\leq\delta_A)"] \ar[d, "\phi(D\leq D')"] &
        \phi(\delta_A) = A \ar[d, "1_A"]\\
        \pi(D') \ar[r, shift left = 1, "u_{D'}"] \ar[r, shift right = 1, "v_{D'}"'] &
        \phi(D') \ar[r, two heads, "\phi(D'\leq\delta_A)"] &
        \phi(\delta_A) = A
      \end{tikzcd}
    \end{center}
    \caption{}
    \label{diagram:define_pi}
  \end{figure}
  \noindent By functoriality of \(\phi\) the right square commutes; from this fact we obtain
  \begin{align*}
    \phi(D'\leq\delta_A)\circ\phi(D\leq D')\circ u_D &= 1_A\circ\phi(D\leq\delta_A)\circ u_D\\
                                                     &= 1_A\circ\phi(D\leq\delta_A)\circ v_D\\
                                                     &= \phi(D'\leq\delta_A)\circ\phi(D\leq D')\circ v_D
  \end{align*}
  and so there is a unique factorization \(\pi(D\leq D')\) through the kernel pair of \(\phi(D'\leq\delta_A)\). This defines a functor \(\fun{\pi}{\downarrow\delta_A}{\cat{A}}\).

  We now recall that \(\phi(D\leq\delta_A)\) is epi by construction of \(\phi\) and thus is a cokernel because \(\cat{A}\) is abelian, but cokernels are coequalizers and if a coequalizer has a kernel pair then it is the coequalizer of its own kernel pair; we thus conclude that \(\phi(D\leq\delta_A)\) is the coequalizer of the pair \((u_D, v_D)\). Now if we apply \(\yo^*\) to the top line of Diagram \ref{diagram:define_pi} we obtain
  \begin{equation*}
    \begin{tikzcd}[sep=3cm]
      \homset{\cat{A}}{\pi(D)}{-} &
      \homset{\cat{A}}{\phi(D)}{-} \ar[l, shift right = 1.5, "\homset{\cat{A}}{u_D}{-}"'] \ar[l, shift left = 1.5, "\homset{\cat{A}}{v_D}{-}"] &
      \homset{\cat{A}}{A}{-} \ar[l, tail, "\homset{\cat{A}}{\phi(D'\leq\delta_A)}{-}"]
    \end{tikzcd}
  \end{equation*}
  in \(\add{\cat{A}}{\catname{Ab}}\). This is an equalizer because it is known that the covariant Yoneda Embedding preserves limits and thus the contravariant version \(\yo^*\) transforms colimits in limits (in this particular case, coequalizers in equalizers). By computing the colimit on each piece of the diagram for \(D\leq\delta_A\) and \(D\in\cat{D}\) one has
  \begin{equation*}
    \begin{tikzcd}[sep=3cm]
      P &
      U \ar[l, shift right = 1.5, "u"'] \ar[l, shift left = 1.5, "v"] &
      \homset{\cat{A}}{A}{-} \ar[l, tail, "w"]
    \end{tikzcd}
  \end{equation*}
  The colimit of the central term really is \(U\) because we're computing it over an initial segment \(\downarrow\delta_A\) of \(\cat{D}\) and the inclusion is a final functor. On the right side we're computing a constant filtered colimit so we obtain the same functor we started with. By Proposition \ref{prop:limits_and_filtered_colimits} filtered colimits commute with equalizers in \(\catname{Ab}\) and thus \(w =\ker(u, v) \).\\

  \noindent\emph{Step 3: \(\pi\) satisfies the hypothesis of Lemma \ref{lemma:third_lemma}}.\\

  \noindent Consider \(D_1\leq D_2\leq\delta_A\); then
  \begin{equation*}
    \begin{tikzcd}[sep=3cm]
      \displaystyle\lim_{D_1<D\leq D_2}\pi(D) \ar[r, shift left = 1.5, "\lim u_D"] \ar[r, shift right = 1.5, "\lim v_D"']&
      \displaystyle\lim_{D_1<D\leq D_2}\phi(D) \ar[r, "l"] &
      A
    \end{tikzcd},
  \end{equation*}
  obtained by taking the ``piece by piece'' limit for \(D_1<D\leq D_2\) of the top line in Diagram \ref{diagram:define_pi}, is a kernel pair by Proposition \ref{prop:interchange_of_limits}. Now consider Diagram \ref{diagram:tris} where all the squares are pullbacks and the factorization \(\fun{m}{\phi(D_1)}{\lim_{D_1<D\leq D_2}\phi(D)}\) is epi by Lemma \ref{lemma:limit_factorization_is_epi}.
  \begin{figure}[h]
    \begin{center}
      \begin{tikzcd}[sep=huge]
        \pi(D_1) \ar[r ,two heads, "a"] \ar[d, two heads, "a'"'] & \bullet \ar[r, "c"] \ar[d, two heads, "b"] & \phi(D_1) \ar[d, two heads, "m"]\\
        \bullet \ar[r, two heads, "b'"] \ar[d, "c'"'] & \displaystyle\lim_{D_1<D\leq D_2}\pi(D) \ar[d, "\lim v_D"] \ar[r, "\lim u_D"] & \displaystyle\lim_{D_1<D\leq D_2}\phi(D) \ar[d, "l"]\\
        \phi(D_1) \ar[r, two heads, "m"] & \displaystyle\lim_{D_1<D\leq D_2}\phi(D) \ar[r, "l"] & A
      \end{tikzcd}
    \end{center}
    \caption{}
    \label{diagram:tris}
  \end{figure}

  \noindent Now, with \(p_D\) projections of \(\lim_{D_1<D\leq D_2}\phi(D)\), by using the fact that \(m\) and \(l\) are factorizations and recalling that \(A =\phi(\delta_A)\) one obtains
  \begin{align*}
    l\circ m &= \phi(D\leq\delta_A)\circ p_D\circ m\\
             &= \phi(D\leq\delta_A)\circ\phi(D_1\leq D)\\
             &= \phi(D_1\leq\delta_A)    
  \end{align*}
  from which follows that the outer square is the pullback (more precisely the kernel pair) that defines \(\pi(D_1)\); particularly we have, referring back to Diagram \ref{diagram:define_pi}, \(c\circ a = u_{D_1}\) and \(c'\circ a' = v_{D_1}\). By applying Proposition \ref{prop:pullback_of_epi} we deduce that the upper left square is made entirely of epimorphisms and thus so is the diagonal. If we can prove that this diagonal is the factorization \(\pi(D_1)\to\lim_{D_1<D\leq D_2}\pi(D)\) then the hypothesis of Lemma \ref{lemma:third_lemma} hold for \(\pi\).

  Let's prove that \(q_D\circ b\circ a = \pi(D_1\leq D)\) where \(q_D\) are the projections of \(\lim_{D_1<D\leq D_2}\pi(D)\); let \(p_D\) denote the projections of \(\lim_{D_1<D\leq D_2}\phi(D)\) (as before) and consider the following:
  \begin{align*}
    u_D\circ q_D\circ b\circ a &= p_D\circ\lim u_D\circ b\circ a\\
                               &= p_D\circ m\circ c\circ a\\
                               &= \phi(D_1\leq D)\circ u_{D_1}\\
                               &= u_D\circ\pi(D_1\leq D).
  \end{align*}
  The same computations can be performed on the other side of the diagram so that one obtains \(v_D\circ q_D\circ b\circ a = v_D\circ\pi(D_1\leq D)\) as well. This shows that \(q_D\circ b\circ a = \pi(D_1\leq D)\) i.e. that the diagonal of the upper left square really is the needed factorization.\\

  \noindent\emph{Step 4: V is full}.\\

  \noindent Choose \(A,B\in\cat{A}\) and a group homomorphism \(\fun{\varphi}{U(A)}{U(B)}\) that is also an \(R\)-linear map with respect to the structure introduced at Step 1. The monomorphism \(w\colon\homset{\cat{A}}{A}{-}\rightarrowtail U\) of Step 2 corresponds via the Yoneda Lemma to an element \(a = w_A(1_A)\in U(A)\); so \(\varphi(a)\in U(B)\) in turn corresponds to a natural transformation \(\nat{\beta}{\homset{\cat{A}}{B}{-}}{U}\) of components defined by \(\beta_C(g) = U(g)(\varphi(a))\) for all \(C\in\cat{A}\) and \(g\in\homset{\cat{A}}{B}{C}\). Finally consider Diagram \ref{diagram:facto}; we recall that \(w = \ker(u, v)\) and thus if \(u\circ\beta = v\circ\beta\) then we obtain a factorization of \(\beta\) through \(w\) that will have the form \(\homset{\cat{A}}{f}{-}\) for some \(\fun{f}{A}{B}\).

  \begin{figure}[ht]
    \begin{center}
      \begin{tikzcd}[sep=huge]
        P \ar[r, bend left = 60, "t"] & U \ar[l, shift right = 1.5, "u"'] \ar[l, shift left = 1.5, "v"] & \homset{\cat{A}}{A}{-} \ar[l, tail, "w"']\\
        & \homset{\cat{A}}{B}{-} \ar[u, "\beta"] \ar[ur, dashed, "\homset{\cat{A}}{f}{-}"'] &
      \end{tikzcd}
    \end{center}
    \caption{}
    \label{diagram:facto}
  \end{figure}

  Suppose toward a contradiction that \(u\circ\beta\not= v\circ\beta\). Since \(U(B) = \colim_{D\leq\delta_B}\homset{\cat{A}}{\phi(D)}{B}\) the element \(\varphi(a)\in U(B)\) can be represented by an arrow \(\fun{x}{\phi(D)}{B}\) for an appropriate \(D\leq\delta_B\). 

  One now has that \(x\circ u_D\not=x\circ v_D\). Absurdly suppose that \(x\circ u_D = x\circ v_D\) and let \(s_D\) be the injection \(\homset{\cat{A}}{\pi(D)}{B}\to P(B)\) and \(s'_D\) the injection \(\homset{\cat{A}}{\phi(D)}{B}\to U(B)\). One thus obtains
  \begin{gather*}
    s_D(x\circ u_D) = s_D(\homset{\cat{A}}{u_D}{B}(x)) = u_B(s'_D(x)) = u_B(\varphi(a)),\\
    s_D(x\circ v_D) = s_D(\homset{\cat{A}}{v_D}{B}(x)) = v_B(s'_D(x)) = v_B(\varphi(a)).
  \end{gather*}
  Now by naturality (in the functorial variable) of the Yoneda Lemma we obtain Diagram \ref{diagram:y3}. Chasing \(\beta\) through the two down-right composites we obtain \(u_B(\varphi(a))\) and \(v_B(\varphi(a))\) that are equal by the previous relations and the absurd assumption; since \(\theta_{P,B}\) is a bijection their coimages in \(\nathom{\homset{\cat{A}}{B}{-}}{P}\) are equal as well. Since the diagram commutes those coimages are \(u\circ\beta\) and \(v\circ\beta\) that are not equal by our first absurd hypothesis. By contradiction \(x\circ u_D\not= x\circ v_D\).
  \begin{figure}[ht]
    \begin{center}
      \begin{tikzcd}[column sep=3cm, row sep=1.5cm]
        \nathom{\homset{\cat{A}}{B}{-}}{U} \ar[d, "\theta_{U,B}"] \ar[r, shift left = 1.5, "\postcomp{u}"] \ar[r, shift right = 1.5, "\postcomp{v}"'] &
        \nathom{\homset{\cat{A}}{B}{-}}{P} \ar[d, "\theta_{P, B}"]\\
        U(B) \ar[r, shift left = 1.5, "u_B"] \ar[r, shift right = 1.5, "v_B"'] & P(B)
      \end{tikzcd}
    \end{center}
    \caption{}
    \label{diagram:y3}
  \end{figure}

  Moreover it is also the case that \(s_D\circ\homset{\cat{A}}{x\circ u_D}{-} = u\circ\beta\) (see left triangle of Diagram \ref{diagram:two_triangles}) where \(\nat{s_D}{\homset{\cat{A}}{\pi(D)}{-}}{P}\) is now the injection of the colimit (of functors) \(P\). To prove this we shall prove that the natural transformations are equal component by component; thus, fixed \(C\in\cat{A}\), we want to prove that \(u_C\circ\beta_C = (s_D)_C\circ\precomp{(x\circ u_D)}\). For \(g\in\cat{A}(B, C)\) one has that Diagram \ref{diagram:s} commutes; where \(\nat{s'_D}{\homset{\cat{A}}{\phi(D)}{-}}{U}\) is an injection of \(U\).
  \begin{figure}[ht]
    \begin{center}
      \begin{tikzcd}[sep=huge]
        U(B) \ar[r, "U(g)"] & U(C) \ar[r, "u_C"] & P(C)\\
        \homset{\cat{A}}{\phi(D)}{B} \ar[u, "(s'_D)_B"] \ar[r, "\postcomp{g}"] &
        \homset{\cat{A}}{\phi(D)}{C} \ar[u, "(s'_D)_C"] \ar[r, "\precomp{u_D}"] &
        \homset{\cat{A}}{\pi(D)}{C} \ar[u, "(s_D)_C"]
      \end{tikzcd}
    \end{center}
    \caption{}
    \label{diagram:s}
  \end{figure}

  \noindent The following computations are thus justified.
  \begin{align*}
    u_C(\beta_C(g)) &= u_C(U(g)(\varphi(a)))\\
                    &= u_C(U(g)((s'_D)_B(x)))\\
                    &= (u_C\circ U(g)\circ (s'_D)_B))(x)\\
                    &= ((s_D)_C\circ \precomp{u_D}\circ \postcomp{g})(x)\\
                    &= (s_D)_C(\precomp{(x\circ u_D)}(g))
  \end{align*}
  And so \(s_D\circ\homset{\cat{A}}{x\circ u_D}{-} = u\circ\beta\). The same can be repeated substituting \(v\) for \(u\) and \(v_D\) for \(u_D\) thus leading us to Diagram \ref{diagram:two_triangles} where the mono \(\nat{w'}{\homset{\cat{A}}{\pi(D)}{-}}{U}\) is obtained from Step 2 by setting \(A = \pi(D)\). Moreover \(\pi\) satisfies the hypothesis of Lemma \ref{lemma:third_lemma} by Step 3 and thus there is a natural transformation \(\nat{t}{P}{U}\) such that \(t\circ s_D = w'\).

  \begin{figure}
    \begin{center}
      \begin{tikzcd}[sep=3cm]
        \homset{\cat{A}}{B}{-} \ar[r, shift left = 1.5, "u\circ\beta"] \ar[r, shift right = 1.5, "v\circ\beta"']
        \ar[dr, shift left = 1.5, "\homset{\cat{A}}{x\circ u_D}{-}"] \ar[dr, shift right = 1.5, "\homset{\cat{A}}{x\circ v_D}{-}"'] &
        P \ar[r, dashed, "t"]& U\\
        & \homset{\cat{A}}{\pi(D)}{-} \ar[u, "s_D"] \ar[ur, tail, "w'"]&
      \end{tikzcd}
    \end{center}
    \caption{}
    \label{diagram:two_triangles}
  \end{figure}

  It is now the case that
  \begin{equation*}
    \tag{\(*\)}
    t\circ u\circ\beta = w'\circ\homset{\cat{A}}{x\circ u_D}{-} \not= w'\circ\homset{\cat{A}}{x\circ v_D}{-} = t\circ v\circ\beta
  \end{equation*}
  since \(x\circ u_D\not= x\circ v_D\) implies \(\homset{\cat{A}}{x\circ u_D}{-}\not=\homset{\cat{A}}{x\circ v_D}{-}\) and \(w'\) is mono. Moreover since \(t\circ u\) and \(t\circ v\) are natural transformations from \(U\) to \(U\) they are elements of the ring \(R = \nathom{U}{U}\) of Step 1. For all \(C\in\cat{A}\) and \(\fun{g}{B}{C}\) the maps \(\varphi\) and \(U(g)\) are \(R\)-linear and thus
  \begin{align*}
    (t_C\circ u_C\circ \beta_C)(g) &= ((t\circ u)_C\circ U(g)\circ\varphi)(a)\\
                                   &= (U(g)\circ(t\circ u)_B\circ\varphi)(a)\\
                                   &= (U(g)\circ\varphi\circ(t\circ u)_A)(a)\\
                                   &= (U(g)\circ\varphi\circ t_A\circ u_A\circ w_A)(1_A)\\
                                   &= (U(g)\circ\varphi\circ t_A\circ v_A\circ w_A)(1_A)\\
                                   &= (U(g)\circ\varphi\circ (t\circ v)_A)(a)\\
                                   &= (U(g)\circ(t\circ v)_B\circ\varphi)(a)\\
                                   &= ((t\circ v)_C\circ U(g)\circ\varphi)(a)\\
                                   &= (t_C\circ v_C\circ \beta_C)(g).
  \end{align*}
  But this shows that \(t\circ u\circ \beta = t\circ v\circ \beta\); a contradiction with \((*)\).

  So we have obtained a morphism \(\fun{f}{A}{B}\) such that \(\homset{\cat{A}}{f}{-}\circ w = \beta\); to complete the proof we will show that \(\varphi = U(f)\). Fix an element \(x\in U(A)\), by the Yoneda Lemma \(x\) is associated with a natural transformation \(\nat{\chi}{\homset{\cat{A}}{A}{-}}{U}\). Consider also the natural transformation \(\nat{w}{\homset{\cat{A}}{A}{-}}{U}\) of Step 2 that corresponds to an element of \(U(A)\) by Yoneda; this last element is represented by an arrow \(\fun{y}{\phi(D)}{A}\) (as before). The situation is represented in Diagram \ref{diagram:chi}.

  \begin{figure}
    \begin{center}
      \begin{tikzcd}[sep=3cm]
        U & \homset{\cat{A}}{\phi(D)}{-} \ar[l, "\zeta"] \ar[d, "s_D"]\\
        \homset{\cat{A}}{A}{-} \ar[u, "\chi"] \ar[ur, tail, "\homset{\cat{A}}{y}{-}" near start] \ar[r, tail, "w"] &
        U \ar[lu, dashed, "r"' near start]
      \end{tikzcd}
    \end{center}
    \caption{}
    \label{diagram:chi}
  \end{figure}

  Since \(s_D\circ\homset{\cat{A}}{y}{-} = w\) and \(w\) is mono \(\homset{\cat{A}}{y}{-}\) must be mono too. So if we have two arrow \(\fun{p, q}{A}{B}\) then
  \begin{equation*}
    \homset{\cat{A}}{y}{-}\circ\homset{\cat{A}}{p}{-} = \homset{\cat{A}}{y}{-}\circ\homset{\cat{A}}{q}{-}
  \end{equation*}
  implies \(\homset{\cat{A}}{p}{-} = \homset{\cat{A}}{q}{-}\) that in turn implies \(p = q\). In other words \(p\circ y = q\circ y\) implies \(p = q\) so \(y\) is actually an epimorphism. By exactness of \(U\) it follows that \(U(y)\) is an epimorphism as well, but \(U(y)\in\catname{Ab}\), and thus \(U(y)\) is a surjection from \(U(\phi(D))\) to \(U(A)\).

  Now it is possible to chose a \(z\in U(\phi(D))\) such that \(U(y)(z) = x\) and let \(\nat{\zeta}{\homset{\cat{A}}{\phi(D)}{-}}{U}\) be the corresponding natural transformation given by Yoneda. By naturality of the Yoneda Lemma we obtain the following commutative square
  \begin{figure}[ht]
    \begin{center}
      \begin{tikzcd}[column sep=3cm, row sep=1.5cm]
        \nathom{\homset{\cat{A}}{\phi(D)}{-}}{U} \ar[r, "\precomp{\homset{\cat{A}}{y}{-}}"] & \nathom{\homset{\cat{A}}{A}{-}}{U}\\
        U(\phi(D)) \ar[u, "\theta_{U,\phi(D)}"] \ar[r, "U(y)"] & U(A) \ar[u, "\theta_{U, A}"']
      \end{tikzcd}
    \end{center}
    \captionsetup{labelformat=empty}
  \end{figure}
  in which chasing \(z\in U(\phi(D))\) through the two possible paths reveals that \(\zeta\circ\homset{\cat{A}}{y}{-} = \chi\). Now by Lemma \ref{lemma:limit_factorization_is_epi} and Theorem \ref{teo:faithful_embedding} the functors \(\phi\) (restricted to \(\downarrow D\)) and \(U\) satisfy the hypothesis of Lemma \ref{lemma:third_lemma} thus there is a natural transformation \(\nat{r}{U}{U}\) such that \(r\circ s_D = \zeta\) (and this completes Diagram \ref{diagram:chi}). Moreover one also has
  \begin{equation*}
    r\circ w = r\circ s_D\circ\homset{\cat{A}}{y}{-} = \zeta\circ\homset{\cat{A}}{y}{-} = \chi.
  \end{equation*}

  Finally since \(\varphi\) is an \(R\)-linear map and \(r\in R = \nathom{U}{U}\) we have
  \begin{align*}
    \varphi(x) &= \varphi(\chi_A(1_A))\\
               &= \varphi(r_A\circ w_A(1_A))\\
               &= r_B\circ\varphi(w_A(1_A))\\
               &= r_B\circ\varphi(a)\\
               &= r_B\circ\beta_B(1_B)\\
               &= (r_B\circ w_B\circ \homset{\cat{A}}{f}{-}_B)(1_B)\\
               &= \chi_B(f)\\
               &= U(f)(x).
  \end{align*}
  This last calculation shows that \(\varphi = U(f)\) thus \(V\) is a full functor.
\end{proof}

\begin{remark}
  \label{rem:final_remark}
  Since the embedding \(\fun{V}{\cat{A}}{\catname{Mod}_R}\) is fully faithful it reflects isomorphisms, \(V\) is also exact so it preserves finite limits and colimits and \(\cat{A}\) is finitely complete and cocomplete; thus \(V\) reflects finite limits and colimits (see \cite[Proposition 2.9.7]{handbook1}). Particularly it reflects monomorphisms and epimorphisms, thus the construction of epi-mono factorizations (and exact sequences).
\end{remark}


\newpage
\section{Applications: classical lemmas}
\label{sec:salamanders}
The classical lemmas of homological algebra (such as the \(3\times 3\) Lemma and the Snake Lemma) for abelian categories, as shown in \cite{salaman}, are quickly obtained from Lemma \ref{lemma:salamander}, called the ``Salamander Lemma''. As a practical demonstration of the power of Mitchell's Embedding Theorem we will prove this lemma and its consequences by working with modules only.\footnote{It should be said that this is in not the only way of obtaining such results; indeed one can work with pseudo-elements as in \cite[\S 1.9, \S 1.10]{handbook2} or \cite[\S VIII.4]{catwork}.}

\begin{definition}
  \label{def:chain_complex}
  In an abelian category \(\cat{A}\) a {\bf chain complex} \(A_\bullet\) is a collection of objects and arrows \((A_n, \fun{f_n}{A_n}{A_{n-1}})_{n\in\mathbb{Z}}\) such that \(f_n\circ f_{n+1} = 0\) for all \(n\in\mathbb{Z}\).
  \begin{center}
    \begin{tikzcd}
      \cdots \ar[r, "f_2"] & A_1 \ar[r, "f_1"] & A_0 \ar[r, "f_0"] & A_{-1} \ar[r, "f_{-1}"] & \cdots
    \end{tikzcd}
  \end{center}
\end{definition}

\begin{definition}
  \label{def:chain_map}
  Given two chain complexes \(A_\bullet = (A_n, f_n)_{n\in\mathbb{Z}}\) and \(B_\bullet = (B_n, g_n)_{n\in\mathbb{Z}}\) a {\bf chain map} \(h\colon A_\bullet\to B_\bullet\) is a collection of arrows \((\fun{h_n}{A_n}{B_n})_{n\in\mathbb{Z}}\) such that \(h_{n-1}\circ f_n = g_n\circ h_n\).
  \begin{center}
    \begin{tikzcd}
      \cdots \ar[r, "f_2"] & A_1 \ar[r, "f_1"] \ar[d, "h_1"] & A_0 \ar[r, "f_0"] \ar[d, "h_0"] & A_{-1} \ar[r, "f_{-1}"] \ar[d, "h_{-1}"] & \cdots\\
      \cdots \ar[r, "g_2"] & B_1 \ar[r, "g_1"] & B_0 \ar[r, "g_0"] & B_{-1} \ar[r, "g_{-1}"] & \cdots
    \end{tikzcd}
  \end{center}
\end{definition}

\begin{definition}
  \label{def:homology}
  Given a chain complex \(A_\bullet\) in an abelian category \(\cat{A}\) we define the {\bf chain homology} of \(A_\bullet\) of degree \(n\) as
  \begin{equation*}
    H_n(A_\bullet) = \ker(f_{n})/\im(f_{n+1}).
  \end{equation*}

  Where we use the quotient notation to indicate the cokernel of the factorization of the image of \(f_{n+1}\) through the kernel of \(f_{n}\) (the dashed arrow in Diagram \ref{diagram:quotient}).

\end{definition}

\begin{figure}[h]
  \begin{center}
    \begin{tikzcd}[column sep = tiny]
      A_{n+1} \ar[rr, "f_{n+1}"] \ar[rd, two heads, "i"] & & A_{n} \ar[rr, "f_{n}"] & & A_{n-1} &\\
      & \im(f_{n+1}) \ar[ru, tail, "j"] \ar[rr, dashed, tail, ""] & & \ker(f_{n}) \ar[lu, tail, ""] \ar[rr, two heads, ""] & & H_n(A_\bullet)
    \end{tikzcd}
  \end{center}
  \caption{}
  \label{diagram:quotient}
\end{figure}

\begin{remark}
  \label{rem:exactness_and_homology}
  Referencing Diagram \ref{diagram:quotient} note that if the pair \((f_{n+1},f_{n})\) is exact then \(\ker(f_{n}) = \im(f_{n+1})\) by definition and thus \(H_n(A_\bullet) = 0\).
\end{remark}

\begin{remark}
  \label{rem:quotient_modules_maps}
  Let \(X,Y,Z,W\) be \(R\)-modules with \(Y\) submodule of \(X\) and \(W\) submodule of \(Z\); moreover let \(\fun{f}{X}{Z}\) be an \(R\)-linear map such that \(f(Y)\subseteq W\). Then the map \(\fun{\hat{f}}{X/Y}{Z/W}\) defined by \([x]\mapsto [f(x)]\) is an \(R\)-linear map.
\end{remark}

\begin{remark}
  \label{rem:induced_homology_maps}
  Let \(A_\bullet\) and \(B_\bullet\) be chain complexes in an abelian category \(\cat{A}\) and \(\fun{h}{A_\bullet}{B_\bullet}\) a chain map between them. Consider now the following diagram in \(\cat{A}\); however we can, via Theorem \ref{teo:mitchell}, assume we're dealing with modules.
    \begin{center}
    \begin{tikzcd}
      \cdots \ar[r, "f_2"] & A_1 \ar[r, "f_1"] \ar[d, "h_1"] & A_0 \ar[r, "f_0"] \ar[d, "h_0"] & A_{-1} \ar[r, "f_{-1}"] \ar[d, "h_{-1}"] & \cdots\\
      \cdots \ar[r, "g_2"] & B_1 \ar[r, "g_1"] & B_0 \ar[r, "g_0"] & B_{-1} \ar[r, "g_{-1}"] & \cdots
    \end{tikzcd}
  \end{center}
  If \(x\in\ker(f_0) \) then \(0 = (h_{-1}\circ f_0)(x) = (g_0\circ h_0)(x)\) so \(h_0(x)\in\ker(g_0)\). Moreover if \(x\in\im(f_1)\) then there is a \(x_1\in A_1\) such that \(f_1(x_1) = x\); now \((h_0\circ f_1)(x_1) = (g_1\circ h_1)(x_1)\) so \(h_0(x)\in\im(g_1)\). Applying Remark \ref{rem:quotient_modules_maps} we obtain that \(h_0\) induces a map \(H_0(A_\bullet)=\frac{\ker(f_0)}{\im(f_1)}\to\frac{\ker(g_0)}{\im(g_1)}=H_0(B_\bullet)\) between the homologies of order \(0\).

  The previous argument can be repeated forevery \(n\) so, in the end, we have that a chain map \(\fun{h}{A_\bullet}{B_\bullet}\) induces maps between the homologies of the same order of the two complexes.
\end{remark}

\begin{definition}
  \label{def:double_complex}
  A {\bf double complex} in an abelian category \(\cat{A}\) is a chain complex of chain complexes i.e. an infinite 2-dimesional grid of objects of \(\cat{A}\) as in Diagram \ref{diagram:double_complex} where every square commutes and the composition of two vertical or horizontal arrows is a zero morphsism.
\end{definition}

\begin{figure}[h]
  \begin{center}
    \begin{tikzcd}
      {} & {} \ar[d, ""] & {} \ar[d, ""] & {} \ar[d, ""] & {}\\
      {} \ar[r, ""] & \bullet \ar[r, ""] \ar[d, ""] & \bullet \ar[r, ""] \ar[d, ""] & \bullet \ar[r, ""] \ar[d, ""] & {}\\
      {} \ar[r, ""] & \bullet \ar[r, ""] \ar[d, ""] & \bullet \ar[r, ""] \ar[d, ""] & \bullet \ar[r, ""] \ar[d, ""] & {}\\
      {} \ar[r, ""] & \bullet \ar[r, ""] \ar[d, ""] & \bullet \ar[r, ""] \ar[d, ""] & \bullet \ar[r, ""] \ar[d, ""] & {}\\
      {} & {} & {} & {} & {}
    \end{tikzcd}
  \end{center}
  \caption{}
  \label{diagram:double_complex}
\end{figure}

\begin{remark}
  \label{rem:finite_double_complexes}
  Double complexes are by definition infinite but it is fine to consider finite ``pieces'' of a double complex as well since one can make them into full double complexes by ``attaching'' zeroes/kernels/cokernels to all sides.
\end{remark}

\begin{definition}
  \label{def:donor_receptor}
  Given an object \(A\) in a double complex label the surrounding arrows as follows.
  \begin{center}
    \begin{tikzcd}[sep=huge]
      \bullet \ar[dr, "p" ] & \bullet \ar[d, "a" ] & \bullet\\
      \bullet \ar[r, "d" ] & A \ar[dr, "q" ] \ar[d, "c"]  \ar[r, "b" ]& \bullet\\
      \bullet & \bullet & \bullet
    \end{tikzcd}
  \end{center}
  We now define:
  \begin{itemize}
  \item the {\bf horizontal homology} \(\hh{A} = \frac{\ker(b)}{\im(d)}\),
  \item the {\bf vertical homology} \(\vh{A} = \frac{\ker(c)}{\im(a)}\),
  \item the {\bf receptor} \(\rec{A} = \frac{\ker(b)\cap\ker(c)}{\im(p)}\),
  \item the {\bf donor} \(\don{A} = \frac{\ker(q)}{\im(a) + \im(d)}\).
  \end{itemize}
\end{definition}

\begin{definition}
  \label{def:intramural_maps}
  By applying Remark \ref{rem:quotient_modules_maps} to the identity \(1_A\) we obtain the following maps:
  \begin{center}
    \begin{tikzcd}
      & \rec{A} \ar[dr, ""] \ar[dl, ""] & \\
      \vh{A} \ar[dr, ""] & & \hh{A} \ar[dl, ""]\\
      & \don{A} &
    \end{tikzcd}
  \end{center}
  that we call {\bf intramural maps}.      
\end{definition}

\begin{lemma}
  \label{lemma:extramural_maps}
  Let \(\fun{f}{A}{B}\) be an arrow of a double complex; then this arrow induces a map \(\don{A}\to\rec{B}\) that sends \([x]\) to \([f(x)]\).
\end{lemma}

\begin{proof}
  Again we can suppose that \(A, B\) are modules and \(f\) a module homomorphism. The situation can be represented in the double complex as
  \begin{center}
    \begin{tikzcd}
      & C \ar[d, "c"] & &\\
      D \ar[r, "d"] & A \ar[r, "f"] & B \ar[r, "b"] \ar[d, "a"] & \bullet\\
      & & \bullet &
    \end{tikzcd}
  \end{center}
  so that \(\don{A}=\frac{\ker(a\circ f)}{\im(c) + \im(d)}\) and \(\rec{B}=\frac{\ker(a)\cap\ker(b)}{\im(f\circ c)}\).

  Let \(x\in\ker(a\circ f)\) so we have \(a(f(x)) = 0\) and \(b(f(x)) = 0\) since \(b\circ f = 0\) by definition of double complex; thus \(f(x)\in\ker(a)\cap\ker(b)\). Moreover if \(x\in\im(c)+\im(d)\) then there are elements \(\overline{c}\in C,\overline{d}\in D\) such that \(x = c(\overline{c}) + d(\overline{d})\), now \(f(x) = f(c(\overline{c})) + f(d(\overline{d})) = f(c(\overline{c}))\) because \(f\circ d = 0\) and so \(f(x)\in\im(f\circ c)\). This gives us, by Remark \ref{rem:quotient_modules_maps}, the map \(\don{A}\to\rec{B},[x]\mapsto [f(x)]\).
\end{proof}

\begin{definition}
  \label{def:extramural_maps}
  Arrows \(\don{A}\to\rec{B}\) induced by an arrow \(\fun{f}{A}{B}\) in a double complex as in Lemma \ref{lemma:extramural_maps} are called {\bf extramural maps}.
\end{definition}

\begin{lemma}
  \label{lemma:induced_and_extramural}
  Given an arrow \(\fun{f}{A}{B}\) in a double complex the induced arrow of Remark \ref{rem:induced_homology_maps} between the (vertical or horizontal) homologies is the appropriate composition of the following extramural and intramural maps:
  \begin{equation*}
    \vh{A}\to\don{A}\to\rec{B}\to\vh{B},\ \hh{A}\to\don{A}\to\rec{B}\to\hh{B}.
  \end{equation*}
\end{lemma}

\begin{proof}
  From Lemma \ref{lemma:extramural_maps}, Definition \ref{def:intramural_maps} and \ref{rem:quotient_modules_maps} we know how intramural and extramural maps operate. By chasing \([x]\in\vh{A}\) through the left composition in the Lemma's statement we obtain
  \begin{equation*}
    [x]\mapsto [x]\mapsto[f(x)]\mapsto[f(x)]
  \end{equation*}
  that is the map \([x]\mapsto[f(x)]\) i.e. exactly the map induced between the vertical homologies by \(f\). Nearly verbatim one handles the case of the right composition.
\end{proof}

\begin{lemma}[Salamander Lemma]
  \label{lemma:salamander}
  Consider the following diagram in a double complex:
  \begin{center}
    \begin{tikzcd}
      & C \ar[d, "c"] & \bullet \ar[d, ""] & \\
      E \ar[r, "e"] & A \ar[r, "f"] \ar[d, ""] & B \ar[d, "d"] \ar[r, "g"] & \bullet\\
      & \bullet & D & 
    \end{tikzcd}
  \end{center}
  then the following sequence of extramural and intramural maps is exact.
  \begin{center}
    \begin{tikzcd}[sep=small]
      \don{C} \ar[rr, ""] \ar[rd, ""] & & \hh{A} \ar[r, ""] & \don{A} \ar[r, ""] & \rec{B} \ar[r, ""] & \hh{B} \ar[rr, ""] \ar[rd, ""]& & \rec{D}\\
      & \rec{A} \ar[ru, ""] & & & & & \don{B} \ar[ru, ""]&
    \end{tikzcd}
  \end{center}  
\end{lemma}

\begin{proof}
  We know by the previous lemmas and definitions how those maps operate, so it is only a matter of checking exactness at each point of the sequence.
  \begin{enumerate}
  \item Exactness of \(\don{C}\to\hh{A}\to\don{A}\).
    
    Let \([x]\in\ker(\hh{A}\to\don{A})\). This means that \(x = c(\overline{c}) + e(\overline{e})\) for some \(\overline{c}\in C\) and \(\overline{e}\in E\). We immediately obtain \(x - c(\overline{c})\in\im(e) \) and thus \([x] = [c(\overline{c})]\) in \(\hh{A}\); hence \(\ker(\hh{A}\to\don{A})\subseteq\im(\don{C}\to\hh{A})\).

    On the other hand elements in \(\im(\don{C}\to\hh{A})\) are classes of elements in the image of \(c\) and since \(\don{A} = \frac{\ker(d\circ f)}{\im(c) + \im(e)}\) they are also elements of \(\ker(\hh{A}\to\don{A})\).
  \item Exactness of \(\hh{A}\to\don{A}\to\rec{B}\).

    Let \([x]\in\ker(\don{A}\to\rec{B})\) so there is a \(\overline{c}\in C\) such that \(f(x) = (f\circ c)(\overline{c})\); from this relation we obtain that \(f(x - c(\overline{c})) = 0\) hence \([x - c(\overline{c})]\in\hh{A}\). Moreover \([x] = [x - c(\overline{c})]\) in \(\don{A}\) because \(x - x + c(\overline{c}) \in\im(c)\); this shows that \([x] = (\hh{A}\to\don{A})([x - c(\overline{c})])\).

    Now let \([x]\in\hh{A}\) so \(x\in\ker(f)\); we have
    \begin{equation*}
      (\don{A}\to\rec{B})(\hh{A}\to\don{A})([x]) = (\don{A}\to\rec{B})([x]) = [f(x)] = 0.
    \end{equation*}
    Thus \(\im(\hh{A}\to\don{A})\subseteq\ker(\don{A}\to\rec{B})\).
  \item  Exactness of \(\don{A}\to\rec{B}\to\hh{B}\).

    Let \([x]\in\ker(\rec{B}\to\hh{B})\), then there is \(\overline{a}\in A\) such that \(f(\overline{a}) = x\). But since \([x]\in\rec{B}\) we have \(d(x) = 0\) thus \((d\circ f)(\overline{a}) = 0\) so \([\overline{a}]\in\don{A}\). Finally we have \((\don{A}\to\rec{B})([\overline{a}]) = [f(\overline{a})] = [x]\) so \(\ker(\rec{B}\to\hh{B})\) is contained in \(\im(\don{A}\to\rec{B})\).

    Obviously every element of \(\im(\don{A}\to\rec{B})\) is an element of \(\ker(\rec{B}\to\hh{B})\) since \(\hh{B}\) is a quotient over \(\im(f)\).
  \item Exactness of \(\rec{B}\to\hh{B}\to\rec{D}\).

    Let \([x]\in\ker(\hh{B}\to\rec{D})\) so there is an element \(\overline{a}\in A\) such that \((d\circ f)(\overline{a}) = d(x)\). From this last relation we obtain \(d(x - f(\overline{a})) = 0\) and moreover \(g(x - f(\overline{a})) = 0\) so \(x - f(\overline{a})\in\ker(d)\cap\ker(g)\), thus \([x - f(\overline{a})] \in\rec{B}\). Since \(x - x + f(\overline{a})\in\im(f)\) then \([x] = [x - f(\overline{a})]\) in \(\hh{B}\) so \(\ker(\hh{B}\to\rec{D})\subseteq\im(\rec{B}\to\hh{B})\).

    On the other hand if \([x]\in\im(\rec{B}\to\hh{B})\) then \([x] = [\overline{b}]\) with \(\overline{b}\in\ker(d)\cap\ker(g)\). Now \((\hh{B}\to\rec{D})([\overline{b}]) = [d(\overline{b})] = 0\) so \([x]\in\ker(\hh{B}\to\rec{D})\).

  \end{enumerate}
\end{proof}

\begin{remark}
  \label{rem:other_salamander}
  We can prove, in a totally analogous way, a ``vertical'' version of the Salamander Lemma. That is: a diagram like
  \begin{center}
    \begin{tikzcd}
      C \ar[r, ""] & A \ar[d, "f"] & {}\\
      {} & B \ar[r, ""] & D
    \end{tikzcd}
  \end{center}
  in a double complex leads to an exact sequence
  \begin{center}
    \begin{tikzcd}[sep=small]
      \don{C} \ar[r, ""] & \vh{A} \ar[r, ""] & \don{A} \ar[r, ""] & \rec{B} \ar[r, ""] & \vh{B} \ar[r, ""] & \rec{D}
    \end{tikzcd}.
  \end{center}
  We use the name ``Salamander Lemma'' to refer to either the vertical of the horizontal version of the result as we see fit.
\end{remark}

The following two corollaries of the Salamander Lemma give sufficient conditions for the extramural and intramural maps of Definitions \ref{def:intramural_maps} and \ref{def:extramural_maps} to be isomorphisms. This will then allow us to connect far homologies in a double complex.

\begin{corollary}
  \label{coro:extramural_isomorphisms}
  If \(\fun{f}{A}{B}\) is a horizontal arrow in a double complex and \(\hh{A}=0,\hh{B}=0\) then the extramural map \(\don{A}\to\rec{B}\) is an isomorphism. Similarly if \(f\) is vertical and \(\vh{A}=0,\vh{B}=0\) then the extramural map is an isomorphism.
\end{corollary}

\begin{proof}
  We prove only the horizontal case. By the Salamander Lemma
  \begin{center}
    \begin{tikzcd}
      \rec{C} \ar[r, ""] & \hh{A} \ar[r, ""] & \don{A} \ar[r, ""] & \rec{B} \ar[r, ""] & \hh{B} \ar[r, ""] & \rec{D}
    \end{tikzcd}
  \end{center}
  is exact (with \(C\) and \(D\) ``head'' and ``tail'' of the salamander). As it is a subsequence \(\hh{A}\to\don{A}\to\rec{B}\to\hh{B}\) is exact and by applying the hypothesis we obtain that \(0\to\don{A}\to\rec{B}\to0\) is exact. By using Proposition \ref{prop:sequences_with_zero} on the two ``pieces'' of this last sequence we obtain that \(\don{A}\to\rec{B}\) is both a mono and an epi thus iso (Proposition \ref{prop:iso_mono_epi}).
\end{proof}

\begin{corollary}
  \label{coro:intramural_isomorphisms}
  In the following diagrams in a double complex if the marked arrows form an exact sequence then some intramural maps (precisely the ones indicated in the squares) are isomorphisms.
  \begin{center}
    \begin{tabular}{cccc}
    \begin{tikzcd}[sep=small]
      0 \ar[r, ""] & A \ar[r, ""] \ar[d, ""] & \bullet \ar[d, ""]\\
      0 \ar[r, "\circ" marking] & B \ar[r, "\circ" marking] & \bullet
    \end{tikzcd}& \(\implies\) &
    \begin{tikzcd}[sep=small]
      \rec{A} \ar[d, ""] \ar[r, "\cong"] & \hh{A} \ar[d, ""]\\
      \vh{A} \ar[r, "\cong"] & \don{A}
    \end{tikzcd} & \((1)\)\\
    \begin{tikzcd}[sep=small]
      0 \ar[d, ""] & 0 \ar[d, "\circ" marking]\\
      A \ar[d, ""] \ar[r, ""] & B \ar[d, "\circ" marking]\\
      \bullet \ar[r, ""] & \bullet
    \end{tikzcd}& \(\implies\) &
    \begin{tikzcd}[sep=small]
      \rec{A} \ar[d, "\cong"] \ar[r, ""] & \hh{A} \ar[d, "\cong"]\\
      \vh{A} \ar[r, ""] & \don{A}
    \end{tikzcd} & \((2)\)
    \end{tabular}
  \end{center}
\end{corollary}

\begin{proof}
  We prove the result only in case \((1)\) as the other can be proved in the same way reversing the role of rows and colums or the direction of the arrows.

  By exactness of the marked pair we have \(\hh{B} = 0\), moreover \(\hh{0}=0\) and so, via Corollary \ref{coro:extramural_isomorphisms}, \(0 = \don{0} \cong \rec{B}\). Now applying the Salamander Lemma to the diagram
  \begin{center}
    \begin{tikzcd}
      0 \ar[d, ""] & \\
      0 \ar[r, ""] & A \ar[d, ""]\\
      & B
    \end{tikzcd}
  \end{center}
  we obtain that
  \begin{center}
    \begin{tikzcd}[sep=small]
      \don{0} \ar[r, ""] & \hh{0} \ar[r, ""] & \don{0} \ar[r, ""] & \rec{A} \ar[r, ""] & \hh{A} \ar[r, ""] & \rec{B}
    \end{tikzcd}
  \end{center}
  is exact. Thus \(0\to\rec{A}\to\hh{A}\to\rec{B} = 0\) is exact so the intramural map \(\rec{A}\to\hh{A}\) is an isomorphism by Proposition \ref{prop:sequences_with_zero}. The second isomorphism is obtained in the same way, applying the Salamander Lemma to the diagram
  \begin{center}
    \begin{tikzcd}
      0 \ar[r, ""] & A \ar[d, ""]&\\
      {} & B \ar[r, ""] & \bullet
    \end{tikzcd}.
  \end{center}
\end{proof}

\begin{lemma}
  \label{lemma:nn}
  Consider the complex in Diagram \ref{diagram:nn} where all rows and all columns but the first ones are exact. Then \(\hh{A_{0k}}\cong\vh{A_{k0}}\). The result holds analogously in a complex with zeroes on the bottom and the right.
\end{lemma}

\begin{figure}[h]
  \begin{center}
    \begin{tikzcd}
      {} & 0 \ar[d, ""] & 0 \ar[d, ""] & 0 \ar[d, ""] & {}\\
      0 \ar[r, ""] & A_{00} \ar[r, ""] \ar[d, ""] & A_{01} \ar[r, ""] \ar[d, ""] & A_{02} \ar[r, ""] \ar[d, ""] & \cdots\\
      0 \ar[r, ""] & A_{10} \ar[r, ""] \ar[d, ""] & A_{11} \ar[r, ""] \ar[d, ""] & A_{12} \ar[r, ""] \ar[d, ""] & \cdots\\
      0 \ar[r, ""] & A_{20} \ar[r, ""] \ar[d, ""] & A_{21} \ar[r, ""] \ar[d, ""] & A_{22} \ar[r, ""] \ar[d, ""] & \cdots\\
      {} & \vdots & \vdots & \vdots & {}
    \end{tikzcd}
  \end{center}
  \caption{}
  \label{diagram:nn}
\end{figure}

\begin{proof}
  We shall prove only that \(\hh{A_{00}}\cong\vh{A_{00}}\) and \(\hh{A_{01}}\cong\vh{A_{10}}\) since, after that, it will be obvious how the result generalizes to the other pairs. The intramural maps
  \begin{center}
    \begin{tikzcd}
      & \don{A_{00}}&\\
      \hh{A_{00}} \ar[ru, ""]& & \vh{A_{00}} \ar[lu, ""]
    \end{tikzcd}
  \end{center}
  are isomorphisms by Corollary \ref{coro:intramural_isomorphisms} so \(\hh{A_{00}}\cong\vh{A_{00}}\). Consider now the following diagram of extramural and intramural maps.
  \begin{center}
    \begin{tikzcd}
                  & \don{A_{01}} \ar[dr, ""] &              & \don{A_{10}} \ar[dl, ""] & \\
      \hh{A_{01}} \ar[ru, ""]&              & \rec{A_{11}} &              & \vh{A_{10}} \ar[lu, ""]
    \end{tikzcd}
  \end{center}
  Using the hypothesis of exactness of rows and columns together with Corollaries \ref{coro:extramural_isomorphisms} and \ref{coro:intramural_isomorphisms} we obtain that every map considered is an isomorphism thus \(\hh{A_{01}}\cong\vh{A_{10}}\).
\end{proof}

\begin{lemma}[\(3\times 3\) Lemma]
  \label{lemma:33}
  Consider Diagram \ref{diagram:33} where everything commutes and suppose all columns and all rows but the first are exact. Then the first row is exact as well. Analogously in the case with zeroes at the right and bottom of the diagram if all the columns and the first two rows are exact then so is the third.
\end{lemma}

\begin{figure}[h]
  \begin{center}
    \begin{tikzcd}
      {} & 0 \ar[d, ""] & 0 \ar[d, ""] & 0 \ar[d, ""]\\
      0 \ar[r, ""] & A_{00} \ar[r, ""] \ar[d, ""] & A_{01} \ar[r, ""] \ar[d, ""] & A_{02} \ar[d, ""]\\
      0 \ar[r, ""] & A_{10} \ar[r, ""] \ar[d, ""] & A_{11} \ar[r, ""] \ar[d, ""] & A_{12} \ar[d, ""]\\
      0 \ar[r, ""] & A_{20} \ar[r, ""] & A_{21} \ar[r, ""] & A_{22} \\
    \end{tikzcd}
  \end{center}
  \caption{}
  \label{diagram:33}
\end{figure}

\begin{proof}
  By exactness of all the columns the arrows from the first row to the second are all mono. By application of Proposition \ref{prop:precomposition_with_mono} the first row is a chain complex so the whole diagram is a double complex. Now by Lemma \ref{lemma:nn} the homologies of the first row are isomorphic to those of the first column, that are all 0 because the column is exact by hypothesis. It follows that the first row is exact.
\end{proof}

\begin{remark}
  \label{rem:lemma_nn}
  Looking at Lemma \ref{lemma:nn} and at the proof of the \(3\times3\) Lemma it's obvious that there is a \(4\times 4\) Lemma, a \(5\times 5\) Lemma and more generally a \(n\times n\) Lemma for any \(n\in\mathbb{N}\).
\end{remark}

\begin{lemma}[Snake Lemma]
  \label{lemma:snake}
  Consider the Diagram \ref{diagram:snake} where the two central rows are exact, the two central squares commute, the \(K_n\)s are kernels and the \(C_n\)s are cokernels. Then there is an exact sequence
  \begin{equation*}
    K_1\to K_2\to K_3\to C_1\to C_2\to C_3
  \end{equation*}
\end{lemma}

\begin{figure}[h]
  \begin{center}
    \begin{tikzcd}
      & K_1 \ar[d, ""] \ar[r, ""] & K_2 \ar[d, ""] \ar[r, ""] & K_3 \ar[d, ""] & \\
      & X_1 \ar[d, ""] \ar[r, ""] & X_2 \ar[d, ""] \ar[d, phantom, ""{coordinate, name=Z}] \ar[r, ""] & X_3 \ar[d, ""] \ar[r, ""] & 0\\
      0 \ar[r, ""] & Y_1 \ar[d, ""] \ar[r, ""] & Y_2 \ar[d, ""] \ar[r, ""] & Y_3 \ar[d, ""] &\\
      & C_1 \ar[r, ""] \arrow[from=uuurr, rounded corners, crossing over, dashed, to path={ -- ([xshift=2.5ex]\tikztostart.east) |- (Z) [near end]\tikztonodes -| ([xshift=-2.5ex]\tikztotarget.west) -- (\tikztotarget)}] & C_2 \ar[r, ""] & C_3 &
    \end{tikzcd}
  \end{center}
  \caption{}
  \label{diagram:snake}
\end{figure}

\begin{proof}
  First of all notice that the maps between the kernels are just the appropriate factorizations; so, by adding zeroes around everything we obtain a double complex. The three columns of Diagram \ref{diagram:snake} are obviously exact.

  By Corollaries \ref{coro:extramural_isomorphisms} (exactness of the columns and of the two central rows) and \ref{coro:intramural_isomorphisms} (see Diagram \ref{diagram:extra2}) the following chain of extramutal and intramural isomorphisms is really a chain of isomorphisms and thus we obtain exactness at \(K_2\).
  \begin{equation*}
    \hh{K_2}\cong\don{K_2}\cong\rec{X_2}\cong\don{X_1}\cong\rec{Y_1}\cong\don{0}=0
  \end{equation*}
  Similarly exactness at \(C_2\) is obtained.

  Now we need to find a map \(K_3\to C_1\) that makes the sequence exact at \(K_3\) and \(C_1\). This means equivalently that we need an isomorphism between \(\coker(K_2\to K_3) = \hh{K_3}\) and \(\ker(C_1\to C_2)= \hh{C_1}\). Such an isomorphism is provided by the following chain of intramural and extramural maps that are really isomorphisms by Corollaries \ref{coro:intramural_isomorphisms} (refer again to Diagram \ref{diagram:extra2}) and \ref{coro:extramural_isomorphisms} (exactness of the columns and the two central rows).
  \begin{equation*}
    \hh{K_3}\cong\don{K_3}\cong\rec{X_3}\cong\don{X_2}\cong\rec{Y_2}\cong\don{Y_1}\cong\rec{C_1}\cong\hh{C_1}
  \end{equation*}

  \begin{figure}[h]
    \begin{center}
      \begin{tikzcd}[sep=small]
        0 \ar[d, ""] & 0 \ar[d, ""]\\
        K_2 \ar[d, ""] \ar[r, ""] & K_3 \ar[d, ""]\\
        X_2 \ar[r, ""] & K_3
      \end{tikzcd}\hspace{2cm}
      \begin{tikzcd}[sep=small]
        0 \ar[d, ""] & 0 \ar[d, ""]\\
        K_3 \ar[d, ""] \ar[r, ""] & 0 \ar[d, ""]\\
        X_3 \ar[r, ""] & 0
      \end{tikzcd}\hspace{2cm}
      \begin{tikzcd}[sep=small]
        0 \ar[r, ""] & C_1 \ar[d, ""] \ar[r, ""] & C_2 \ar[d, ""]\\
        0 \ar[r, ""] & 0 \ar[r, ""] & 0
      \end{tikzcd}
    \end{center}
    \caption{}
    \label{diagram:extra2}
  \end{figure}
\end{proof}


\newpage
\begin{thebibliography}{9}

\bibitem{handbook1}
  Francis Borceux,
  \textit{Handbook of categorical algebra 1},
  Cambridge University Press,
  1st edition,
  1994.

\bibitem{handbook2}
  Francis Borceux,
  \textit{Handbook of categorical algebra 2},
  Cambridge University Press,
  1st edition,
  1994.

\bibitem{catwork}
  Saunders Mac Lane,
  \textit{Categories for the Working Mathematician},
  Springer,
  second edition,
  1997.

\bibitem{salaman}
  George M. Bergman,
  \textit{On diagram-chasing in double complexes},
  \href{https://arxiv.org/abs/1108.0958}{arXiv:1108.0958},
  2011.

\end{thebibliography}


\end{document}
